\documentclass[UTF8]{beamer}
\usepackage{ctex}

% other packages
\usepackage{latexsym,amsmath,xcolor,multicol,booktabs,calligra}
\usepackage{graphicx,pstricks,listings,stackengine}
\usepackage{xcolor}
\usepackage{hyperref}
\usepackage{ragged2e} % For better text justification
\usepackage{etoolbox} % For conditional formatting

% Theme customization (optional)
\usetheme{Madrid}
\usefonttheme{professionalfonts}

%设置中文的定义、定理、命题
\newtheorem{proposition}{命题}
\newtheorem{remark}{注}

%%=================设置时间和日期=======================
\date{}
\title{基于时间离散和空间离散的两类随机微分方程数值格式比较}
\author{}
\institute{}


\begin{document}
	
	\section{解析求解SDE的困难}
	
	
		\begin{frame}{随机微分方程 (SDE) 总结}
			\scriptsize
			\resizebox{\textwidth}{!}{% 自动调整表格宽度至文本宽度
				\centering
				\begin{tabular}{@{}lllll@{}}
					\toprule
					\textbf{名称} & \textbf{SDE 形式} & \textbf{解析解} & \textbf{技巧} & \textbf{分布类型} \\
					\midrule
					几何布朗运动 (GBM) & $dX_t = \mu X_t dt + \sigma X_t dW_t$ & $X_t = X_0 e^{(\mu - \frac{1}{2}\sigma^2)t + \sigma W_t}$ & Itô 引理 + 指数化 & 对数正态 \\
					\addlinespace
					Ornstein–Uhlenbeck (OU) & $dX_t = -\theta X_t dt + \sigma dW_t$ & $X_t = X_0 e^{-\theta t} + \sigma \int_0^t e^{-\theta(t-s)} dW_s$ & 积分因子法 & 高斯分布 \\
					\addlinespace
					% ... 填写其他行 ...
					\bottomrule
				\end{tabular}%
			}
			\vspace{2mm}
			\tiny
			\begin{remark}
				几何布朗运动(GBM):是 Black–Scholes 模型的核心......
			\end{remark}
		\end{frame}
	
	
	\begin{frame}
		
	\end{frame}
	
	
	
	\begin{frame}{常见可解析 SDE 一览表(含解法与分布)}
		\scriptsize % 或者使用 \tiny
		
		$$
		\begin{array}{|c|c|c|c|c|}
			\hline
			\textbf{名称} & \textbf{SDE 形式} & \textbf{解析解} & \textbf{技巧} & \textbf{分布类型} \\
			\hline
			\text{几何布朗运动 (GBM)} &
			dX_t = \mu X_t dt + \sigma X_t dW_t &
			X_t = X_0 e^{(\mu - \frac{1}{2}\sigma^2)t + \sigma W_t} &
			\text{Itô 引理 + 指数化} &
			\text{对数正态} \\
			\hline
			\text{Ornstein–Uhlenbeck (OU)} &
			dX_t = -\theta X_t dt + \sigma dW_t &
			X_t = X_0 e^{-\theta t} + \sigma \int_0^t e^{-\theta(t-s)} dW_s &
			\text{积分因子法} &
			\text{高斯分布} \\
			\hline
			\text{Cox–Ingersoll–Ross (CIR)} &
			dX_t = \kappa(\theta - X_t) dt + \sigma \sqrt{X_t} dW_t &
			\text{显式分布,不显式路径} &
			\text{特殊函数法} &
			\text{非中心卡方} \\
			\hline
			\text{Bessel 过程} &
			dX_t = \frac{n - 1}{2X_t} dt + dW_t &
			X_t = \| B_t \| \text{ in } \mathbb{R}^n &
			\text{几何解释} &
			\text{贝塞尔} \\
			\hline
			\text{对数布朗运动} &
			dX_t = \mu dt + \sigma dW_t,\quad X_0 = \log S_0 &
			S_t = S_0 \exp(X_t) &
			\text{变量替换法} &
			\text{对数正态} \\
			\hline
			\text{线性扩散} &
			dX_t = (aX_t + b)dt + (cX_t + d)dW_t &
			\text{根据参数结构可解} &
			\text{Itô 引理 + 拉姆珀提变换} &
			\text{视结构而定} \\
			\hline
		\end{array}
		$$
		
		\begin{remark}
			几何布朗运动(GBM):是 Black–Scholes 模型的核心,最经典的显式解。
			OU 过程:均值回归过程,用于建模利率、速度等,具有高斯平稳性。
			CIR 过程:用于利率与波动率模型,不能为负,解涉及非中心卡方分布,但不能得到显式路径公式。
			Bessel 过程:是多个独立布朗运动的模长,常用于物理中径向扩散建模。
			Lamperti 变换:将非恒定扩散系数的 SDE 变换为常系数扩散。
		\end{remark}
		
	\end{frame}
	
	
	\section{数值解的存在唯一性 }
	
	\begin{frame}{欧拉法}
		欧拉折线法与阿尔泽拉-阿斯科利(Arzela-Ascoli)定理
	\end{frame}
	
	\begin{frame}[allowframebreaks]{Conditions for Existence and Uniqueness}
		
		Consider an $n$-dimensional stochastic differential equation:
		\begin{equation}
			dX_t = \mu(X_t) \, dt + \sigma(X_t) \, dW_t, \label{eq:sde}
		\end{equation}
		where:
		\begin{itemize}
			\item $dW_t$ is an $n$-dimensional white noise,
			\item $\mu: \mathbb{R}^n \to \mathbb{R}^n$ is the drift coefficient,
			\item $\sigma: \mathbb{R}^n \to \mathbb{R}^{n \times n}$ is the diffusion coefficient.
		\end{itemize}
		
		The following conditions ensure the existence and uniqueness of solutions:
		\begin{itemize}
			\item \textbf{Global Lipschitz condition}:
			\[
			|\mu(x) - \mu(y)| + |\sigma(x) - \sigma(y)| \leq K_1 |x - y|.
			\]
			\item \textbf{Linear growth condition}:
			\[
			|\mu(x)|^2 + |\sigma(x)|^2 \leq K_2 (1 + |x|^2).
			\]
		\end{itemize}
		Under these conditions, the SDE \eqref{eq:sde} has a unique strong solution.
		
	\end{frame}
	
	\begin{frame}{Numercial approximation}
		
		Most SDEs cannot be solved analytically, and numerical methods are used to approximate their solutions. Two main approaches are {\color{blue}time-discretization schemes} and  {\color{blue}space-discretization schemes}.
	\end{frame}
	
	
	
	
	\section{空间离散方法的平均驻留时间分析}
	
	\begin{frame}{空间离散格式}
 		\subsection{连续时间随机游走方法}
		
		因此,SDE (2.1.2) 的弱解可以通过求解 Fokker-Planck 方程 (2.1.3) 或 Kolmogorov 方程 (2.1.4) 来获得。本文考虑利用这一联系来设计 SDE 的数值解法。
		
		基本思想是使用数值偏微分方程理论,构造 SDE 的无穷小生成元的近似形式。我们通过离散化 SDE 的生成元 $\mathcal{L}$,构造一个离散状态空间上的生成元 $Q$,其形式如下:
		
		$$
		Qf(x) = \sum_{i=1}^K q(x, y_i(x)) \left[ f(y_i(x)) - f(x) \right] 
		$$
		
		在这里,我们引入了一个反应(或跳跃)率函数 $q : \Omega \times \Omega \to [0, \infty)$,以及 $K$ 个反应通道(或跳跃状态)$y_i(x) \in \Omega$,对于每个 $x \in \Omega$,$1 \leq i \leq K$。这一术语源自化学动力学。
		
		令 $h$ 为空间步长参数,$p$ 为一个正实数,用以设定该方法的逼近阶数。
		
		我们要求无穷小生成元 $Q$ 满足以下两个条件:
		
		(Q1) 局部 $p$ 阶精度:
		
		$$
		Qf(x) = \mathcal{L}f(x) + \mathcal{O}(h^p), \quad \forall x \in \Omega
		$$
		
		(Q2) 可实现性(Realizability):
		
		$$
		q(x, y) \geq 0, \quad \forall x, y \in \Omega, \, x \ne y
		$$
		
		我们将在后文中具体说明 (2.1.5) 中函数 $f$ 的类别,以保证 (Q1) 条件的成立。条件 (Q1) 是任何偏微分方程空间离散化所需满足的基本要求。如果该条件成立,并且数值方案稳定,则近似方案在有限时间区间上具有 $p$ 阶精度(见如定理 4.5.1 中的具体表述)。
		
		相比之下,条件 (Q2) 并非常规 PDE 离散化中的标准要求,它来源于对 Kolmogorov 方程解的概率解释。简言之,(Q2) 要求 (2.1.5) 中出现的有限差分权重为非负值。我们将满足条件 (Q2) 的空间离散称为“可实现的”,因为它能够诱导一个连续时间马尔可夫跳跃过程。考虑到 $\mathcal{L}$ 本身也诱导了一个马尔可夫过程,这一条件对于 $Q$ 而言是自然的。
		
		然而,与 $\mathcal{L}$ 不同,$Q$ 所诱导的过程可以通过“随机模拟算法”(Stochastic Simulation Algorithm, SSA) 在时间上进行精确模拟。该模拟无需对状态空间进行显式网格化,因为所需输入(如跳跃率函数和反应通道)可以在模拟过程中即时计算。如果不满足 (Q2),那么所得的近似过程仅仅是一个常规的数值 PDE 解法,因此将受到“维数灾难”的限制,仅适用于低维系统——关于这一点将在 §2.6 中进一步展开。
		
		此外,我们强调,在 (2.1.5) 中的“反应通道”可以设计成使跳跃位移(即 $y_i - x$)在 $1 \leq i \leq K$ 中是\textbf{统一有界}的(这里的 $K$ 是通道的数量)。
		
		设 $X$ 是由 $Q$ 所诱导的马尔可夫过程。为了强调精确模拟 $X$ 的过程非常简单,我们简要回顾 SSA 的工作机制:
		
		算法 2.1:随机模拟算法(SSA) 
		
		给定当前时间 $t$、当前状态 $X(t) = x$,以及在该状态下评估得到的 $K$ 个跳跃率 $\{ q(x, y_i(x)) \}_{i=1}^K$,算法输出在时间 $t + \delta t$ 时系统的新状态 $X(t + \delta t)$,分两步进行:
		
		(第1步):生成一个参数为
		
		$$
		\lambda(x) = \sum_{i=1}^K q(x, y_i(x))
		$$
		
		的指数分布随机变量,从而获得时间增量 $\delta t$。
		
		(第2步):根据如下概率更新状态:
		
		$$
		\mathbb{P}(X(t+\delta t) = y_i(x) \mid X(t) = x) = \frac{q(x, y_i(x))}{\lambda(x)}, \quad 1 \leq i \leq K.
		$$
		
		我们强调,给定 $X(t) = x$,上述两个步骤中生成的随机时间 $\delta t$ 与生成的新状态 $X(t+\delta t)$ 是相互独立的。
		
		由算法 2.1 可以得到生成元 $Q$ 的一种简洁表达:
		
		$$
		Qf(x) = \lambda(x) \, \mathbb{E}[f(\xi_x) - f(x)] 
		$$
		
		其中期望是对一个随机变量 $\xi_x \in \Omega$ 取的,$\xi_x$ 的取值在反应通道集合 $\{ y_i(x) \mid 1 \leq i \leq K \}$ 中,其概率分布为:
		
		$$
		\mathbb{P}(\xi_x = y_i(x)) = \frac{q(x, y_i(x))}{\lambda(x)}, \quad 1 \leq i \leq K.
		$$
		
		
		如我们将在第四章中所展示的,这种近似过程的上述性质大大简化了理论分析。
		
		接下来,我们将讨论该近似过程的状态空间结构,并构造满足条件 (Q1) 和 (Q2) 的无穷小生成元。
		
		
		
		总而言之,本文提出通过一个可以使用算法 2.1 进行模拟的马尔可夫过程,对一般 SDE 的解进行(弱)近似。该过程的更新机制包括:在(步骤一)中生成一个随机时间增量,在(步骤二)中触发一个反应。
		
		通过将(步骤二)中触发的反应限制在距离当前状态 $x$ 有界距离范围内的反应通道,该近似在数值上通常是稳定的,这一点我们将在 §4.2 中详细说明。
		
		由(步骤一)可知,平均时间增量(或称平均保持时间)为 $\lambda(x)^{-1}$。因此,$\mathcal{L}$ 的空间离散形式决定了该过程在任一状态上所停留的平均时间。
		
		此外,请注意该过程满足:对所有 $s \in [t, t + \delta t)$,都有 $X(s) = x$。这一性质意味着,该过程的样本路径在时间上是右连续且具有左极限的(即具有 càdlàg 性质)。
		
		还应注意,该过程仅通过跳跃改变其状态,因此它是一个纯跳跃过程。
		
		由于时间和状态的更新仅依赖于当前状态 $X(t_0)$,且(步骤一)中的保持时间服从指数分布,因此该近似具有马尔可夫性,并适应于其所依赖噪声的自然滤过。
		
		因此,该近似是一个马尔可夫跳跃过程。
		
		如我们将在第四章展示的,这些近似过程的性质简化了我们的分析。
		
		
	
	\end{frame}
	
	\begin{frame}[allowframebreaks]{Asymptotic Analysis of Mean Holding Time in 1D}
		
		Space-discretization scheme(Bou-Rabee and Vanden-Eijnden, 2018).
		
		\cite{bou2018continuous}Simulate the evolution of SDEs based on the space-discretization Kolmogorov equation and random walk.
		
		The time lapse between two grid points: \( Y(0) = x_i \) and \( Y(t^e) = x_{i-1} \) where \( x_i > x_{i-1} \gg 0 \) satisfies
		
		\[ t^e = \int_{x_{i-1}}^{x_i} \frac{dx}{|\mu(x)|}. \]
		
		For simplicity, we assume that the spacing between grid points is uniform: \( \delta x_i^+ = \delta x_i^- = \delta x \). By using integration by parts and the mean value theorem, observe that can be written as:
		
		\[ t^e = \frac{\delta x}{|\mu(x_i)|} - \int_{x_{i-1}}^{x_i} (x - x_{i-1})\mu(x)^{-2}\mu'(x)dx \]
		
		\[ = t^* - \frac{1}{2} \frac{\mu'(\xi)}{\mu(\xi)^2} \delta x^2 \]
		
		for some \( \xi \in (x_{i-1}, x_i) \), and where we have introduced: \( t^* = \delta x/|\mu_i| \). 
		
		
		\[ t^u = ((Q_u)_{i,i+1} + (Q_u)_{i,i-1})^{-1} = \frac{\delta x^2}{2 + |\mu_i| \delta x} \]
		
		This expression can be rewritten as
		
		\[ t^u = t^* - t^* \frac{2}{2 + |\mu_i| \delta x}. \]
		
		we see that \( t^u \) approaches \( t^* \) as \( |x_i| \) becomes large, as the next Proposition states.
		
		\begin{proposition}
			For any \( \delta x > 0 \), the mean holding time of \( \tilde{Q}_u \) satisfies:
			
			\[ \frac{|t^u - t^*|}{t^*} \to 0 \quad \text{as } |x_i| \to \infty. \]
		\end{proposition}
		
		Likewise, if the second term decays faster than the first term, then the relative error between \( t^e \) and \( t^* \) also tends to zero, and thus, the estimate predicted by \( Q_u \) for the mean holding time asymptotically agrees with the exact mean holding time. 
		
		The second term  decays sufficiently fast if, e.g., the leading order term in \( \mu(x) \) is of the form \(-a x^{2p+1}\) for \( p \geq 0 \) and \( a > 0 \).
		
		Repeating these steps for the mean holding time predicted by \(\tilde{Q}_c\) yields:
		
		\[t^c = ((Q_c)_{i,i+1} + (Q_c)_{i,i-1})^{-1} = \frac{\delta x^2}{2} \text{sech} \left( \mu_i \frac{\delta x}{2} \right).\]
		
		It follows from this expression that even though \(Q_c\) is a second-order accurate approximation to \(L\), it does not capture the right asymptotic mean holding time, as the next Proposition states.
		
		\begin{proposition}
			
			For any \(\delta x > 0\), the mean holding time of \(\tilde{Q}_c\) satisfies:
			
			\[\frac{|t^c - t^*|}{t^*} \to 1 \quad \text{as } |x_i| \to \infty.\]
			
		\end{proposition}
		
		Simply put, the mean holding time of \(Q_c\) converges to zero too fast. 
		
	\end{frame}
	
	
	\begin{frame}[allowframebreaks]{Zu的工作:引进 \(\tilde{Q}^u\) 并且与精确时间\(t^e\) 比较}
		
		\(Q_c\)方案具有二阶近似精度,但其平均保持时间的渐近行为较差,且不适应小噪声情况。
		
		\(Q_u\)方案可以克服这些困难,但仅具有一阶近似精度。泊松过程可以被近似为确定性过程的漂移加上布朗运动,{\color{red}\(Q_u\)方案可能会因泊松近似SDEs漂移项会导致非零方差,而产生人为的扩散效应}.寻找一种更有效的\(Q\)方法变得迫切
		
		在\cite{zu2023random}中,提出了一种改进的\(Q_u\)方案\(\tilde{Q}_u\)方案,通过减少公式\eqref{5.1}中的\(sigma\) 来补偿额外的人为扩散项,
		
		注意在\cite{zu2023random}中的SDE形式为 
		
		\begin{equation}
			dX_t = \mu(X_t)dt + \sigma(X_t)dW, \quad X_t(0) \in \Omega, \tag{5.1} \label{5.1}
		\end{equation}
		
		因此此时的\(M_{ii}(x) = \frac{1}{2}\sigma_{ii}^2(x)\)
		
		改进的$\tilde{Q}_u$格式定义如下:
		
		\begin{align}
			\tilde{Q}_u f(x) &= \sum_{i=1}^{n} \left[\frac{\mu_i(x) \vee 0}{h_i(x)} + \frac{M_{ii}^+(x)}{h_i(x) h_i^+(x)}\right] (f(x + h_i^+(x) e_i) - f(x)) \nonumber \\
			&+ \left[\frac{-\mu_i(x) \wedge 0}{h_i(x)} + \frac{M_{ii}^-(x)}{h_i(x) h_i^-(x)}\right](f(x - h_i^-(x) e_i) - f(x)).
		\end{align}
		
		其中
		\[
		M_{ii}^+(x) = \frac{1}{2} \left( \sigma_{ii}^2(x) - |\mu_i(x)| h_i^+(x) \right) \vee 0, \quad
		M_{ii}^-(x) = \frac{1}{2} \left( \sigma_{ii}^2(x) - |\mu_i(x)| h_i^-(x) \right) \vee 0.
		\]
		
		这个公式通过调整扩散项 \( \sigma \),使得$\tilde{Q}_u$格式在漂移项中补偿了泊松近似带来的误差。漂移项的人为扩散效应在小噪声情况下尤为显著,因此$\tilde{Q}_u$格式在小噪声的场景下显著提高了精度
		
		Eric没有进行\( t^u \)与精确时间\( t^e \)之间的比较,Zu提出改进格式,并且与精确时间\( t^e \)比较
		
		\begin{theorem}
			Assume that \(|\mu(x)|\) is large enough and  
			\[\frac{\mu'(x)}{\mu(x)^2} \sim o\left(\frac{1}{\mu(x)}\right) \text{ as } |x| \to 0.\]  
			For any \(h > 0\), the relative error between \(t^e\) and  \( \tilde{t}^u \) satisfies:  
			\[\frac{| \tilde{t}^u  - t^e|}{t^e} \sim O\left(\frac{\mu'(x)}{\mu(x)}\right) \to 0, \quad \text{as } |x| \to \infty,\]  
			where \(t^e\) is the time \(X(t)\) takes to move a fixed distance \(h\) from \(x\) to \(x-h\).
		\end{theorem}
		
		
	\end{frame}
	
	
	
	\begin{frame}[allowframebreaks]{新工作:\(t^u\)与 \(t^e\)比较,\(t^u\)与 \( {\tilde{t}^u} \) 比较}
		
		\begin{theorem}
			\[ \frac{|t^u - t^e|}{|t^e|} \sim O\left( \frac{\mu'(x)}{\mu(x)} \right) \to 0, \quad \text{as } |x| \to \infty \]
			而且\[ \lvert t^u - t^e  \rvert > \lvert {\tilde{t}^u} -t^e \rvert \]
		\end{theorem}
		
		改进后的\( \tilde{Q}_u \)格式优于\( Q_u \)格式
	\end{frame}
	
	
	% 推导过程
	
	%  对于  \( \tilde{Q}_u \)格式
	%  \[
	%  \frac{|t^u - t^e|}{t^e} = \left| \frac{h}{|\mu_i|} - \frac{h}{|m_i|} + \frac{1}{2} \frac{u'(s)}{u(s)^2} h^2 \right|
	%  \]
	
	%  \[
	%  \left| \frac{h}{|\mu_i|} - \frac{1}{2} \frac{u'(s)}{u(s)^2} h^2 \right|
	%  \]
	
	%  \[
	%  = \frac{\left| \frac{1}{2} \frac{u'(s)}{u(s)^2} h^2 \right|}{\frac{h}{|u(x)|} - \frac{1}{2} \frac{u'(s)}{u(s)^2} b^2}
	%  \]
	
	%  \[
	%  \frac{\sqrt{\frac{h}{2}} \frac{u(x)}{u(x)} + \frac{h^2}{4} \frac{u(x)^2}{u(x)^2}}{\frac{h}{|u(x)|} - \frac{1}{2} \frac{u(x)}{u(s)^2} h^2}
	%  \]
	
	% 对于  \( Q_u \)格式
	
	%  \[
	%  t^u = \frac{h}{\mu(x)} - \frac{h}{2 + |\mu(x)| h}
	
	%  t^e = \frac{h}{|\mu(x)|} - \frac{1}{2} \frac{\mu'(x)}{\mu(x)^2} h^2
	
	%  t^u - t^e = -\frac{1}{2} \frac{\mu'(x)}{\mu(x)^2} h^2 \quad \text{where} \quad \mu(x) = -x
	
	%  \left| \frac{t^u - t^e}{t^e} \right| = \left| \frac{\frac{h}{|\mu(x)|} - \frac{h}{2 + |\mu(x)| h} - \frac{1}{2} \frac{\mu'(x)}{\mu(x)^2} h^2}{t^e} \right|
	
	%  \approx -\frac{h}{|\mu(x)|^2} - \frac{1}{2} \frac{\mu'(x)}{\mu(x)^2} h^2
	
	%  \quad \text{with} \quad \mu(x) = -x^3 \quad \text{and as} \quad |\mu(x)| \to \infty
	
	%  \]
	
	
	
	
	
	\section{Time-discretization Scheme}
	
	\subsection{Backgrounds}
	
	\begin{frame}[allowframebreaks]{Background on SDE Numerical Methods}
		
		\justifying % For better text justification
		
		It is well known that under the Lipschitz condition and linear growth condition, an SDE has a unique strong solution and the EM approximation has a strong convergence rate of order \( \frac{1}{2}\) (see Kloeden and Platen \cite{kloeden1992stochastic}). 
		
		The Euler-Maruyama method performs poorly here due to the locally Lipschitz drift. The Markov chain diverges \cite{mattingly2002ergodicity}
		
		
		%------------------------------------------------------------
		
		
		% Hutzenthaler证明在有限时间,即使漂移系数和扩散系数均为 \( C^1 \) 函数(导数无界,非全局Lipschitz),显式EM方法的矩也可能在有限时间内发散
		
		% 证明了一大类具有超线性增长系数函数的SDE中,欧拉近似在强\( L^p \) 意义和数值弱意义上的差异在有限时间内发散至无穷
		
		% In \cite{hutzenthaler2011strong} Hutzenthaler prove for a large class of stochastic differential equations with non-globally Lipschitz continuous coefficients that Euler’s  approximation converges neither in the strong mean square sense nor in the numerically weak sense to the exact solution at a finite time point. Even worse, the difference of the exact solution and of the numerical approximation at a finite time point diverges to infinity in the strong mean square sense and in the numerically weak sense.
		
		
		Without the linear growth condition, the explicit Euler scheme may not converge to the exact solution of an SDE in the strong mean square sense. Even worse, \cite{hutzenthaler2011strong} showed that the moments of the standard EM approximate solution at a finite time may diverge to infinity even if the true solution is finite. 
		
		%------------------------------------------------------------
		% \begin{frame}[allowframebreaks]
			%     \frametitle{Higham文章}
			
			%     Higham等人通过限制精确解和数值解的p阶矩有界性,证明了在局部Lipschitz条件和线性增长条件下数值解的强收敛性,将SDE数值模拟的强均方收敛理论扩展到全局Lipschitz问题之外。给出了在向量场局部Lipschitz(如 \( C^1 \) )且矩有界时EM方法的强收敛定理。这种分析方式在能建立矩界时非常有用,既适用于EM方法,也适用于其他可证明"接近"EM的方法。通常对 \( f, g \in C^1\)  的显式方法何时能期望矩有界尚不明确
			
			% 在这篇随机微分方程(SDE)数值分析的论文中,"矩有界"(bounded moments)是一个核心概念,其数学含义和重要性如下:
			
			% 对于随机过程 \( y(t) \) 或其数值近似 \( X_k \),\( p \)阶矩有界是指:
			% \[
			% \mathbb{E}\left[ \sup_{0 \leq t \leq T} |y(t)|^p \right] < \infty \quad \text{或} \quad \mathbb{E}\left[ \sup_k |X_k|^p \right] < \infty
			% \]
			% 其中 \( p \geq 1 \),\( \mathbb{E} \) 表示期望,\( |\cdot| \) 为向量范数。
			
			
			% 论文中矩有界假设的关键点:
			
			% 收敛性前提:  
			%   要求精确解 \( y(t) \) 和数值解 \( \overline{X}(t) \) 的 \( p \)阶矩(\( p>2 \))有界,才能证明欧拉-丸山(EM)方法的强收敛性。
			
			% 稳定性保障  
			%   对隐式方法(如SSBE),通过单边Lipschitz条件(假设3.1)和多项式增长条件(假设4.1)主动证明矩有界性(引理3.7),从而避免对显式方法难以验证的矩假设。
			
			% 收敛速率优化(定理4.4):  
			%   若进一步要求所有 \( p \geq 1 \) 的矩有界(假设4.2),则可提升收敛速率至最优阶 \( O(\Delta t) \)。
			
			
			% 物理意义:矩有界意味着解(或数值解)的轨迹不会以高概率发生极端爆炸性增长。例如:
			%    \( p=2 \)(二阶矩):控制能量或方差不发散。
			%   \( p \to \infty \):几乎必然一致有界。
			
			
			%   漂移项 \( f \) 和扩散项 \( g \) 的增长行为(如多项式增长、单边Lipschitz条件)直接影响解矩的有界性。例如:
			%   若 \( f \) 满足 \( \langle f(y), y \rangle \leq \alpha + \beta |y|^2 \),则可通过Itô公式证明矩有界(引理3.2)。
			%   若 \( g \) 是全局Lipschitz的,扩散项对矩的控制更直接。
			
			
			%   强收敛性证明(如定理2.2)依赖局部误差的全局控制,矩有界确保误差积累不会因解的爆炸而失效。
			
			%   显式EM方法在非全局Lipschitz条件下可能无法保证矩有界(导致发散),而隐式方法(如SSBE、BE)通过隐式结构天然抑制爆炸性增长
			
			%   许多实际SDE模型(如金融中的Heston模型、生物化学中的反应扩散方程)具有非全局Lipschitz系数,矩有界分析扩展了数值方法的适用范围。
			
			% "矩有界"本质反映了随机微分方程及其数值解在有限时间内的稳定性,是连接SDE理论性质和数值实践的关键桥梁
			
			% \end{frame}
		Many implicit methods have therefore been proposed to study the numerical solutions of SDEs with nonlinear coefficients. For example, Higham, Mao and Stuart \cite{higham2002strong} proved that the implicit EM numerical solutions converge strongly to the exact solutions of SDEs with the globally one-side Lipschitz continuous drift term and globally Lipschitz continuous diffusion term.
		
		%------------------------------------------------------------
		
		\framebreak % This will create a continuation slide if needed
		
		On the other hand, some explicit methods have also been proposed for nonlinear SDEs.
		
		For example, \cite{hutzenthaler2012strong} proposed the tamed EM schemes to approximate SDEs with the global Lipschitz diffusion coefficient and one-sided Lipschitz drift coefficient, whose numerical solutions converge strongly to the exact solution with order \( \frac{1}{2} \). 
		
		Moreover, the tamed Milstein \cite{wang2013tamed} and the stopped EM method \cite{liu2013strong} as well as their variants have also been proposed to deal with the strong converge problem for nonlinear SDEs. 
		
		Recently, Mao \cite{mao2015truncated}, \cite{mao2016convergence} proposed a new explicit method called the truncated EM method for nonlinear SDEs and established the strong convergence and obtained the convergence rate under the local Lipschitz condition and the Khasminskii-type condition. The authors of \cite{hu2018convergence} showed that the truncated EM method may enable to use a larger stepsize than the tamed Euler method in \cite{sabanis2016euler} to achieve the same error.
		
	\end{frame}
	
	
	\subsection{我们的问题}
	
	\begin{frame}{Time-discentization}
		
		The \textbf{Euler-Maruyama method} uses a time step size $\Delta$, with $t_k = t_0 + k\Delta$. The approximation is given by:
		\[
		\hat{X}_{t_{k+1}}^\Delta = \hat{X}_{t_k}^\Delta + \mu(\hat{X}_{t_k}^\Delta)\Delta + \sigma(\hat{X}_{t_k}^\Delta)(W_{t_{k+1}} - W_{t_k}),
		\]
		where $W_{t_{k+1}} - W_{t_k} \sim \mathcal{N}(0, \Delta)$ are independent Gaussian random vectors.
		
		\subsection{1D Cubic Oscillator with Additive Noise}
		Consider the stochastic differential equation:
		\[
		dX = -X^3 dt + \sigma dW, \quad X(0) \in \mathbb{R}.
		\]
		The solution is geometrically ergodic with a stationary distribution density:
		\[
		\nu(x) = Z^{-1} \exp\left(-\frac{x^4}{2\sigma^2}\right), \quad \text{where} \quad Z = \int_{\mathbb{R}} \exp\left(-\frac{x^4}{2\sigma^2}\right) dx.
		\]
		
		
		The Euler-Maruyama method performs poorly here due to the locally Lipschitz drift. The Markov chain $\{\hat{X}_{n\Delta}\}$ diverges \cite{mattingly2002ergodicity}:
		\[
		\mathbb{E}\left[\left(\hat{X}_{\lfloor t/\Delta \rfloor \Delta}\right)^2\right] \to \infty \quad \text{as} \quad t \to \infty.
		\]
		
		
	\end{frame}
	
	
	\begin{frame}{我们的问题}
		
		\cite{higham2002strong} indicated that many important SDE models satisfy only a local Lipschitz property and,
		since Brownian paths can make arbitrarily large excursions, the global Lipschitz-based theory is not  directly relevant. \cite{higham2002strong} (实践中,许多重要的随机微分方程(SDE)模型仅满足局部Lipschitz条件,并且由于布朗运动路径可能会进行任意大的偏移,基于全局Lipschitz条件的理论并不直接适用)
		
		当 \( x \to \infty \)时, 空间离散化的Mean holding time 和时间离散化方法中,固定空间距离,使用的时间 两者之间的比较;
		
		{\color{red}传统的欧拉方法在漂移场较大时通常表现较差.}考虑当时间离散的Euler格式, 由于布朗运动的性质,可能会出现 \( (W_{t_{k+1}} - W_{t_k}) \to \infty\)的情况,注意到此时由于我们使用的离散格式, \[
		\hat{X}_{t_{k+1}}^{\Delta} = \hat{X}_{t_{k}}^{\Delta} + \mu(\hat{X}_{t_{k}}^{\Delta}) \Delta  + \sigma(\hat{X}_{t_{k}}^{\Delta}) (W_{t_{k+1}} - W_{t_k}),
		\],这样会使得 \( \hat{X}_{t_{k+1}}^{\Delta} \to \infty \),造成数值解进入极端不稳定区域(导致数值解在正负无穷之间振荡) ,当然在连续情况不会发生这种情况,考虑\eqref{equation:2},当\( dW_t \to \infty\),方程会迅速将轨线拉回到0附近
		
	\end{frame}
	
	
	
	
	\subsection{改进的explicit方法}
	
	\begin{frame}[allowframebreaks]{Tamed EM method}
		
		Huzenthaler在 \cite{hutzenthaler2012strong}提出Tamed 方法
		
		\begin{equation}
			X^{\Delta}_{t_{k+1}} = X^{\Delta}_{t_k} + \frac{\Delta \cdot \mu( X^{\Delta}_{t_k} )}{1 + \Delta \| \mu( X^{\Delta}_{t_k} )\|} + \sqrt{2} \cdot \sqrt{\Delta} \cdot N(0,1)
		\end{equation}
		
		当 \(|X^{\Delta}_{t_k}| < M\),此时对于 \(\Delta < \varepsilon \),
		
		\[
		\frac{\Delta \cdot \mu(X^{\Delta}_{t_k})}{1 + \Delta \|\mu(X^{\Delta}_{t_k})\|} \rightarrow 0
		\]
		
		若 \(\forall M >0, X^{\Delta}_{t_k} > M \),
		
		\[\frac{\Delta \cdot \|\mu(X^{\Delta}_{t_k})\|}{1 + \Delta \|\mu(X^{\Delta}_{t_k})\|} \to 1 \]
		
		
		一旦出现小概率事件,不会出现振荡.对于我们的方程,此时有
		\[
		X^{\Delta}_{t_{k+1}} -X^{\Delta}_{t_k} = \frac{\Delta \cdot \lvert {-X^{\Delta}_{t_k} \rvert }^3}{1 + \Delta \cdot \lvert {-X^{\Delta}_{t_k} \rvert }^3} + \sqrt{2} \cdot \sqrt{\Delta} \cdot N(0,1)
		\]
		于是我们有,对于 \( h > 0 \)
		
		\[
		t^\delta = \frac{h}{\mathbb{E}[X^{\Delta}_{t_{k+1}} -X^{\Delta}_{t_k}]} \sim O(h)
		\]
		
		同样我们可以得到,如果M足够大,我们有 \( t^e \to 0, t^{\delta} \) fixed 
		
		
		于是我们有如下结果,
		
		\begin{theorem}
			\[
			\frac{\lvert  t^{\delta} - t^e  \rvert}{t^e} \to \infty
			\]
		\end{theorem}
	\end{frame}
	
	
	
	\begin{frame}[allowframebreaks]{Truncated EM Method}
		
		Mao在\cite{mao2015truncated}新的显式方法——截断EM方法,其核心思想是通过截断技术控制系数增长,从而保证数值解的收敛性
		
		我们的方程满足Mao的方法的使用条件,
		
		Let \( u(x) = x^3 \), then \( u^{-1}(x) = x^{\frac{1}{3}} \).
		
		Let \( h(x) = x^{-\frac{1}{4}} \), then \( u^{-1}(h(x)) = (x^{-\frac{1}{4}})^{\frac{1}{3}} = x^{-\frac{1}{12}} \).
		
		If \( |x| \to \infty \), \( |x| \wedge u^{-1}(h(\Delta)) = \Delta^{-\frac{1}{12}} \).
		
		As \( |x| \) becomes smaller, it approaches \( u^{-1}(h(\Delta)) \).
		
		At this point, 
		\[ X^\Delta_{t_{k+1}} - X^\Delta_{t_k} = -(\Delta^{-\frac{1}{12}}) \cdot \Delta + \sqrt{2} \cdot \sqrt{\Delta} \cdot N(0,1) \]
		\[ = -\Delta^{-\frac{3}{4}} + \sqrt{2} \cdot \sqrt{\Delta} \cdot N(0,1). \]
		
		Taking expectation:
		\[ \mathbb{E}[X^\Delta_{t_{k+1}} - X^\Delta_{t_k}] = -\Delta^{-\frac{3}{4}} + \sqrt{2} \cdot \sqrt{\Delta}. \]
		
		When \( |x| \to \infty \), i.e., \( |x| > M \) in numerical terms,
		the expected value of the time driven by \( h \) is:
		\[ t^{\Delta} = \frac{h}{\mathbb{E}[X^\Delta_{t_{k+1}} - X^\Delta_{t_k}]} = \frac{h}{-\Delta^{-\frac{3}{4}} + \sqrt{2} \cdot \sqrt{\Delta}} \cdot \Delta \sim O(h \cdot \Delta^{\frac{1}{4}}) \]
		
		\begin{note}
			First choose \( \Delta \) and \( h \), then choose \( M \) (from the equality condition outside the truncation).
		\end{note}
		
		对于空间离散格式,我们有
		\[ t^e =  t^* - \frac{1}{2} \frac{\mu'(\xi)}{\mu(\xi)^2} \delta x^2 \]
		
		如果M足够大,我们有 \( t^e \to 0, t^{\Delta} \) fixed 
		
		
		于是我们有如下结果,
		
		\begin{theorem}
			\[
			\frac{\lvert  t^{\Delta} - t^e  \rvert}{t^e} \to \infty
			\]
		\end{theorem}
		
		
	\end{frame}
	
	
	
	\section{符号说明}
	
	\begin{frame}{符号说明}
		
		\( \mu(X_t) \) \quad  漂移项
		
		\(\sigma(X_t)\) \quad 扩散项
		
		\( \delta x = h\) \quad 空间离散步长
		
		\(\Delta\) \quad 时间离散步长
		
		\( Q_u\) \quad Eric提出的有限差分格式
		
		\( Q_c\) \quad Eric提出的有限体积格式
		
		\( \tilde{Q}_u \) \quad Zu提出的改进的有限差分格式
		
		\( t^e \)  \quad 当 \( \lvert x \rvert \to \infty\)时,精确的mean holding time
		
		\( t^* \) \quad  \( t^e \)的主要部分
		
		\( t^u \)  \quad \( Q_u\)格式的mean holding time
		
		\( t^c \)  \quad \( Q_c\)格式的mean holding time
		
		\( \tilde{t}^u \) \quad  \( \tilde{Q}_u \) 格式的mean holding time
		
		\( t^\Delta\) \quad  截断方法运动h距离所需要的平均时间
		
		\( t^\delta\) \quad  tamed方法运动h距离所需要的平均时间
		
	\end{frame}
	
	
	% Reference slide (optional)
	\begin{frame}[allowframebreaks]
		\frametitle{References}
		\bibliographystyle{IEEEtran}
		\bibliography{references}
	\end{frame}
	
\end{document}

