\documentclass[UTF8]{ctexbeamer}

\usetheme{Madrid}
\usecolortheme{default}
\setbeamertemplate{navigation symbols}{}

\usepackage{amsmath,amssymb,amsfonts}
\usepackage{graphicx}
\usepackage{bm}
\usepackage{mathtools}

% 图像路径
\graphicspath{{fig/}}

\title[时间离散 vs 空间离散]{基于时间离散和空间离散的两类随机微分方程数值格式比较}
\author[华光辉]{华光辉}
\institute[东北师范大学]{东北师范大学~数学与统计学院}
%\date[2026年5月]{2026年5月}
\date{2025年12月}

%\titlegraphic{%
%	\includegraphics[height=1.4cm]{fig/xiaohui.png}\hspace{1em}%
%	\includegraphics[height=1.4cm]{fig/xiaoming.png}%
%}

\begin{document}
	
	%--------------------------------------
	\begin{frame}
		\titlepage
	\end{frame}
	
	%--------------------------------------
	\begin{frame}{报告提纲}
		\tableofcontents
	\end{frame}
	
	%======================================
	\section{引言}
	%======================================
	
	\begin{frame}{研究背景与问题动机}
		\begin{itemize}
			\item 随机微分方程(SDE)
			\[
			dX_t = \mu(X_t)\,dt + \sigma(X_t)\,dW_t,\quad X_0 = x_0,
			\]
			在金融、物理、生物、化学、神经科学等众多领域中刻画含噪动力学行为。
			\item 实际模型中漂移 $\mu$、扩散 $\sigma$ 往往仅满足局部 Lipschitz 条件,甚至具有超线性增长,经典 Euler--Maruyama 格式可能发散。
			\item 数值模拟 SDE 的两大途径:
			\begin{itemize}
				\item 时间离散:基于时间步长 $\Delta t$ 的改进 Euler--Maruyama 类方法;
				\item 空间离散:基于生成元的连续时间随机游走(CTRW)方法。
			\end{itemize}
			\item 问题:在给定的 SDE 模型下,时间离散与空间离散哪一种更适合刻画长时间行为、逃逸概率和平均首达时间?
		\end{itemize}
	\end{frame}
	
	%--------------------------------------
	\begin{frame}{时间离散与空间离散:两类典型方法}
		\begin{itemize}
			\item \textbf{时间离散方法}
			\begin{itemize}
				\item Euler--Maruyama 方法:$1/2$ 阶强收敛、1 阶弱收敛,在全局 Lipschitz 情形下表现良好;
				\item Tamed Euler、truncated Euler--Maruyama 等改进方法:通过驯服或截断漂移/扩散,恢复非全局 Lipschitz 情形下的强收敛性。
			\end{itemize}
			\item \textbf{空间离散方法}
			\begin{itemize}
				\item Bou-Rabee \& Vanden-Eijnden 提出的 CTRW/生成元离散方法;
				\item 在空间网格上离散无穷小生成元 $L$ 得到稀疏矩阵 $Q$,再用 SSA 等算法模拟跳跃过程。
			\end{itemize}
			\item 本文聚焦:
			\begin{itemize}
				\item 固定空间距离跨越时间(MFPT)的比较;
				\item 随机 Canard 快--慢系统中逃逸概率和 MFPT 的比较;
				\item 在误差--代价(work--error)意义下系统比较两类方法的数值表现。
			\end{itemize}
		\end{itemize}
	\end{frame}
	
	%--------------------------------------
	\begin{frame}{本文研究内容与主要贡献}
		\begin{itemize}
			\item 在一维立方振子模型中,引入漂移主导时间$t^e$ 作为基准,比较:
			\begin{itemize}
				\item 空间离散 CTRW 平均驻留时间 $t^u$;
				\item 驯服/截断 EM 格式下的 MFPT $t^\delta,t^\Delta$。
			\end{itemize}
			\item \理论贡献
			\begin{itemize}
				\item 证明在强漂移极限下,$t^u$ 与 $t^e$ 的渐近一致性以及多项式漂移 $\mu(x)=-a x^{2p+1}$ 情形下的相对误差估计;
				\item 对驯服 Euler 与截断 EM,给出固定空间跨越时间 MFPT 的渐近展开与误差估计;
				\item 基于生成元理论,对随机 Canard 系统中逃逸概率和 MFPT的空间离散/时间离散收敛性进行分析。
			\end{itemize}
			\item 数值贡献
			\begin{itemize}
				\item 在一维立方振子与随机 Canard 系统上,系统比较两类方法在路径几何、逃逸概率、MFPT 以及 work--error 曲线上的表现。
			\end{itemize}
		\end{itemize}
	\end{frame}
	
	%======================================
	\section{预备知识}
	%======================================
	
	\begin{frame}{It\^o 扩散与无穷小生成元}
		\begin{itemize}
			\item 考虑 $\mathbb R^d$ 上的 It\^o 扩散
			\[
			dX_t = \mu(X_t)\,dt + \sigma(X_t)\,dW_t,\qquad X_0=x.
			\]
			\item 对 $f\in C^2(\mathbb R^d)$,无穷小生成元
			\[
			(Lf)(x)
			= \sum_i \mu_i(x)\,\partial_i f(x)
			+ \frac12\sum_{i,j}a_{ij}(x)\partial_{ij} f(x),
			\quad a = \sigma\sigma^{\mathsf T}.
			\]
			\item 对应 Kolmogorov 后向/前向方程刻画:
			\begin{itemize}
				\item 逃逸概率、committor 函数等弱量;
				\item 平稳分布、MFPT 等长期统计量。
			\end{itemize}
		\end{itemize}
	\end{frame}
	
	%--------------------------------------
	\begin{frame}{空间离散:连续时间随机游走(CTRW)}
		\begin{itemize}
			\item 在空间网格 $\{x_i\}$ 上离散生成元 $L$,得到 $Q$-矩阵
			\[
			(Qf)_i = \sum_j q_{ij} f(x_j),
			\quad q_{ij}\ge 0,~\sum_{j\neq i}q_{ij} = -q_{ii}.
			\]
			\item 得到在有限状态空间上的 $Q$-过程 $X_t$:
			\begin{itemize}
				\item 状态 $x_i$ 的跳出速率 $\lambda_i = -q_{ii}$;
				\item 驻留时间服从 $\mathrm{Exp}(\lambda_i)$,然后按 $q_{ij}/\lambda_i$ 跳转。
			\end{itemize}
			\item 定理(已有):$Q$-过程的无穷小生成元
			\begin{equation*}
				(\mathcal A f)(i) = \lim_{t\to 0^+}\frac{\mathbb E^i[f(X_t)]-f(i)}{t}
				= \sum_j q_{ij} f(j).
			\end{equation*}
			因此 $Q$ 正是 $X_t$ 的无穷小生成元。
		\end{itemize}
	\end{frame}
	
	%--------------------------------------
	\begin{frame}{时间离散:驯服/截断 Euler--Maruyama 格式}
		\begin{itemize}
			\item \textbf{驯服 Euler--Maruyama}(tamed EM):
			\[
			X_{k+1}
			= X_k + \frac{\mu(X_k)}{1+\delta |\mu(X_k)|}\,\delta
			+ \sigma(X_k)\Delta W_k.
			\]
			\item \textbf{截断 Euler--Maruyama}(truncated EM):
			\begin{itemize}
				\item 在球 $B(0,h(\Delta))$ 内截断漂移/扩散,使其有界且满足全局 Lipschitz;
				\item 对截断后的系数应用标准 EM。
			\end{itemize}
			\item 在 Khasminskii 型条件下,两者都能在非全局 Lipschitz 情形下达到
			\[
			\Big(\mathbb E |X_{t_k} - X_k|^p\Big)^{1/p} \le C\,\Delta^{1/2},
			\]
			即保持 $1/2$ 阶强收敛。
		\end{itemize}
	\end{frame}
	
	%--------------------------------------
	\begin{frame}{本章小结:理论基础}
		\begin{itemize}
			\item It\^o 扩散的无穷小生成元 $L$ 为构造 PDE 与生成元离散提供了统一框架;
			\item CTRW/$Q$-过程将连续状态 SDE 转化为有限状态的马尔可夫跳跃过程;
			\item 驯服/截断 EM 保证在局部 Lipschitz 且超线性漂移下仍具有 $1/2$ 阶强收敛;
			\item 为后续比较\alert{固定空间跨越时间}以及\alert{随机 Canard 系统的逃逸概率与 MFPT} 提供了理论基础。
		\end{itemize}
	\end{frame}
	
	%======================================
	\section{固定空间距离跨越时间的比较}
	%======================================
	
	\begin{frame}{模型与顺漂移设定}
		\begin{itemize}
			\item 一维加性噪声 SDE
			\[
			dX_t = \mu(X_t)\,dt + \sigma\,dW_t,\qquad X_0=x,\quad |x|\gg 1.
			\]
			\item 选取\alert{立方振子}作为典型模型:
			\[
			\mu(x) = -x^3,\qquad \sigma>0.
			\]
			\item 研究从 $x$ 向左跨越固定距离 $\delta$ 所需时间:
			\[
			L = x-\delta,\qquad 0<\delta\ll x,\ \mu(x)<0.
			\]
			\item 当漂移方向与目标一致时(\alert{顺漂移}),跨越时间由漂移主导,可由常微分方程
			\[
			\dot X_t = \mu(X_t),\ X(0)=x
			\]
			给出漂移主导时间 $t^{e}$。
		\end{itemize}
	\end{frame}
	
	%--------------------------------------
	\begin{frame}{漂移主导时间与空间离散平均驻留时间}
		\begin{itemize}
			\item 漂移主导时间
			\[
			t^{e} = \int_{x-\delta}^{x}\frac{ds}{|\mu(s)|},
			\]
			在 $\delta\ll x$ 下展开得到
			\[
			t^{e}
			= \frac{\delta}{|\mu(x)|}
			- \frac{\mu'(x)}{2\mu(x)^2}\,\delta^2 + \mathcal O(\delta^3).
			\]
			\item CTRW 中,状态 $x_i$ 的跳出速率
			\[
			\lambda(x_i) \approx \frac{|\mu(x_i)|}{h},
			\quad \Rightarrow\quad
			\mathbb E S_i = \frac1{\lambda(x_i)} \approx \frac{h}{|\mu(x_i)|}.
			\]
			\item \alert{平均驻留时间} $t^u$ 与 $t^{e}$ 的比较是评价空间离散在强漂移极限下时间刻画能力的关键。
		\end{itemize}
	\end{frame}
	
	%--------------------------------------
	\begin{frame}{\alert{定理:空间离散平均驻留时间的渐近一致性}}
		\begin{theorem}
			假设 $|\mu(x)|$ 足够大且
			\[
			\frac{\mu'(x)}{\mu(x)^2}
			\sim o\!\left(\frac{1}{\mu(x)}\right),
			\quad |x|\to\infty.
			\]
			则对任意 $h>0$,空间离散平均驻留时间 $t^u$ 与漂移主导时间 $t^{e}$ 满足
			\[
			\frac{|t^u - t^{e}|}{t^{e}}
			\sim
			O\!\left(\frac{\mu'(x)}{\mu(x)}\right)
			\xrightarrow[|x|\to\infty]{} 0.
			\]
		\end{theorem}
		\vspace{1ex}
		\begin{itemize}
			\item 说明:在强漂移区域,\alert{CTRW 平均驻留时间与解析漂移时间同阶},网格步长 $h$ 固定时误差随 $|x|$ 增大而衰减。
		\end{itemize}
	\end{frame}
	
	%--------------------------------------
	\begin{frame}{\alert{定理:多项式漂移下的误差估计}}
		\begin{theorem}
			假设
			\[
			\mu(x) = -a x^{2p+1},\quad p\ge 0,\ a>0,
			\]
			记漂移主导时间主部
			\(
			t^* = \delta/|\mu(x)|,
			\)
			则对任意 $h>0$,有
			\[
			\frac{|t^u - t^*|}{t^*}
			\xrightarrow[|x|\to\infty]{} 0.
			\]
		\end{theorem}
		\vspace{1ex}
		\begin{itemize}
			\item 对立方振子 $\mu(x)=-x^3$,空间离散在强漂移区域能\alert{高精度再现}局部跨越时间的主导标度。
		\end{itemize}
	\end{frame}
	
	%--------------------------------------
	\begin{frame}{MFPT 与连续模型的 PDE 表述}
		\begin{itemize}
			\item 对一维 SDE 若以区间端点为吸收边界,平均首达时间 $m(x)$ 满足常微分方程
			\[
			\mu(x) m'(x) + \frac{\sigma^2}{2}m''(x) = -1,
			\]
			并配以适当的边界条件(吸收/反射)。
			\item 漂移主导时间 $t^{e}$ 可看作该 PDE 解在强漂移极限下的主导项。
			\item 空间离散/CTRW 对应在网格上求解线性方程
			\[
			Q m = -\mathbf 1,
			\]
			是 MFPT PDE 的\alert{有限维近似}。
			\item 驯服/截断 EM Monte Carlo 则通过时间步进模拟路径,统计首达时间样本得到 $t^\delta,t^\Delta$。
		\end{itemize}
	\end{frame}
	
	%--------------------------------------
	\begin{frame}{\alert{定理:驯服 Euler 方法的 MFPT 渐近与误差}}
		\begin{theorem}[驯服 Euler 方法的 MFPT 渐近与误差]
			设 $\mu\in C^2$ 在 $x$ 的邻域内单调,且 $\mu(x)<0$,$\sigma$ 有界。
			在极限 $x\to\infty,\ \delta/x\to 0,\ \Delta\to 0$ 下,驯服 EM 下从 $x$ 向左跨越距离 $\delta$ 的 MFPT $t^\delta$ 满足
			\begin{align*}
				t^\delta
				&= \frac{\delta}{|\mu(x)|}
				+ \delta\,\Delta
				+ \mathcal O\big(\Delta^2\delta\big)
				+ \mathcal O\big(\delta^{2}|\mu|^{-2}|\mu'|\big), \\
				t^\delta - t^{e}
				&= \delta\,\Delta
				- \frac{\mu'(x)}{2\mu(x)^2}\,\delta^2
				+ o(\Delta\delta) + o(\delta^2).
			\end{align*}
			以 $t^{e}$ 的主导项 $\delta/|\mu(x)|$ 计,相对误差满足
			\[
			\frac{t^\delta - t^{e}}{\delta/|\mu(x)|}
			= \Delta |\mu(x)| + \mathcal O\Big(\frac{\delta}{x}\Big).
			\]
		\end{theorem}
	\end{frame}
	
	%--------------------------------------
	\begin{frame}{\alert{定理:截断 EM 方法的 MFPT 渐近与误差}}
		\begin{theorem}[截断 Euler--Maruyama 方法的 MFPT 渐近与误差]
			沿用上面的设定,设截断半径 $h(\Delta)$ 使得在 $[x-\delta,x]$ 内截断生效($x\gg h(\Delta)$)。
			则存在常数 $c>0$ 使得
			\[
			t^{\Delta}
			= \frac{\delta}{|\mu(h(\Delta))|}
			+ \mathcal O(\delta^2),
			\]
			并且
			\[
			t^{\Delta}-t^{e}
			= \delta\!\left(
			\frac{1}{|\mu(h(\Delta))|}
			- \frac{1}{|\mu(x)|}
			\right)
			+ \mathcal O(\delta^2).
			\]
			对立方振子 $\mu(y)=-y^3$,若取常用约束 $\Delta^{1/4}h(\Delta)\le 1$ 并令 $h(\Delta)\asymp\Delta^{-1/4}$,则有
			\[
			t^{\Delta}\asymp \delta\,\Delta^{1/4},\qquad
			\frac{t^{\Delta}}{t^{e}}\asymp x^{3}\Delta^{1/4}.
			\]
		\end{theorem}
	\end{frame}
	
	%--------------------------------------
	\begin{frame}{固定空间跨越时间:理论比较小结}
		\begin{itemize}
			\item 空间离散:在强漂移极限下,平均驻留时间 $t^u$ 与漂移主导时间 $t^{e}$ 渐近一致;
			\item 多项式漂移下,$t^u$ 与主部 $t^*$ 的相对误差随 $|x|$ 增大而衰减;
			\item 驯服 EM:MFPT $t^\delta$ 的相对误差主要由 $\Delta |\mu(x)|$ 控制;
			\item 截断 EM:误差与截断半径 $h(\Delta)$ 强耦合,立方漂移下要求
			\[
			\Delta \lesssim \varepsilon^{4} x^{-12}
			\]
			才能保证给定的相对误差;
			\item 为数值实验中的\alert{误差--代价比较}提供了指导。
		\end{itemize}
	\end{frame}
	
	%--------------------------------------
	\begin{frame}{一维立方振子上的数值实验设置}
		\begin{itemize}
			\item 模型:$dX_t = -X_t^3\,dt + \sigma\,dW_t$,初值 $X_0 = x\gg 1$。
			\item 空间离散:
			\begin{itemize}
				\item 在一维网格上构造 CTRW/SSA,步长 $h$;
				\item 通过生成元 $Q_u$ 的平均驻留时间给出 $t^u$。
			\end{itemize}
			\item 时间离散:
			\begin{itemize}
				\item 使用驯服 EM 和截断 EM,时间步长分别为 $\Delta$;
				\item 通过 Monte Carlo 估计 MFPT $t^\delta,t^\Delta$。
			\end{itemize}
			\item 误差度量:
			\begin{itemize}
				\item 绝对误差 $|t^{\mathrm{num}}(x)-t^{e}(x)|$;
				\item 相对误差以及时间离散/空间离散误差比值
				\[
				R_{\mathrm{tame}}(x),\ R_{\mathrm{trunc}}(x).
				\]
			\end{itemize}
		\end{itemize}
	\end{frame}
	
	%--------------------------------------
	\begin{frame}{图 3.1:平均时间误差比较}
		\begin{center}
			\includegraphics[width=0.48\textwidth]{fig/Absolute_error.png}
			\hfill
			\includegraphics[width=0.48\textwidth]{fig/Error_ratio.png}
		\end{center}
		\vspace{1ex}
		\small
		左:不同方法相对于 $t^{e}(x\to L)$ 的绝对误差;右:时间离散误差与空间离散误差比值
		$R_{\mathrm{tame}}(x),R_{\mathrm{trunc}}(x)$。
		\begin{itemize}
			\item 随着 $x$ 增大,三种方法的平均首达时间均趋近 $t^{e}$,其中\alert{CTRW/SSA 的误差衰减最快}。
			\item 大 $x$ 区域下 $R_{\mathrm{tame}}(x),R_{\mathrm{trunc}}(x)\gg 1$,表明在固定空间跨越距离下,
			\alert{空间离散在强漂移极限下具有明显优势}。
		\end{itemize}
	\end{frame}
	
	%======================================
	\section{随机 Canard 快--慢系统的动力学行为分析}
	%======================================
	
	\begin{frame}{随机 Canard 快--慢系统模型}
		\begin{itemize}
			\item 考虑具有典型 Canard 结构的一类二维快--慢系统
			\[
			\begin{cases}
				dX_t = f(X_t,Y_t)\,dt + \sigma_x\,dW^1_t,\\
				dY_t = \varepsilon g(X_t,Y_t)\,dt + \sigma_y\,dW^2_t,
			\end{cases}
			\]
			其中 $0<\varepsilon\ll 1$。
			\item 相空间结构:
			\begin{itemize}
				\item 存在吸引/排斥慢流形及折叠点;
				\item 轨道沿慢流形缓慢演化,在折叠附近发生\alert{快跳},形成 Canard 轨道。
			\end{itemize}
			\item 噪声引入随机性:
			\begin{itemize}
				\item 快跳时间与位置呈分布;
				\item 慢流形附近存在驻留与穿越的统计结构。
			\end{itemize}
		\end{itemize}
	\end{frame}
	
	%--------------------------------------
	\begin{frame}{逃逸概率与 MFPT:连续模型}
		\begin{itemize}
			\item 在相空间中选取两个不相交区域 $A,B\subset D$:
			\begin{itemize}
				\item $A$:左侧稳态/代谢态;
				\item $B$:右侧目标区域(例如另一代谢态或崩溃态)。
			\end{itemize}
			\item \alert{逃逸概率(committor)}
			\[
			q(z) = \mathbb P^z\{\tau_B < \tau_A\},\quad z\in D\setminus(A\cup B).
			\]
			满足椭圆型边值问题
			\[
			L q = 0\ \text{in }D\setminus(A\cup B),\quad
			q|_A = 0,\ q|_B = 1.
			\]
			\item \alert{平均首达时间(MFPT)}
			\[
			m(z) = \mathbb E^z[\tau_{A\cup B}],
			\]
			满足
			\[
			Lm = -1,\quad m|_{A\cup B}=0.
			\]
			\item 这两个量是描述随机 Canard 系统\alert{转迁概率与时间尺度}的典型弱指标。
		\end{itemize}
	\end{frame}
	
	%--------------------------------------
	\begin{frame}{空间离散下逃逸概率的收敛性}
		\begin{itemize}
			\item 在二维网格 $D_h$ 上离散生成元 $L$,得到 CTRW 生成元 $Q_h$。
			\item 空间离散 committor $q_h$ 满足线性方程
			\[
			Q_h q_h = 0\ \text{on 内点},\quad
			q_h|_A = 0,\ q_h|_B = 1.
			\]
		\end{itemize}
		\begin{theorem}[空间离散下逃逸概率的收敛]
			在适当正则性与一致逼近条件下,存在常数 $C>0$,当 $h>0$ 足够小时,有
			\[
			\max_{z_i\in D_h} |q_h(z_i)-q(z_i)| \le C h^p,
			\]
			其中 $p$ 由局部离散逼近阶决定。对改进的 $\widetilde Q_u$ 方案有 $p=2$,逃逸概率达到二阶空间精度。
		\end{theorem}
	\end{frame}
	
	%--------------------------------------
	\begin{frame}{时间离散下逃逸概率的收敛性}
		\begin{itemize}
			\item 基于驯服/截断 EM 的 Monte Carlo:
			\begin{itemize}
				\item 在网格节点上发射大量样本路径;
				\item 统计从 $z$ 出发首先到达 $B$ 的频率,得到离散逃逸概率 $q_\Delta(z)$。
			\end{itemize}
			\item 在 Khasminskii 型条件下,路径强收敛
			\[
			\max_{0\le k\le N}
			\Big(\mathbb E |X_{t_k}-X_k^\Delta|^p\Big)^{1/p}\le C\Delta^{1/2}
			\]
			可以推广为对 committor 的弱收敛控制:
			\[
			|q_\Delta(z)-q(z)|
			\lesssim C\,\Delta^{1/2},
			\]
			即误差阶受限于时间步进的强收敛阶。
			\item 这从生成元角度揭示了\alert{空间离散在逃逸概率等长期弱量上具有更高阶的空间精度}。
		\end{itemize}
	\end{frame}
	
	%--------------------------------------
	\begin{frame}{随机 Canard 系统数值实验设置}
		\begin{itemize}
			\item 测试模型:论文中给定的随机 Canard 系统参数组;
			\item 空间离散:
			\begin{itemize}
				\item 在相空间矩形区域上构造规则网格;
				\item 利用 $\widetilde Q_u$ 生成元构造 CTRW,并用一次线性求解得到 $q_h,m_h$。
			\end{itemize}
			\item 时间离散:
			\begin{itemize}
				\item 使用驯服 EM 与截断 EM,时间步长 $\Delta$ 取若干值;
				\item 在所有网格节点上独立发射 Monte Carlo 轨道,估计 $q_\Delta,m_\Delta$。
			\end{itemize}
			\item 工作量(work):
			\begin{itemize}
				\item CTRW:线性方程求解所需的稀疏矩阵操作数;
				\item 时间离散:所有路径的总时间步数。
			\end{itemize}
		\end{itemize}
	\end{frame}
	
	%--------------------------------------
	\begin{frame}{图 4.2:样本路径与快跳时间分布}
		\begin{center}
			\includegraphics[width=0.48\textwidth]{fig/Sample_trajectories_100.png}
			\hfill
			\includegraphics[width=0.48\textwidth]{fig/Fast_jump_time_distribution_100.png}
		\end{center}
		\vspace{1ex}
		\small
		左:三种数值方法在 $T=40$ 内的样本路径;右:对应的快跳时间分布直方图。
		\begin{itemize}
			\item 三种方法均能再现贴靠慢流形、折叠附近快跳的几何结构;
			\item 快跳时间主要集中在有限区间内,各方法的分布区间相互重叠,峰值略有差异;
			\item 在\alert{路径级强误差}指标下,时间离散方法在给定代价下更容易获得平滑轨道。
		\end{itemize}
	\end{frame}
	
	%--------------------------------------
	\begin{frame}{图 4.3:占据度热图}
		\begin{center}
			\includegraphics[width=0.7\textwidth]{fig/Occupancy_100.png}
		\end{center}
		\vspace{1ex}
		\small
		轨道在相空间中的长期驻留分布:左图为驯服 Euler 方法,右图为 CTRW/SSA。
		\begin{itemize}
			\item 占据度热图刻画轨道在不同区域出现的相对频率;
			\item 两种方法均在慢流形附近给出较高占据度,但在折叠附近的\alert{边界层结构}上存在差异;
			\item CTRW 在长时间统计上更接近由生成元刻画的理论结构。
		\end{itemize}
	\end{frame}
	
	%--------------------------------------
	\begin{frame}{图 4.4:逃逸概率与 MFPT 热图(时间离散 vs 空间离散)}
		\begin{center}
			\includegraphics[width=0.45\textwidth]{fig/committor_heatmap.jpg}
			\hfill
			\includegraphics[width=0.45\textwidth]{fig/committor_trunc_heatmap.png}
			
			\vspace{0.8em}
			\includegraphics[width=0.45\textwidth]{fig/mfpt_heatmap.jpg}
			\hfill
			\includegraphics[width=0.45\textwidth]{fig/mfpt_trunc_heatmap.png}
		\end{center}
		\vspace{1ex}
		\small
		上:committor 函数 $q$ 的空间结构;下:MFPT $m$ 的空间结构。左列为 CTRW/SSA,右列为截断 EM Monte Carlo。
		\begin{itemize}
			\item committor 在左侧区域接近 0,在右侧目标区域 $B$ 附近接近 1,在折叠附近形成\alert{概率边界层};
			\item MFPT 在慢流形上呈现沿轨道方向缓慢变化,在快跳后迅速下降;
			\item CTRW 结果整体更平滑,空间结构与 PDE 解更为一致,时间离散结果受限于统计噪声和时间步长。
		\end{itemize}
	\end{frame}
	
	%--------------------------------------
	\begin{frame}{图 4.5:逃逸概率的 work--error 曲线}
		\begin{center}
			\includegraphics[width=0.7\textwidth]{fig/work_error_committor_trunc_vs_ctrw.pdf}
		\end{center}
		\vspace{1ex}
		\small
		截断 EM Monte Carlo 与 CTRW/Qu 在逃逸概率 RMS 误差下的 work--error 比较。
		\begin{itemize}
			\item 在\alert{相同或更小工作量}下,CTRW/Qu 生成元离散通常能提供显著更小的 RMS 误差;
			\item 时间离散方法的误差随 work 单调衰减,但斜率受限于强收敛阶和 Monte Carlo 方差;
			\item 说明在 Canard 系统这类快--慢模型中,\alert{空间离散更适合求解逃逸概率等长期弱指标}。
		\end{itemize}
	\end{frame}
	
	%--------------------------------------
	\begin{frame}{本文提出的主要定理概览}
		\begin{itemize}
			\item 固定空间跨越时间方面:
			\begin{itemize}
				\item 空间离散平均驻留时间 $t^u$ 与漂移主导时间 $t^{e}$ 在强漂移极限下渐近一致;
				\item 对多项式漂移 $\mu(x)=-ax^{2p+1}$,给出 $t^u$ 与主部 $t^*$ 的相对误差估计;
				\item 驯服 Euler 与截断 EM 的 MFPT 渐近展开与步长约束条件。
			\end{itemize}
			\item 随机 Canard 系统方面:
			\begin{itemize}
				\item 基于生成元的逃逸概率 $q_h$、MFPT $m_h$ 的空间收敛性;
				\item 时间离散下 committor 估计的弱收敛阶及其对步长的要求。
			\end{itemize}
			\item 这些定理为后续\alert{work--error 数值比较}提供了理论支撑。
		\end{itemize}
	\end{frame}
	
	%--------------------------------------
	\begin{frame}{时间离散与空间离散方法的特点对比}
		\begin{itemize}
			\item \textbf{时间离散(tamed / truncated EM)}
			\begin{itemize}
				\item 优点:实现简单,易于与现有代码库兼容;适合关注单点或少量初值的路径级分析;
				\item 缺点:步长受限于稳定性与收敛阶;对弱量需要大量 Monte Carlo 以降低方差。
			\end{itemize}
			\item \textbf{空间离散(CTRW / $Q$-矩阵)}
			\begin{itemize}
				\item 优点:直接在生成元层面离散,适合求解全场的 committor 和 MFPT;在弱量上可达到更高空间精度;
				\item 缺点:高维问题中网格数量巨大,需要稀疏线性代数与多重网格等技术支持。
			\end{itemize}
			\item 本文通过\alert{理论定理 + 数值实验}揭示了两类方法在不同指标下的适用场景。
		\end{itemize}
	\end{frame}
	
	%--------------------------------------
	\begin{frame}{数值实验总体设置}
		\begin{itemize}
			\item 测试模型:
			\begin{itemize}
				\item 一维立方振子:强调强漂移极限下的局部跨越时间;
				\item 二维随机 Canard 系统:强调快--慢结构、逃逸概率与 MFPT。
			\end{itemize}
			\item 指标体系:
			\begin{itemize}
				\item 路径几何:相图、快跳时间、占据度;
				\item 弱量:committor 函数、MFPT;
				\item 综合指标:work--error 曲线。
			\end{itemize}
			\item 所有实验中保持\alert{符号与参数选择与论文正文一致},并复用同一组图像文件,便于在答辩中与论文内容一一对应。
		\end{itemize}
	\end{frame}
	
	%======================================
	\section{综合比较与结论}
	%======================================
	
	\begin{frame}{固定空间距离跨越时间上的比较小结}
		\begin{itemize}
			\item 理论上,空间离散平均驻留时间 $t^u$ 与漂移主导时间 $t^{e}$ 在强漂移极限下渐近一致;
			\item 对多项式漂移 $\mu(x)=-ax^{2p+1}$,$t^u$ 相比主部 $t^*$ 的相对误差随 $|x|$ 增大而消失;
			\item 驯服/截断 EM 下的 MFPT $t^\delta,t^\Delta$ 存在显式渐近展开,但相对误差受限于时间步长 $\Delta$ 的选取;
			\item 一维立方振子数值实验表明:在相同空间跨越距离与相近计算代价下,
			\begin{itemize}
				\item \alert{CTRW/SSA 的平均时间估计更稳定、更精确};
				\item 时间离散误差则受限于时间步进与统计方差。
			\end{itemize}
		\end{itemize}
	\end{frame}
	
	%--------------------------------------
	\begin{frame}{随机 Canard 系统上的比较小结}
		\begin{itemize}
			\item 在有限时间窗口和路径几何上:
			\begin{itemize}
				\item 驯服 Euler 与截断 EM 更擅长再现 Canard 轨道的几何细节;
				\item CTRW/SSA 也能保持整体几何结构,但局部平滑性略逊。
			\end{itemize}
			\item 在逃逸概率与 MFPT 等弱/长期指标上:
			\begin{itemize}
				\item CTRW/Qu 通过一次稀疏线性求解即可获得\alert{全场解},空间精度高、噪声小;
				\item 时间离散方法在给定 work 下误差较大,受限于强收敛阶和 Monte Carlo 方差。
			\end{itemize}
			\item work--error 曲线表明:
			\begin{itemize}
				\item 在逃逸概率 RMS 误差固定时,\alert{空间离散往往具有更低的计算代价};
				\item 在路径级强误差指标下,时间离散在同等 work 下优于 CTRW/SSA。
			\end{itemize}
		\end{itemize}
	\end{frame}
	
	%--------------------------------------
	\begin{frame}{本文主要结论}
		\begin{itemize}
			\item 针对具有局部 Lipschitz 漂移的 SDE,系统比较了两类数值离散方式:
			\begin{itemize}
				\item 改进的 Euler--Maruyama 时间离散方法(驯服 EM、截断 EM);
				\item 基于生成元的空间离散 CTRW 方法($Q_u,\widetilde Q_u$)。
			\end{itemize}
			\item 在固定空间跨越时间问题中,证明了:
			\begin{itemize}
				\item 空间离散平均驻留时间与漂移主导时间的渐近一致性;
				\item 驯服/截断 EM 的 MFPT 渐近展开与误差估计。
			\end{itemize}
			\item 在随机 Canard 系统中,建立了:
			\begin{itemize}
				\item 空间离散下逃逸概率的高阶空间收敛;
				\item 时间离散下逃逸概率的弱收敛控制;
				\item 基于 committor/MFPT 的 work--error 比较。
			\end{itemize}
		\end{itemize}
	\end{frame}
	
	%--------------------------------------
	\begin{frame}{展望与进一步工作}
		\begin{itemize}
			\item 将 CTRW/生成元离散方法推广到更高维、带约束或具复杂几何形状的相空间;
			\item 结合自适应网格与自适应时间步长,提高长时间模拟的整体效率;
			\item 探索时间离散与空间离散的\alert{混合策略},例如在关键区域采用高精度空间离散,在其余区域使用时间离散;
			\item 将 committor/MFPT 分析与\alert{机器学习方法}(如 PINN、深度强化学习)结合,用于复杂随机系统的稀有事件分析。
		\end{itemize}
	\end{frame}
	
	%--------------------------------------
	\begin{frame}
		\centering
		\Huge 谢谢大家!
	\end{frame}
	
\end{document}
