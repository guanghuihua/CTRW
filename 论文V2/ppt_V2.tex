\documentclass[aspectratio=169]{beamer}
\usepackage[UTF8]{ctex}
\usepackage{amsmath,amssymb}
\usepackage{graphicx}
\usepackage{booktabs}
\usepackage{hyperref}
\usepackage{cite}
\usepackage{tikz}
\usetikzlibrary{arrows.meta}
\usetheme{Madrid}
\usecolortheme{whale}
\setbeamertemplate{navigation symbols}{} % 隐藏导航符号
\setbeamertemplate{footline}[frame number] % 显示页码
\setbeamerfont{footnote}{size=\tiny}

\title{基于时间离散和空间离散的两类随机微分方程数值格式比较}
\subtitle{硕士毕业论文预答辩}
\author{华光辉}
\institute{东北师范大学数学与统计学院}
\date{2026年5月}

\AtBeginSection[]{
	\begin{frame}
		\frametitle{目录}
		\tableofcontents[currentsection]
	\end{frame}
}

\begin{document}
	
	\begin{frame}
		\titlepage
	\end{frame}
	
	\begin{frame}{目录}
		\tableofcontents
	\end{frame}
	
	\section{研究背景与意义}
	\begin{frame}{研究背景}
		\begin{itemize}
			\item 随机微分方程(SDE)在分子动力学、数理金融、生物系统等领域广泛出现
			\item 对于仅满足局部Lipschitz条件且漂移具有超线性增长的方程,经典Euler–Maruyama格式可能数值发散
			\item 基于无穷小生成元的空间离散方法(如连续时间随机游走)在稳定性和长期统计性质上往往更可靠
			\item \textbf{核心问题}:时间离散与空间离散哪一种更适合刻画给定SDE的长期行为?
		\end{itemize}
	\end{frame}
	
	\begin{frame}{研究意义}
		\begin{itemize}
			\item 在统一框架下比较两类数值格式的精度与计算代价
			\item 为复杂随机动力系统中选择和设计合适的数值格式提供参考
			\item 对具有超线性增长和非全局Lipschitz系数的SDE数值方法研究具有理论价值
		\end{itemize}
	\end{frame}
	
	\section{主要研究内容}
	\begin{frame}{研究内容概述}
		\begin{enumerate}
			\item \textbf{模型选取}:
			\begin{itemize}
				\item 一维立方振子
				\item 随机Canard快-慢系统
			\end{itemize}
			\item \textbf{数值格式}:
			\begin{itemize}
				\item 时间离散:驯服Euler-Maruyama、截断Euler-Maruyama
				\item 空间离散:连续时间游走格式$Q_u$及其改进格式$\widetilde{Q}_u$
			\end{itemize}
			\item \textbf{比较指标}:
			\begin{itemize}
				\item 固定空间跨越距离的时间尺度
				\item 逃逸概率函数(committor function)
				\item 平均首达时间(MFPT)
			\end{itemize}
		\end{enumerate}
	\end{frame}
	
	\section{理论基础}
	\begin{frame}{Itô扩散的无穷小生成元}
		\begin{definition}[Itô扩散的无穷小生成元]
			考虑$n$维SDE:
			\begin{equation*}
				dX_t = \mu(X_t) dt + \sigma(X_t) dW_t
			\end{equation*}
			其无穷小生成元为:
			\begin{equation*}
				Lf(x) = \sum_{i=1}^n \mu_i(x) \frac{\partial f}{\partial x_i} + \sum_{i,j=1}^n M_{ij}(x) \frac{\partial^2 f}{\partial x_i \partial x_j}
			\end{equation*}
			其中$M(x) = \frac{1}{2}\sigma(x)\sigma(x)^T$。
		\end{definition}
	\end{frame}
	
	\begin{frame}{空间离散格式}
		\begin{itemize}
			\item \textbf{Bou-Rabee和Vanden-Eijnden (2018)}提出:
			\begin{itemize}
				\item 有限差分离散化$Q_u$格式
				\item 有限体积离散化$Q_c$格式
			\end{itemize}
			\item \textbf{Zu (2023)改进格式}:$\widetilde{Q}_u$,补偿了额外扩散项
		\end{itemize}
		\begin{figure}[h]
			\centering
			\includegraphics[width=0.8\textwidth]{fig/Q_formats_comparison.png}
			\caption{不同空间离散格式示意图}
		\end{figure}
	\end{frame}
	
	\begin{frame}{改进的时间离散格式}
		\begin{itemize}
			\item \textbf{驯服Euler-Maruyama格式}:
			\begin{equation*}
				X_{k+1} = X_k + \frac{\mu(X_k)\Delta}{1+\|\mu(X_k)\|\Delta} + \sigma(X_k)\Delta W_k
			\end{equation*}
			\item \textbf{截断Euler-Maruyama格式}:
			\begin{equation*}
				X_{k+1} = X_k + \mu_\Delta(X_k)\Delta + \sigma_\Delta(X_k)\Delta W_k
			\end{equation*}
			其中$\mu_\Delta,\sigma_\Delta$为截断后的系数。
		\end{itemize}
	\end{frame}
	
	\section{固定空间距离跨越时间比较}
	\begin{frame}{一维立方振子模型}
		\begin{itemize}
			\item 模型方程:$dX_t = -X_t^3 dt + \sqrt{2} dW_t$
			\item 研究目标:从$x$向左跨越固定距离$\delta$的时间比较
			\item 比较基准:
			\begin{itemize}
				\item 精确漂移时间:$t^e = \int_{x-\delta}^x \frac{ds}{|\mu(s)|}$
				\item 空间离散平均驻留时间:$t^u$
				\item 时间离散平均首达时间:$t^\delta, t^\Delta$
			\end{itemize}
		\end{itemize}
	\end{frame}
	
	\begin{frame}{主要定理(时间比较)}
		\begin{theorem}[空间离散平均驻留时间与精确漂移时间的比较]
			假设$|\mu(x)|$足够大且$\frac{\mu'(x)}{\mu(x)^2} \sim o\left(\frac{1}{\mu(x)}\right)$,则对于任意$h>0$,
			\[
			\frac{|t^u - t^e|}{t^e} \sim O\left(\frac{\mu'(x)}{\mu(x)}\right) \to 0, \quad |x| \to \infty
			\]
		\end{theorem}
		
		\begin{theorem}[驯服Euler方法的MFPT渐近]
			设$\mu\in C^2$且在$x$的邻域内单调并满足$\mu(x)<0$,$\sigma$有界,则
			\[
			t^{\delta} = \frac{\delta}{|\mu(x)|} + \delta\Delta + \mathcal{O}(\Delta^2\delta) + \mathcal{O}(\delta^2|\mu|^{-2}|\mu'|)
			\]
		\end{theorem}
	\end{frame}
	
	\begin{frame}{数值实验结果}
		\begin{figure}[h]
			\centering
			\includegraphics[width=0.9\textwidth]{fig/Absolute_error.png}
			\caption{不同方法相对于$t^e$的绝对误差}
		\end{figure}
	\end{frame}
	
	\begin{frame}{数值实验结果}
		\begin{figure}[h]
			\centering
			\includegraphics[width=0.9\textwidth]{fig/Error_ratio.png}
			\caption{时间离散误差与空间离散误差比值}
		\end{figure}
	\end{frame}
	
	\section{随机Canard系统分析}
	\begin{frame}{随机Canard快-慢系统}
		\begin{itemize}
			\item 模型方程:
			\begin{align*}
				dx_t &= \left(y_t - \left(\frac{x_t^3}{3} - x_t\right)\right) dt + \sigma_x dW_t^{(x)} \\
				dy_t &= \varepsilon(a - x_t) dt + \sigma_y dW_t^{(y)}
			\end{align*}
			\item 参数:$\varepsilon=10^{-2}, a=1.0, \sigma_x=0, \sigma_y=0.08$
			\item 特征:具有典型的Canard现象,在折叠点附近发生快速跃迁
		\end{itemize}
	\end{frame}
	
	\begin{frame}{逃逸概率函数}
		\begin{definition}[逃逸概率]
			给定两个互不相交的集合$A,B\subset D$,逃逸概率函数定义为:
			\[
			q(x) := \mathbb{P}_x(\tau_B < \tau_A)
			\]
			满足椭圆边值问题:
			\[
			\begin{cases}
				L q(x) = 0, & x\in D\setminus(A\cup B) \\
				q(x) = 0, & x\in A \\
				q(x) = 1, & x\in B
			\end{cases}
			\]
		\end{definition}
	\end{frame}
	
	\begin{frame}{空间离散下的收敛性定理}
		\begin{theorem}[空间离散下逃逸概率的收敛]
			在适当正则性假设下,存在常数$C>0$,当$h>0$足够小时,
			\[
			\max_{z_i\in D_h} |q_h(z_i) - q(z_i)| \le C h^p
			\]
			其中$p$为逼近阶,对改进的$\widetilde{Q}_u$格式有$p=2$。
		\end{theorem}
	\end{frame}
	
	\begin{frame}{时间离散下的收敛性}
		\begin{theorem}[时间离散下逃逸概率的收敛]
			在Khasminskii型条件下,驯服Euler和截断Euler-Maruyama方法近似$Z_t^\Delta$满足强收敛:
			\[
			\mathbb{E}\left(\sup_{0\le t\le T}\|Z_t - Z_t^\Delta\|^q\right) \le C\Delta^{q\gamma}, \quad \gamma=\frac{1}{2}
			\]
			相应地,离散逃逸概率$q_\Delta$收敛到$q$,但误差阶不超过$O(\Delta^{1/2})$。
		\end{theorem}
	\end{frame}
	
	\begin{frame}{数值实验:样本轨道}
		\begin{figure}[h]
			\centering
			\includegraphics[width=0.9\textwidth]{fig/Sample_trajectories_100.png}
			\caption{随机Canard系统的典型样本轨道}
		\end{figure}
	\end{frame}
	
	\begin{frame}{数值实验:快跳时间分布}
		\begin{figure}[h]
			\centering
			\includegraphics[width=0.9\textwidth]{fig/Fast_jump_time_distribution_100.png}
			\caption{首次满足$x\ge x_{\text{th}}$的时间直方图}
		\end{figure}
	\end{frame}
	
	\begin{frame}{数值实验:占据度热图}
		\begin{figure}[h]
			\centering
			\includegraphics[width=0.9\textwidth]{fig/Occupancy_100.png}
			\caption{长期驻留分布(占据度热图)}
		\end{figure}
	\end{frame}
	
	\begin{frame}{数值实验:逃逸概率热图}
		\begin{figure}[h]
			\centering
			\includegraphics[width=0.45\textwidth]{fig/committor_heatmap.jpg}
			\includegraphics[width=0.45\textwidth]{fig/committor_trunc_heatmap.png}
			\caption{逃逸概率热图:左为CTRW/$Q$-解,右为截断EM Monte Carlo}
		\end{figure}
	\end{frame}
	
	\begin{frame}{数值实验:MFPT热图}
		\begin{figure}[h]
			\centering
			\includegraphics[width=0.45\textwidth]{fig/mfpt_heatmap.jpg}
			\includegraphics[width=0.45\textwidth]{fig/mfpt_trunc_heatmap.png}
			\caption{MFPT热图:左为CTRW/$Q$-解,右为截断EM Monte Carlo}
		\end{figure}
	\end{frame}
	
	\begin{frame}{work-error曲线比较}
		\begin{figure}[h]
			\centering
			\includegraphics[width=0.8\textwidth]{fig/work_error_committor_trunc_vs_ctrw.pdf}
			\caption{逃逸概率的work-error曲线:空间离散与时间离散对比}
		\end{figure}
	\end{frame}
	
	\section{结论与展望}
	\begin{frame}{主要结论}
		\begin{enumerate}
			\item \textbf{固定空间跨越时间}:在漂移主导区域,空间离散方法对局部时间的刻度更精确,误差随$x$增大快速衰减;时间离散方法的误差受限于时间步长。
			\item \textbf{逃逸概率与MFPT}:对于生成元型弱量,空间离散方法通过一次线性求解即可获得全场解,误差阶可达$O(h^2)$;时间离散方法需要通过Monte Carlo模拟,误差阶不超过$O(\Delta^{1/2})$。
			\item \textbf{计算效率}:在同等精度要求下,空间离散方法通常具有更低的计算代价,特别适用于长期统计性质的计算。
			\item \textbf{适用场景}:时间离散方法在捕捉单条样本轨道和短时动力学方面更具灵活性;空间离散方法在长期统计方面更高效。
		\end{enumerate}
	\end{frame}
	
	\begin{frame}{创新点}
		\begin{itemize}
			\item 提出了固定空间跨越距离的时间比较框架,统一了时间离散与空间离散的性能评估
			\item 在一维立方振子和随机Canard系统上,分别从时间尺度和动力学行为两个角度系统比较了两类方法
			\item 给出了空间离散格式在平均驻留时间、逃逸概率和MFPT上的收敛性理论分析
			\item 通过数值实验验证了理论分析,并提供了误差-代价的定量比较
		\end{itemize}
	\end{frame}
	
	\begin{frame}{未来工作展望}
		\begin{itemize}
			\item 将理论分析推广到高维、退化扩散或具有复杂边界条件的随机系统
			\item 设计更高阶的显式或半隐式时间离散算法,保持强收敛与矩有界性
			\item 研究时间离散与空间离散的自适应混合策略,发挥两类方法的互补优势
			\item 在更多具有代表性的模型(如多稳态反应网络、高维随机梯度系统等)上验证比较框架
		\end{itemize}
	\end{frame}
	
	\section{致谢}
	\begin{frame}{致谢}
		\begin{itemize}
			\item 感谢导师祖建副教授的悉心指导
			\item 感谢评阅专家和答辩委员会各位老师的宝贵意见
			\item 感谢东北师范大学数学与统计学院提供的学习和科研环境
			\item 感谢家人和朋友的支持与鼓励
		\end{itemize}
		\vspace{1cm}
		\centering
		\LARGE 谢谢!
	\end{frame}
	
\end{document}