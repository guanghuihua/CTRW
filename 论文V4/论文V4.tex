%%%%%%%%%%%%%%%%%%%%%%%%%%%%%%%%%%%%%%%%%%%%%%%%%%%%%%%%%%%%%%%%%%%%%%%%%%%%%%%%%%%%%%%%%%%%%%%%%%%%%%%%%
%%%%%%%%%%%%%%%%   东北师范大学硕士学位论文模板(2023版) v1.0alpha
%%%%%%%%%%%%%%%%   DESIGNED BY ZHAO HONGLIANG
%%%%%%%%%%%%%%%%   在 CTeX_3.0.212.1 中使用  pdflatex 编译通过
%%%%%%%%%%%%%%%%   在 texlive 2022 中使用  pdflatex 编译通过
%%%%%%%%%%%%%%%%%%%%%%%%%%%%%%%%%%%%%%%%%%%%%%%%%%%%%%%%%%%%%%%%%%%%%%%%%%%%%%%%%%%%%%%%%%%%%%%%%%%%%%%%%
%%%%%%%%%%%%%%%%   需将校徽和校名图片文件 xiaohui.png 和 xiaoming.png 放在源程序文件夹中
%%%%%%%%%%%%%%%%   在\begin{document} 命令后输入论文相关信息,用以自动生成封面等
%%%%%%%%%%%%%%%%   使用 pdflatex 编译
%%%%%%%%%%%%%%%%   脚注序号为带圈数字,每页最大编号为 10,超过 10 需手动处理
%%%%%%%%%%%%%%%%%%%%%%%%%%%%%%%%%%%%%%%%%%%%%%%%%%%%%%%%%%%%%%%%%%%%%%%%%%%%%%%%%%%%%%%%%%%%%%%%%%%%%%%%%
%%%%%%%%%%%%%%%%
%%%%%%%%%%%%%%%%   导言区
%%%%%%%%%%%%%%%%
%%%%%%%%%%%%%%%%%%%%%%%%%%%%%%%%%%%%%%%%%%%%%%%%%%%%%%%%%%%%%%%%%%%%%%%%%%%%%%%%%%%%%%%%%%%%%%%%%%%%%%%%%
%%=====================================================================================================%%
%%
%%                设置文档类别   和   页面格式
%%
%%=====================================================================================================%%
\documentclass[a4paper,twoside,openany,UTF8]{article}  %A4纸张,双面排版,每章后不留空白页,UTF-8编码
\usepackage[heading=true]{ctex}  %添加中文及版式的支持
\usepackage[total={160mm,257mm},inner=25mm,outer=25mm,top=20mm,includeheadfoot,%
           headheight=15mm,headsep=8mm,footskip=17.5mm,centering]{geometry}  % 使用 geomerty 宏包设置页面格式
\renewcommand{\baselinestretch}{1.25}  %设置行距=默认行距(1.2倍)*1.25=1.5倍
%%=====================================================================================================%%
%%
%%                加载所需宏包,可根据需要增删 (宏包功能可参考 LaTeX 编辑部 网站的简要说明)
%%
%%=====================================================================================================%%
\usepackage{mathrsfs}  % 加载 mathrsfs 字体宏包,在数学中使用 Raph Smith’s Formal Script 字体
\usepackage{amsmath,amssymb,amsthm}  % 加载数学公式、数学符号、定理和证明排版宏包
\usepackage{graphicx,curves,epic}  % 加载图形宏包、绘图宏包、绘图宏包
\usepackage{subfig}  % 加载子图宏包。 subfig 宏包是 subfigure 宏包的升级版,且二者冲突
\usepackage{tikz,pgfplots,circuitikz}  % 加载绘图、2D3D和散点图绘制、电路图绘制宏包
\usepackage{xcolor}  % 加载颜色处理宏包,是 color 宏包的加强版
\usepackage{calc}  % 加载 LaTeX 的算术运算增强宏包
\usepackage{array,tabularx}  % array 和 tabular 环境功能增强宏包、自动计算表格列宽宏包
\usepackage{booktabs}  % 表格顶部、中部和底部使用不同粗细的水平线宏包
\usepackage{tabularray}  % 超好用的新一代表格排版宏包
\usepackage[labelsep=quad]{caption}  % 加载图表标题宏包,本文设置分隔符为一个\quad
\usepackage[T1]{fontenc}  % 加载字体宏包,调用 T1 科克编码字体
\usepackage{extarrows}  % 加载长度自适应箭标宏包
\usepackage{bm}  % 以粗体方式显示数学公式宏包。它提供一个在数学模式中使用的 \bm{数学式} 命令
%\usepackage{appendix}  % 加载附录宏包
\usepackage{float,floatflt}  % 加载新浮动体宏包、图文混排宏包
%\usepackage{floatrow}  % 加载灵活排版插图和表格浮动体宏包,建议同时加载 graphicx 宏包和 subcaption 宏包
%\usepackage{graphicx}
%\usepackage{subcaption}  % 加载设置子标题宏包
%\usepackage{wrapstuff}  % 加载另一个图文混排宏包
%%=====================================================================================================%%
%%
%%                设置页眉页脚,页眉居中显示东北师范大学硕士学位论文,页脚居中显示页码,页眉有横线
%%
%%=====================================================================================================%%
\usepackage{fancyhdr}  %change page margings and sizes, headers and footers,
\pagestyle{fancy}   %紧跟 \usepackage{fancyhdr},
\fancyhead{}  %清除页眉页脚
%\fancyhead[L,R]{}  %设置页眉左右位置为空
\fancyhead[C]{东北师范大学硕士学位论文}  %设置页眉居中位置
\fancyfoot{}  %清除页脚
%\fancyfoot[L,R]{}  %设置页角左右位置为空
\fancyfoot[C]{\thepage}  %设置角眉居中位置显示页码
\renewcommand{\headrulewidth}{1pt}  %设置页眉线宽度为 1 磅
%%=====================================================================================================%%
%%
%%                 使用 CTeX 宏包设置节、小节、小小节标题格式
%%
%%=====================================================================================================%%
\ctexset{
  section={  %  设置节标题格式
    name         = {,\hspace{-0.5\ccwd}},
    beforeskip   = 48pt,
    fixskip      = true,
    format       = \centering\heiti\bf\zihao{3},
    numberformat = \heiti\bf\zihao{3},
    afterskip    = 24pt,
  },
  subsection={  %  设置小节标题格式
    name         = {,\hspace{-0.5\ccwd}},
    beforeskip   = 6pt,
    format       = \heiti\bf\zihao{4},
    numberformat+ = \heiti\bf\zihao{4},
    afterskip    = 0pt,
  },
  subsubsection={  %  设置小小节标题格式
    name         = {,\hspace{-0.5\ccwd}},
    beforeskip   = 6pt,
    format       = \songti\bf\zihao{-4},
    numberformat = \bf\zihao{-4},
    afterskip    = 0pt,
  }
}
%%=====================================================================================================%%
%%
%%                 设置目录深度、格式
%%
%%=====================================================================================================%%
\usepackage{titletoc} %加载目录格式设置宏包
%==============    设置章节目录格式
\setcounter{tocdepth}{3}  %设置章节目录深度。article版式没有章层次标题,一级标题为节
\renewcommand\contentsname{目\hspace{2\ccwd}录}     %修改目录标题
\titlecontents{section}[0\ccwd]                     %标题名:节,左间距为 0(首行无缩进与突出)
    {\addvspace{.3\baselineskip}\zihao{-4}\heiti}   %标题格式:与上一个标题增加0.3倍行距,小四号黑体
    {\contentslabel{1\ccwd}}                        %标题标志:标题标志宽度为 1个汉字宽度
    {\hspace*{-1\ccwd}}                             %无序号标题格式:前移 1个汉字宽度
    {\hspace{0.5\ccwd}\titlerule*{.}\contentspage}  %指引线与页码:与标题内容间距半个汉字,点填充,页码

\titlecontents{subsection}[1\ccwd]                  %标题名:小节,左间距为 1个汉字宽度(首行无缩进与突出)
    {\addvspace{.3\baselineskip}\zihao{-4}\songti}
    {\contentslabel{1\ccwd}\hspace{1\ccwd}}         %标题标志:标题标志宽度为 1个汉字宽度,后面增加 1个汉字宽度
    {\hspace*{-1\ccwd}}
    {\hspace{0.5\ccwd}\titlerule*{.}\contentspage}

\titlecontents{subsubsection}[2\ccwd]
    {\addvspace{.3\baselineskip}\zihao{-4}\songti}
    {\contentslabel{1\ccwd}\hspace{1.5\ccwd}}
    {\hspace*{-1\ccwd}}
    {\hspace{0.5\ccwd}\titlerule*{.}\contentspage}

%==============    设置插图目录格式
\renewcommand{\listfigurename}{插图目录}             %修改插图目录标题
\titlecontents{figure}[0\ccwd]                      %标题名:图,左间距为 0(首行无缩进与突出)
    {\addvspace{.3\baselineskip}\zihao{-4}\songti}  %标题格式:与上一个标题增加0.3倍行距,小四号宋体
    {图~\thecontentslabel{\makebox[3mm]{}}}         %标题标志:标题标志与标题内容间距 3mm
    {}                                              %无序号标题格式:空置
    {\hspace{0.5\ccwd}\titlerule*{.}\contentspage}  %指引线与页码:与标题内容间距半个汉字,点填充,页码

%==============    设置附表目录格式
\renewcommand{\listtablename}{附表目录}             %修改表格目录标题
\titlecontents{table}[0\ccwd]                       %标题名:表,左间距为 0(首行无缩进与突出)
    {\addvspace{.3\baselineskip}\zihao{-4}\songti}  %标题格式:与上一个标题增加0.3倍行距,小四号黑体
    {表~\thecontentslabel{\makebox[3mm]{}}}         %标题标志:标题标志与标题内容间距 3mm
    {}                                              %无序号标题格式:前移 1个汉字宽度
    {\hspace{0.5\ccwd}\titlerule*{.}\contentspage}  %指引线与页码:与标题内容间距半个汉字,点填充,页码

%%=====================================================================================================%%
%%
%%                 设置脚注显示符号为带圈的数字,大于 10 的不能正常显示
%%
%%=====================================================================================================%%
%\renewcommand{\thefootnote}{\fnsymbol{footnote}}
\usepackage{pifont}  %加载提供文稿中常见的符号的宏包,选择命令 \ding {代号}
\usepackage[perpage,stable,symbol*]{footmisc}  %加载自定义脚注符号宏包,每页独立编号,
\newcommand*\dingctr[1]{\protect\ding{\number\numexpr\value{#1}+171\relax}}  % 调用 \ding 中带圈的数字
\renewcommand*\thefootnote{\dingctr{footnote}}  % 生成带圈数字脚注,大于 10 的不能正常显示

\makeatletter
%%%% 悬挂的脚注格式
\renewcommand\@makefntext[1]{%
    \setlength\leftskip{1.2\ccwd}%
    \setlength\parindent{2\ccwd}\selectfont
    \noindent\llap{\@thefnmark\ }#1}
%%%% 无悬挂的脚注格式
%\renewcommand\@makefntext[1]{%
%    \setlength\parindent{2\ccwd}\selectfont
%    \@thefnmark\ #1}
\makeatother

\renewcommand{\footnotesize}{\zihao{-5}}

%%=====================================================================================================%%
%%
%%                 其它设置
%%
%%=====================================================================================================%%

%==============    默认罗马字体
\renewcommand{\rmdefault}{ptm}  % pdf架构下设置默认罗马字体为 Times New Roman

%==============    设置公式、图表编号格式
\numberwithin{equation}{section}  %needs amsmath packge %公式在节内编号
\renewcommand{\theequation}{\thesection-\arabic{equation}}
\renewcommand{\thefigure}{\thesection.\arabic{figure}}
\renewcommand{\thetable}{\thesection.\arabic{table}}

\numberwithin{equation}{section}  % 这样所有公式(不管是 equation 还是 align)都会按照 (章节号.公式号) 的规则编号
%==============    设置参考文献名
\CTEXoptions[bibname={参考文献}]

%==============    设置附录名
\renewcommand\appendix{\par
    \setcounter{section}{0}
    \setcounter{subsection}{0}
    \gdef\thesection{附录 \Alph{section}}}

%==============    定义新的列表环境,使说明文字左对齐,用以排版等
\newenvironment{newdescription}[1]%
   {\begin{list}{}{\renewcommand{\makelabel}[1]{\songti{##1}\hfil}%
      \settowidth{\labelwidth}{\songti{#1}}%
      \setlength{\labelsep}{0.5\ccwd}%
      \setlength{\parsep}{0pt}%
      \setlength{\itemsep}{2.5pt}%
      \setlength{\leftmargin}{\labelwidth+\labelsep}}}%
   {\end{list}}

%==============    使用 \tikz 定义带圈数字
\newcommand*\circled[1]{\tikz[baseline=(char.base)]{\node[shape=circle,draw,inner sep=0.2pt] (char) {#1};}}

%==============俄文字母
\font\ewenb=wncyb10 \font\eweni=wncyi10 \font\ewenr=wncyr10
\font\ewensc=wncysc10 \font\ewenss=wncyss10

%==============定理设置==============%
\newtheorem{corollary}{推论}[section]
\newtheorem{criterion}{Criterion}[section]
\newtheorem{definition}{定义}[section]
\newtheorem{example}{例}[section]
\newtheorem{lemma}{引理}[section]
\newtheorem{notation}{Notation}[section]
\newtheorem{proposition}{命题}[section]
\newtheorem{remark}{Remark}[section]
\newtheorem{theorem}{定理}[section]
\newtheorem{assumption}{假设}[section]

%==============定义带左右标号的公式环境==============%
\makeatletter
\def\xlabel#1#2{%
{\@bsphack\protected@write\@auxout{}%
{\string\newlabel{#2}{{#1}{\thepage}}}%
\@esphack}{\mathrm(#1)}}
\makeatother
%定义结束%
%下面是一个例子,注意&&的用法%
      %\begin{flalign}
      %\xlabel{H1}{eq:refL}&&x=y+z&&
      %\label{eq:ee1}\\
      %\xlabel{H2}{eq:xxy}&&a=b^2+c^2-a&&
      %\label{eq:ee2}
      %\end{flalign}
%例子结束%

\renewcommand{\thetheorem}{\thesection.\arabic{theorem}}
\renewcommand{\thelemma}{\thesection.\arabic{lemma}}
\renewcommand{\thecorollary}{\thesection.\arabic{corollary}}
\renewcommand{\theremark}{\thesection.\arabic{remark}}
\renewcommand{\thedefinition}{\thesection.\arabic{definition}}
\renewcommand{\theproposition}{\thesection.\arabic{proposition}}
\renewcommand{\theexample}{\thesection.\arabic{example}}

%==============定义上角标引用参考文献==============%
\newcommand{\upcite}[1]{\textsuperscript{\cite{#1}}}

%==============定义新函数==============%
\DeclareMathOperator*{\esssup}{\mathrm{ess}\sup}

%==============设置算法环境==============%
\usepackage[linesnumbered,ruled,vlined]{algorithm2e}
\usepackage{bm}
% \usepackage[ruled]{algorithm2e}
% 中文替换与格式设定
\renewcommand{\algorithmcfname}{算法}
\SetKwInOut{KwInput}{输入}     % 取消缩进的输入
\SetKwInOut{KwOutput}{输出}    % 取消缩进的输出
\SetNlSty{}{}{}                % 行号样式(无加粗括号)
\LinesNumbered                 % 显示行号
\SetAlCapHSkip{0pt}           % 标题与正文左对齐

\SetAlgoNlRelativeSize{0}  % 控制编号字号
\SetNlSkip{0.2em}          % 控制编号右边的空隙
\SetNlSty{textbf}{\ }{}    % 控制编号格式

%==============生成书签==============%
\usepackage{hyperref}

%%%%%%%%%%%%%%%%%%%%%%%%%%%%%%%%%%%%%%%%%%%%%%%%%%%%%%%%%%%%%%%%%%%%%%%%%%%%%%%%%%%%%%%%%%%%%%%%%%%%%%%%%
%%%%%%%%%%%%%%%%
%%%%%%%%%%%%%%%%   开始正文区
%%%%%%%%%%%%%%%%
%%%%%%%%%%%%%%%%%%%%%%%%%%%%%%%%%%%%%%%%%%%%%%%%%%%%%%%%%%%%%%%%%%%%%%%%%%%%%%%%%%%%%%%%%%%%%%%%%%%%%%%%%

\begin{document}

%%=====================================================================================================%%
%%
%%                输入论文的相关信息,以自动生成封面等
%%
%%=====================================================================================================%%

%==============    封面有关信息
\newcommand{\Cmytitle}{基于时间离散和空间离散的两类随机微分方程数值格式比较} %论文标题(中文)
\newcommand{\Emytitle}{Comparison of Two Types of Numerical Schemes for Stochastic Differential Equations Based on Temporal and Spatial Discretization} %论文标题(英文)
\newcommand{\Cmyname}{华光辉} %作者姓名(中文)
\newcommand{\Emyname}{Hua Guanghui} %作者姓名(英文)
\newcommand{\myID}{2023102228} %研究生学号
\newcommand{\Cmysupervisor}{祖建\quad 副教授} %导师姓名、职称(中文)
\newcommand{\Emysupervisor}{Zu Jian\quad Associate Professor} %导师姓名导师姓名、职称(英文)
\newcommand{\Cmysecurityleve}{无} %密级(中文)
\newcommand{\Emysecurityleve}{Open level} %密级(中文)

\newcommand{\CmyprimarySC}{数学} %一级学科(中文)
\newcommand{\EmyprimarySC}{Mathematics} %一级学科(英文)
\newcommand{\CmysecondarySC}{应用数学} %二级学科(中文)
\newcommand{\EmysecondarySC}{Applied mathematics} %二级学科(英文)
\newcommand{\Cmyresearcharea}{微分方程与动力系统} %研究方向(中文)
\newcommand{\Emyresearcharea}{Differential equations and dynamic systems} %研究方向(英文)
\newcommand{\Cmydate}{2026~年~5~月} %日期(中文)
\newcommand{\Emydate}{2026,~05}    %日期(英文)

%==============    评阅专家信息 (若没有评阅人 4 和 5,相关信息设置为空)
\newcommand{\reviewerA}{评阅人1}    %评阅人1姓名
\newcommand{\reviewerAEP}{匿名评阅} %评阅人1工作单位/职称
\newcommand{\reviewerAOR}{优秀}     %评阅人1总体评价

\newcommand{\reviewerB}{评阅人2}    %评阅人2姓名
\newcommand{\reviewerBEP}{东北师范大学/副教授} %评阅人2工作单位/职称
\newcommand{\reviewerBOR}{良好}     %评阅人2总体评价

\newcommand{\reviewerC}{评阅人3}    %评阅人3姓名
\newcommand{\reviewerCEP}{匿名评阅} %评阅人3工作单位/职称
\newcommand{\reviewerCOR}{优秀}     %评阅人3总体评价

\newcommand{\reviewerD}{评阅人4}    %评阅人4姓名
\newcommand{\reviewerDEP}{} %评阅人4工作单位/职称
\newcommand{\reviewerDOR}{}     %评阅人4总体评价

\newcommand{\reviewerE}{评阅人5}    %评阅人5姓名
\newcommand{\reviewerEEP}{} %评阅人5工作单位/职称
\newcommand{\reviewerEOR}{}     %评阅人5总体评价

%==============    答辩委员会人员信息  (若没有答辩委员会委员5和6,相关信息设置为空)

\newcommand{\presidentofDC}{XXX}    %答辩委员会主席姓名
\newcommand{\presidentofDCA}{XXXXX大学} %主席单位
\newcommand{\presidentofDCT}{教授}     %主席职称

\newcommand{\memberofDCa}{XXX}    %答辩委员会委员1姓名
\newcommand{\memberofDCaA}{XXXXX大学} %答辩委员会委员1工作单位
\newcommand{\memberofDCaT}{研究员}     %答辩委员会委员1职称

\newcommand{\memberofDCb}{XXX}    %答辩委员会委员2姓名
\newcommand{\memberofDCbA}{XXXXX大学} %答辩委员会委员2工作单位
\newcommand{\memberofDCbT}{教授}     %答辩委员会委员2职称

\newcommand{\memberofDCc}{XXX}    %答辩委员会委员3姓名
\newcommand{\memberofDCcA}{XXXXX大学} %答辩委员会委员3工作单位
\newcommand{\memberofDCcT}{教授}     %答辩委员会委员3职称

\newcommand{\memberofDCd}{XXX}    %答辩委员会委员4姓名
\newcommand{\memberofDCdA}{XXXXX大学} %答辩委员会委员4工作单位
\newcommand{\memberofDCdT}{研究员}     %答辩委员会委员4职称

\newcommand{\memberofDCe}{XXX}    %答辩委员会委员5姓名
\newcommand{\memberofDCeA}{} %答辩委员会委员5工作单位
\newcommand{\memberofDCeT}{}     %答辩委员会委员5职称

\newcommand{\memberofDCf}{XXX}    %答辩委员会委员6姓名
\newcommand{\memberofDCfA}{} %答辩委员会委员6工作单位
\newcommand{\memberofDCfT}{}     %答辩委员会委员6职称

%%=====================================================================================================%%
%%
%%                生成封面,信息自动生成,不需修改
%%
%%=====================================================================================================%%
\begin{titlepage}
\vspace{10mm}
\begin{center}
{\songti\zihao{4}硕士研究生学位论文}
\vspace{5mm}

%%%%%%%%% 生成学号、密级
{\songti\zihao{5}学校代码:10200\quad   研究生学号:\myID \quad   密级:\Cmysecurityleve}

\noindent\rule[2mm]{\textwidth}{1pt}

\vspace{20mm}
\includegraphics[scale=0.6]{fig/xiaohui.png}
\vspace{15mm}
\parbox[b][50mm][c]{\textwidth}{\centering  \zihao{3} \Cmytitle \vspace{5mm}\\
\Emytitle}

%%%%%%%%% 生成作者、指导教师、一级学科、二级学科、研究方向
\parbox[b][40mm][c]{\textwidth}{\centering \songti\zihao{-4}

作\qquad 者~~\underline{\makebox[55mm][c]{\Cmyname}}

指导教师~~\underline{\makebox[55mm][c]{\Cmysupervisor}}

一级学科~~\underline{\makebox[55mm][c]{\CmyprimarySC}}

二级学科~~\underline{\makebox[55mm][c]{\CmysecondarySC}}

研究方向~~\underline{\makebox[55mm][c]{\Cmyresearcharea}}
}
\vspace{10mm}
\includegraphics[scale=0.6]{fig/xiaoming.png}~~\raisebox{2mm}{\songti\zihao{-4}学位评定委员会}
\vspace{5mm}

%%%%%%%%% 生成日期
{\songti\zihao{4}\Cmydate}

\end{center}

\end{titlepage}

%%=====================================================================================================%%
%%
%%                空白页
%%
%%=====================================================================================================%%
\newpage
\pagestyle{empty} %设置页面格式为 empty,无页眉无页脚
\mbox{}

%%=====================================================================================================%%
%%
%%                生成英文封面,信息自动生成,不需修改
%%
%%=====================================================================================================%%

\begin{titlepage}

\vspace{10mm}
\begin{center}
{\zihao{4}A Thesis}
\vspace{5mm}

%==============    生成学校代码、学号、密级
{\zihao{5}School code: 10200 \quad   Student ID:\myID \quad   Security level:\Emysecurityleve}

\noindent\rule[2mm]{\textwidth}{1pt}

\vspace{20mm}
\includegraphics[scale=0.6]{fig/xiaohui.png}
\vspace{15mm}
\parbox[b][50mm][c]{\textwidth} %高为 50mm、宽为页面宽度的段落盒子,用以排版英文论文标题
    {
     \centering \zihao{3}
%==============    生成论文英文标题
     \Emytitle
    }

\parbox[b][40mm][c]{0.8\textwidth}{ \songti\zihao{-4}

%%%%%%%%% 生成作者、指导教师、一级学科、二级学科、研究方向,下划线的长度需手动调节使右端对齐
Author~~\underline{\hspace{24mm}  \Emyname  \hspace{67mm}}

Supervisor~~\underline{\hspace{18mm}  \Emysupervisor  \hspace{21mm}}

Primary Subject Classification~~\underline{\hspace{15mm}  \EmyprimarySC  \hspace{35mm}}

Secondary Subject Classification~~\underline{\hspace{1mm}  \EmysecondarySC  \hspace{2mm}}

Research Area~~\underline{\hspace{10mm}  \Emyresearcharea  \hspace{19.5mm}}
}

\vspace{10mm}
{\zihao{-4}Northeast Normal University Academic Degree Evaluation Committee}
\vspace{5mm}

%==============    生成日期
{\songti\zihao{4}\Emydate}

\end{center}

\end{titlepage}

%%=====================================================================================================%%
%%
%%                空白页
%%
%%=====================================================================================================%%
\newpage
\mbox{}

%%=====================================================================================================%%
%%
%%                学位论文评阅专家及答辩委员会人员信息,自动生成,不需要修改
%%
%%=====================================================================================================%%
\newpage
\begin{center}
{\zihao{3}\bf\songti 学位论文评阅专家及答辩委员会人员信息}

\vspace{7mm}

{\songti\zihao{-4}
\begin{tblr}{width=\textwidth,stretch=1.42,colspec={|Q[c,13mm]|Q[c,11mm]|Q[c,25mm]|Q[c,50mm]|X[c]|},
             rowspec={|Q[m]|Q[m]|Q[m]|Q[m]|Q[m]|Q[m]|Q[m]|Q[m]|Q[m]|Q[m]|Q[m]|
             Q[m]|Q[m]|Q[m]|Q[m]|Q[m]|Q[m]|},
             cell{1}{2}={c=4}{c},cell{2}{2}={c=4}{c},cell{3}{2}={c=4}{c},
             cell{4}{1}={r=6}{c},cell{4-9}{2}={c=2}{c},
             cell{10}{1}={r=8}{c},cell{10}{2}={c=2}{c},
             cell{12}{2}={r=6}{c}}
{\bf 论\quad 文\\题\quad 目} & \Cmytitle         &       &               &\\
{\bf 作\quad 者\\姓\quad 名} & \Cmyname          &       &               &\\
{\bf 指\quad 导\\教\quad 师} & \Cmysupervisor    &       &               &\\
{\bf 论\\~\\文\\~\\评\\~\\阅\\~\\人} & {\bf 姓\quad 名}  &       & {\bf 工作单位/职称} & {\bf 总体评价}\\
                             & \reviewerA     &       & \reviewerAEP      &\reviewerAOR\\
                             & \reviewerB     &       & \reviewerBEP      &\reviewerBOR\\
                             & \reviewerC     &       & \reviewerCEP      &\reviewerCOR\\
                             & \reviewerD     &       & \reviewerDEP      &\reviewerDOR\\
                             & \reviewerE     &       & \reviewerEEP      &\reviewerEOR\\
{\bf 学\vspace{3mm}\\位\vspace{3mm}\\论\vspace{3mm}\\文\vspace{3mm}\\答\vspace{3mm}
          \\辩\vspace{3mm}\\委\vspace{3mm}\\员\vspace{3mm}\\会}
          & {\bf 姓\quad 名} &                 & {\bf 工作单位}     &{\bf 职\quad 称}\\
          & {\bf 主\\席}     & \presidentofDC  & \presidentofDCA    & \presidentofDCT\\
          & {\bf 委\\员}     & \memberofDCa    & \memberofDCaA      & \memberofDCaT  \\
          &                  & \memberofDCb    & \memberofDCbA      & \memberofDCbT  \\
          &                  & \memberofDCc    & \memberofDCcA      & \memberofDCcT  \\
          &                  & \memberofDCd    & \memberofDCdA      & \memberofDCdT  \\
          &                  & \memberofDCe    & \memberofDCeA      & \memberofDCeT  \\
          &                  & \memberofDCf    & \memberofDCfA      & \memberofDCfT  \\
\end{tblr}
}
\end{center}

%%=====================================================================================================%%
%%
%%                空白页
%%
%%=====================================================================================================%%
\newpage
\mbox{}


%%=====================================================================================================%%
%%
%%                中文摘要,需作者编辑
%%
%%=====================================================================================================%%
\newpage
\pagestyle{plain}  %设置页面格式为 plain,无页眉有页脚
\setcounter{page}{1}
\renewcommand{\thepage}{\Roman{page}}

	\section*{摘\hspace{2\ccwd} 要}
	\addcontentsline{toc}{section}{摘要}
	\songti\zihao{-4} %宋体小四号
	
	随机微分方程在分子动力学、数理金融和生物系统等领域广泛出现。对于仅满足局部 Lipschitz 条件且漂移具有超线性增长的方程,经典 Euler–Maruyama 等时间离散格式可能数值发散,而基于无穷小生成元的空间离散方法(如连续时间随机游走)在稳定性和长期统计性质上往往更可靠。本文围绕时间离散与空间离散哪一种更适合刻画给定 SDE 的长期行为这一问题,在统一框架下比较两类数值格式的精度与计算代价。
	
	在模型方面,本文选取一维立方振子和随机 Canard 快-慢系统作为原型,时间离散侧采用驯服 Euler-Maruyama 与截断 Euler–Maruyama格式;空间离散侧采用连续时间游走格式 \(Q_u\) 及其改进格式 \(\widetilde Q_u\)。对一维立方振子,本文在漂移主导区域(当\(X_t \gg 1\)时)引入固定空间跨越距离的时间计算,将近似性漂移时间、连续时间随机游走的平均驻留时间以及真实 SDE 的平均首达时间(MFPT)置于同一渐近框架,并给出 MFPT 的局部渐近展开,从而定量比较时间离散与空间离散在 MFPT 上的误差阶和适用区间。
	
	对于随机 Canard 系统,本文引入逃逸概率函数和MFPT 作为长期指标,从生成元视角得到其满足的方程,并构造相应的离散线性方程。在一定的光滑性和一致逼近假设下,可以证明改进格式 \(\widetilde Q_u\) 对 逃逸概率 与 MFPT 具有高阶空间收敛;相应地基于驯服与截断 Euler-Maruyama 的估计,其弱收敛阶通常不超过 \(O(\Delta^{1/2})\),为达到同等精度需要显著更高的计算代价。
	
	数值实验方面,本文通过样本路径的几何特征,首达时间统计以及逃逸概率/MFPT 的热力图,对比两类方法在快-慢结构、逃逸概率与长期统计性质上的表现,并在统一的误差–代价指标下给出定量曲线。结果表明:空间离散方法在长期统计方面明显更高效,而时间离散方法在捕捉单条样本轨道和短时动力学方面更具灵活性。本文的分析和实验为今后在复杂随机动力系统中选择和设计合适的数值格式提供了参考。
	\\   % 摘要文本与关键词之间空一行
	
	{\songti\bf\zihao{-4}关键词:}随机微分方程;时间离散;空间离散;截断Euler–Maruyama;
	
	%%=====================================================================================================%%
	%%
	%%                英文摘要,需作者编辑
	%%
	%%=====================================================================================================%%
	\newpage
	\section*{Abstract}
	\addcontentsline{toc}{section}{Abstract}
	\large %设置英文字号
	Stochastic differential equations arise widely in molecular dynamics, mathematical finance, and biological systems. For equations whose coefficients satisfy only a local Lipschitz condition and whose drift has superlinear growth, classical time-discretization schemes such as the Euler--Maruyama method may diverge numerically, whereas spatial-discretization methods based on the infinitesimal generator (such as continuous-time random walks) are often more reliable in terms of stability and long-time statistical properties. This thesis addresses the question of whether time discretization or space discretization is better suited to capturing the long-time behaviour of a given SDE, and compares, within a unified framework, the accuracy and computational cost of these two classes of numerical schemes.
	
	At the level of model problems, we take the one-dimensional cubic oscillator and a stochastic canard slow--fast system as prototypes. On the time-discretization side we employ the Tamed Euler-Maruyama--Maruyama and truncated Euler--Maruyama schemes; on the space-discretization side we use the continuous-time random walk scheme \(Q_u\) and its improved variant \(\widetilde Q_u\). For the one-dimensional cubic oscillator, in the drift-dominated regime (when \(X_t \gg 1\)) we introduce a time scale associated with a fixed spatial step, place the approximate drift time, the mean holding time of the continuous-time random walk, and the true mean first passage time (MFPT) of the SDE into a common asymptotic framework, and derive a local asymptotic expansion for the MFPT. This yields a quantitative comparison of the MFPT error order and validity range for time- and space-discretization schemes.
	
	For the stochastic canard system, we use the escape probability function and the MFPT as long-time diagnostics. From the generator viewpoint we derive the equations satisfied by these quantities and construct the corresponding discrete linear systems. Under suitable smoothness and uniform-approximation assumptions, we prove that the improved scheme \(\widetilde Q_u\) attains higher-order spatial convergence for both the escape probability and the MFPT. In contrast, estimates based on the tamed and truncated Euler--Maruyama schemes typically have weak convergence order no higher than \(O(\Delta^{1/2})\), and therefore require substantially larger computational cost to achieve the same accuracy.
	
	On the numerical side, we compare the two approaches by examining sample-path geometric features, statistics of first passage times, and heat maps of the escape probability and MFPT, and we present quantitative work--error curves under a unified accuracy–cost metric. The results show that spatial-discretization methods are markedly more efficient for long-time statistical quantities, whereas time-discretization methods offer greater flexibility for capturing individual sample paths and short-time dynamics. The analysis and experiments in this thesis provide guidance for choosing and designing appropriate numerical schemes for complex stochastic dynamical systems.
	\\   % 摘要文本与关键词之间空一行
	
	{\bf\large Key words:}  stochastic differential equations; time discretization; space discretization; continuous–time random walk; Tamed Euler-Maruyama; truncated Euler–Maruyama; committor function; 
	%%=====================================================================================================%%
	%%
	%%                生成章节目录、插图目录、附表目录,自动生成
	%%
	%%=====================================================================================================%%
	\newpage
	\tableofcontents  %生成目录
	
	\newpage
	\listoffigures   %生成插图目录
	\addcontentsline{toc}{section}{插图目录}
	
	\newpage
	\listoftables   %生成附表目录
	\addcontentsline{toc}{section}{附表目录}
	
	%%=====================================================================================================%%
	%%
	%%                符号和缩略语说明页
	%%
	%%=====================================================================================================%%
	
	\newpage
	\section*{符号和缩略语说明}
	\addcontentsline{toc}{section}{符号和缩略语说明}
	
	\begin{newdescription}{XXXXX}  %第二个 {} 中填写最长的符号和缩略语,以确定列表标号位置的宽度
		
		\item[\( \mu(X_t) \)]    漂移项(drift term) 
		\item[$\sigma(X_t)$] 扩散项(diffusion term)
		\item[$\delta x,h $] 空间离散步长
		\item[$\Delta$] 时间离散步长
		\item[$Q_u$] \cite{bou2018continuous}中的有限差分格式,本文称为一阶格式
		\item[$Q_c$] \cite{bou2018continuous}中的有限体积格式,本文称为二阶格式
		\item[$\widetilde{Q}_u$] 改进的一阶格式
		\item[$t^*$] $t^e$ 的主要部分
		\item[$t^u$] $Q_u$ 格式对应的平均等待时间 (mean holding time)
		\item[$t^c$] $Q_c$ 格式对应的平均等待时间 (mean holding time)
		\item[$\tilde{t}^u$] $\widetilde{Q}_u$ 格式对应的平均等待时间 (mean holding time)
		\item[$t^{\delta}$] 驯服方法运动 $h$ 距离所需的平均时间
		\item[$t^{\Delta}$] 截断方法运动 $h$ 距离所需的平均时间
		\item[$f(x)\asymp g(x)$]
		渐近同阶\footnote{指存在常数 $c_1,c_2>0$ 与 $R>0$,使得对所有足够大的 $x$(默认 $x\ge R$),有
		\(		c_1\,|g(x)| \;\le\; |f(x)| \;\le\; c_2\,|g(x)|.
		\)
		隐含常数 $c_1,c_2$ 不依赖于自变量(如 $x$)与网格/步长参数(如 $h,\Delta$)。若以 $h\downarrow 0$ 或 $n\to\infty$ 为极限,定义作相应替换理解。}
		%与 $f(x)\sim g(x)$(比值极限为 1)不同,$\asymp$ 只要求双边常数界;例如在顺漂移+小位移极限下,
		%$t^e=\frac{\delta}{x^3}+\frac{3}{2}\frac{\delta^2}{x^4}+O\!\big(\frac{\delta^3}{x^5}\big)$,
		%因此 $t^e \asymp \delta/x^3$(当 $x\to\infty$ 且 $\delta/x\to 0$)。
		
		\item[$f(x)\sim g(x)$]
		渐近等价 \footnote{指 $\displaystyle \lim_{x\to\infty}\frac{f(x)}{g(x)}=1$(或相应极限)。相比 $f\asymp g$ 更强。}
		
		% ——— 符号与渐近不等式(建议置于“符号说明”表内) ———
		\item[\(\Delta \lesssim \delta\)]
		\(\Delta = \mathcal O(\delta)\)
		\footnote{即存在常数 \(C>0\)(与 \(\Delta,\delta\) 无关)使
		\(\Delta \le C\,\delta\)。本文默认:隐含常数允许依赖模型的固定参数(如维度 \(d\)、指数、Lipschitz 常数等),
		但不依赖于离散尺度参数(如 \(\Delta,\delta,h(\Delta)\) 等)。}
		
		
	\end{newdescription}
	
	
	%%=====================================================================================================%%
	%%
	%%                正文,由作者编辑
	%%
	%%=====================================================================================================%%
	
	\newpage
	\pagestyle{fancy}     %开始使用 fancy 页版式,带页眉、页脚
	\setcounter{page}{1}  %设置起始页码为 1
	\renewcommand{\thepage}{\arabic{page}}  %设置阿拉伯数字显示页码
	\songti\zihao{-4}  %设置正文字体为宋体小四号
	
	
	%%=====================================================================================================%%
	%%
	%%                第  1 节  引言(绪论)
	%%
	%%=====================================================================================================%%
	
	\section{引言}  %\footnote{测试标题中的脚注的有效性。}\footnote{本模板的脚注序号从\ding{172}到\ding{181}}}
	
	% 研究生学位论文是研究生在学期间独立完成的主要科研成果,较为全面地反映研究生的学术水平,是学校授予研究生学位的重要依据。
	% 为提高学位论文质量,规范研究生学位论文格式,研究生院依据《科学技术报告、学位论文和学术论文的编写格式》(GB/T 7713—1987)、
	% 《学位论文编写规则》(GB/T 7713.1—2006)、《科技报告编写规则》(GB/T 7713.3—2014)、《信息与文献 参考文献著录规则》(GB/T 7714—2015)、
	% 《学术论文编写规则》(GB/T 7713.2—2022)等相关国家标准,特制定《东北师范大学研究生学位论文写作格式规范》,
	% 供东北师范大学博士、硕士研究生参考使用。~\footnote{引自《东北师范大学研究生学位论文写作格式规范》。}
	
	
	\subsection{随机微分方程的的起源与发展}
	随机微分方程(Stochastic Differential Equations, SDE)在金融市场、热传导、生物化学反应网络、大气海洋科学、流行病学、种群动力学及数理金融等众多领域中被广泛用于刻画随机扰动下的动力学行为。一般地,SDE可表示为
	
	\begin{equation}
	dX_t = \mu(X_t)\,dt + \sigma(X_t)\,dW_t, \quad X_0 = x_0,  
	\end{equation}
	
	其中\( \mu(x)\)是漂移项(drift),\(\sigma(x)\)是扩散项(diffusion),\(W_t\)为标准布朗运动。由于SDE通常难以求得解析解,因此构造合理有效的数值方法寻求数值解来替代精确解是十分有必要的。在数值方法的发展脉络上,SDE领域与常微分方程(ODE)领域有一定相似之处,例如基于时间离散化的Euler–Maruyama方法、Milstein方法等经典算法被提出用于求解SDE。但由于随机微积分的特殊性,SDE数值解法与ODE方法仍存在重要差异。ODE解在给定初值时通常光滑且唯一,因此可在离散时间点之间使用插值来近似连续轨道,这使得通过时间步进得到高阶方法成为可能。而SDE的解尽管形式上看似类似于对应的确定性系统,但其样本路径连续而处处不可导,并且针对每一不同的布朗运动轨道,同一初值可产生一族随机样本路径。因此,直接套用标准的时间步进积分方法来模拟SDE,将面临比ODE情形更多的挑战。
	
	其中一个突出挑战在于长时间稳定性及对不变分布的采样能力。对于具有平稳分布的随机系统,希望数值解在长时间模拟下能保持对该分布的正确采样。然而,传统数值积分方案往往难以满足这一点。事实上,即使原始SDE是遍历的(存在唯一的平稳分布并且解遍历于此分布),其对应的数值离散方案通常并不遍历,无法保证收敛到正确的稳态分布。另外,在漂移项仅满足局部Lipschitz连续(而非全局)的常见情形下,数值解可能会出现发散或爆炸。例如,Euler–Maruyama显式方法在漂移或扩散项具有超线性增长时可能失稳:仿真得到的Markov链轨道可能偏离真实解并发散,其高阶矩甚至可能在有限时间内趋于无穷,而真实解的矩仍保持有限值。这表明,当漂移系数不满足线性增长条件时,即使SDE本身存在唯一解,显式Euler方法也可能不收敛于真解的轨道。这样的不稳定不仅影响长期模拟结果的可信度,也会削弱有限时间区间内的数值精度。为此,我们需要发展更稳健的数值方法来处理此类非全局Lipschitz情形。
	
	隐式方法(如Implicit Euler)在理论上可以处理非Lipschitz驱动的SDE并保持收敛性,但每一步都需解非线性方程,计算代价高昂。因此,研究者们倾向于通过改造显式方法来提升其鲁棒性和稳定性。近年来出现了两类主要思路:其一是改进的时间离散算法,通过调整步进公式使显式方法也能在非全局Lipschitz条件下稳定收敛;其二是全新的空间离散算法,通过对SDE的生成元进行空间离散来得到马尔可夫跳跃过程,以避开直接的时间积分。
	

\subsection{随机微分方程数值解的研究现状}

由于多数 SDE 难以获得闭式精确解,研究者通常转向构造稳定且高效的数值算法,以数值近似替代解析表达,现有的工作可以分为两种离散方法:其一是基于时间步进的时间离散方法,其二是基于无穷小生成元离散化的空间离散方法。
时间离散框架中,Maruyama 于 1955 年提出 Euler--Maruyama(EM)格式\cite{Maruyama1955}。对一般形式的 SDE,其基本递推可写为
\begin{equation}
	X_{k+1}=X_{k}+ \mu \left(X_{k}\right) \Delta
	+\sigma \left(X_{k}\right)\left(W_{k+1}-W_{k}\right).
\end{equation}
经典结果表明,EM 方法在强意义下具有 $0.5$ 阶收敛、在弱意义下具有 $1.0$ 阶收敛。随后 Milstein 于 1974 年基于随机 Taylor 展开在一阶处截断,提出 Milstein 方法\cite{Milstein1975},其格式为
\begin{equation}
	\begin{aligned}
		X_{k+1}=X_{k}+\mu \left(X_{k}\right) \Delta
		+\sigma \left(X_{k}\right)\left(W_{k+1}-W_{k}\right)
		+\frac{1}{2} \sigma\left(X_{k}\right) \sigma^{\prime}\left(X_{k}\right)
		\left[\left(W_{k+1}-W_{k}\right)^{2}-\Delta \right].
	\end{aligned}
\end{equation}
该方法在强、弱意义下均可达到 $1.0$ 阶收敛\cite{weinan2021applied}。需要注意的是,Milstein 格式通常涉及导数项的计算,即便在一维情形下也会显著增加实现与计算成本。更系统的数值方法谱系及其收敛阶框架,可见 Kloeden 与 Platen 的专著\cite{kloeden1992stochastic}。

在理论假设方面,EM 方法对全局 Lipschitz 连续系数的 SDE 近似效果较为稳健。针对非全局 Lipschitz 情形,Higham、Mao 与 Stuart\cite{higham2002strong}在漂移与扩散满足局部 Lipschitz 条件并具有 $p$ 阶矩有界等假设下,证明了 EM 数值解在强均方意义下收敛到精确解;同时在漂移满足单边 Lipschitz、扩散满足全局 Lipschitz 的框架下,给出了隐式 Euler 方法的强收敛性结论。然而,对于漂移具有超线性增长并仅满足单边 Lipschitz 连续的更一般模型,EM 在有限时间强均方意义下的收敛性并不能得到保证。Hutzenthaler、Jentzen 与 Kloeden\cite{hutzenthaler2011strong}进一步指出:对一类非全局 Lipschitz 的 SDE,EM 近似在有限时间点 $T\in(0,\infty)$ 下可能既不强收敛也不弱收敛,甚至数值误差会在强、弱意义下发散。

为克服上述困难,Hutzenthaler 等人在 2012 年提出驯服 Euler(tamed Euler)思想\cite{hutzenthaler2012strong},其代表性格式为
\begin{equation}\label{eq:tamed_EM}
	X_{k+1}
	= X_{k}
	+\frac{\mu \left(X_{k}\right)\Delta}{1+\left\| \mu \left(X_{k}\right)\right\|\Delta}
	+\sigma \left(X_{k}\right)\left(W_{k+1}-W_{k}\right).
\end{equation}
该方法通过对漂移增量进行有界化处理,将单步更新中的漂移贡献限制在可控范围内,从而抑制因超线性漂移导致的数值爆炸。在全局单侧 Lipschitz 漂移与全局 Lipschitz 扩散等条件下,驯服 Euler 仍可保持与 EM 相同的强收敛阶($1/2$ 阶),并且作为显式格式在计算效率上通常优于隐式 Euler。

另一类重要改进是截断方法,Mao\cite{mao2015truncated}在局部 Lipschitz 条件与 Khasminskii 型条件
\begin{equation}\label{eq:Khasminskii}
	x^{T} \mu(x)+\frac{p-1}{2}|\sigma(x)|^{2} \leq K\left(1+|x|^{2}\right), \quad p > 2
\end{equation}
下提出截断 Euler--Maruyama 方法,并证明其强收敛性。其核心做法是引入与步长 $\Delta$ 相关的阈值,对漂移 $\mu(x)$ 与扩散 $\sigma(x)$ 在大状态区域进行适度截断:当数值状态超出阈值范围时,用阈值边界处的函数值替代原函数值,从而强制漂移与扩散在计算中保持有界,避免显式 EM 在超线性增长下出现爆炸。在局部 Lipschitz 与 Khasminskii 型线性增长条件下,该方法在均方意义下可证收敛;在扩散项满足多项式增长等附加条件时,还可得到更强的路径型收敛结论。除此之外,停止时间 Euler 方法\cite{liu2013strong}通过引入停时机制,使得当数值轨道即将进入不合理区域时终止或修正模拟,从而增强有界性与物理合理性。进一步地,Mao\cite{mao2016convergence}给出了截断 EM 在 $L^{q}(q\ge 2)$ 意义下的收敛速度,并指出在某些条件下收敛阶可任意接近 $q/2$;Hu等人\cite{Hu2018}则在更一般条件下改进了相关收敛速度估计并讨论了稳定性问题。尽管如此,上述时间离散显式格式在长时间模拟或多尺度刚性问题中仍可能面临效率与误差积累方面的局限。
除时间步进外,Bou-Rabee 与 Vanden-Eijnden\cite{bou2018continuous}于 2018 年提出从无穷小生成元出发构造连续时间数值方法的思路。其基本框架是:先对生成元采用有限差分或有限体积等方式离散,得到定义在离散网格上的离散算子;若该离散算子满足可实现性条件(可解释为某连续时间马尔可夫链的生成元,即迁移率矩阵满足非负性与行和为零),则可用随机模拟算法精确生成该跳跃过程的样本路径。该类方法的一个突出优点是空间步长由网格尺度 $\delta$ 直接控制,单次跳跃幅度受限,从而数值轨道可被严格约束在指定定义域内,避免越过物理边界;同时,跳跃率可随状态自适应变化,使得在漂移变化剧烈区域产生更频繁的小步跳跃,而在相对平稳区域长时间驻留,从而在一定程度上缓解多尺度刚性带来的困难。若进一步设计跳跃率满足相应平衡条件,还可使离散马尔可夫链的稳定分布与原 SDE 的平稳分布一致,从而在稳态采样意义下具有严格的理论保证。相关工作中,作者提出了基于有限差分的 $Q_u$ 方案与基于有限体积的 $Q_c$ 方案,它们在弱意义下具有较高精度并能较好地保持某些几何性质。

需要强调的是,连续时间随机行走(CTRW)类方法在刻画漂移效应时往往借助泊松跳跃机制:确定性漂移被等效为随机跳跃与随机等待时间的组合。由于等待时间具有随机性,漂移贡献可能引入额外方差,从而表现为一定程度的人工扩散。当系统噪声较强时,该影响通常相对次要;但在扩散项很小的情形,人工扩散可能显著扭曲数值行为。为说明这一现象,考虑加性噪声 SDE
\begin{equation}\label{eq:add_noise}
	dX_{t}=\mu\left(X_{t}\right) d t+\sigma d W_{t},
\end{equation}
其中 $\sigma=\operatorname{diag}\left(\sigma_{11}, \sigma_{22}, \ldots, \sigma_{nn}\right)$,并且对 $i=1,2,\ldots,n$ 有 $\sigma_{ii}\in\mathbb{R}^{+}$。当对方程\eqref{eq:add_noise}应用 $Q_u$ 类 CTRW 方案时,漂移离散化会带来额外的扩散效应。针对这一不足,Zu\cite{zu2023random}提出改进的 $\tilde{Q}_u$ 跳跃格式:在保留 $Q_u$ 基本结构的同时,通过对\eqref{eq:add_noise}中的噪声强度 $\sigma$ 作相应修正以补偿额外扩散,等价于为跳跃过程加入校正后的扩散项。相关分析表明,$\tilde{Q}_u$ 相较原始 $Q_u$ 与 $Q_c$ 在精度上更具优势,尤其在小噪声情形下仍能保持良好适用性;并且在平均驻留时间的渐近展开方面,该格式具有更好的性质,可在网格加密时实现较高阶的逼近效果。

综上,对于漂移仅满足局部 Lipschitz(而非全局 Lipschitz)并可能具有超线性增长的随机系统,现有研究主要形成两类离散策略:一类是通过驯服、截断、停时等机制改造的时间离散显式格式;另一类是基于生成元离散的空间离散 CTRW/SSA 跳跃格式(包括 $Q_u$、$Q_c$ 及其改进形式)。在实际计算中,一个关键问题是:对给定 SDE,何种离散方式能够在稳定性、精度与计算效率之间取得更优平衡。本文将围绕这一问题,对时间离散与空间离散两类方法在稳定性、精度与收敛性等方面进行对比分析。



	\subsection{本文的主要研究内容}
	
	本文围绕随机微分方程的数值模拟,分别从时间尺度与空间尺度两个角度,对两类离散格式的性能进行系统比较。具体而言,一方面,我们通过考察系统在跨越固定空间距离 \(h\) 时所需的时间,并与连续模型的真实时间进行对比,用以评估两种格式在时间尺度上的逼近能力;另一方面,我们通过分析典型快-慢随机系统的相空间动力学行为,借助逃逸概率函数作为弱量指标,比较两类格式在长期行为和转迁结构上的刻画效果。
	
	在时间尺度的比较方面,本文首先选取一维立方振子模型作为原型问题。在该模型中,当 \(|X_t| \gg 1 \)时,方程~\eqref{eq:add_noise} 的演化主要由漂移项主导,此时可用
	\begin{equation}\label{drift dominated}
		d X_{t}=\mu\left(X_{t}\right)\, d t
	\end{equation}
	给出的漂移主导常微分方程来近似随机动力学。由此得到的跨越距离 \(h\) 所需时间记为 \(t^{e}\),其主导部分记为 \(t^{*}\)。对于基于生成元空间离散的连续时间随机游走(CTRW)方法,我们将从 \(x\) 运动到 \(x+L\) 的平均等待时间记为 \(t^{u}\),并将 \(t^{u}\) 与 \(t^{e}\) 进行比较。
	
	对于漂移和扩散项仅满足局部 Lipschitz 且可能超线性增长的随机微分方程,经典的 Euler Maruyama 格式在强意义和弱意义下都可能失效或发散。为此,本文采用截断 Euler Maruyama(truncated EM)格式和驯服 Euler(tamed EM)格式作为改进的时间步进方法。记在这两种时间离散方法下,从 \(x\) 运动到 \(x+L\) 的平均首达时间分别为 \(t^{\Delta}\) 和 \(t^{\delta}\),并将它们与漂移主导时间 \(t^{e}\) 进行对比。该评价准则可理解为:一方面比较空间离散方法中由生成元 \(Q\) 所诱导的平均停留时间 \(t^{u}\),另一方面比较时间离散方法中由等效时间步长产生的平均首达时间 \(t^{\Delta}, t^{\delta}\),从而在相同空间跨越距离下直观比较两类算法的时间刻画能力与数值偏差。
	
	在动力学行为的比较方面,本文选取具有典型快-慢结构的随机 Canard 系统作为测试模型。该系统在相空间中呈现出沿慢流形缓慢演化、随后发生快速跃迁的典型 Canard 现象。为了评估两类格式对真实动力学行为的刻画能力,本文引入逃逸概率函数(committor function)作为弱量指标:通过构造对应于两个不同慢流形区域的集合 \(A,B\),比较空间离散与时间离散格式所得到的 逃逸概率 函数与参考解之间的偏差,从定性结构(相空间图像)和定量误差(work--error 曲线)两个层面评价两类方法在长期行为描述上的优劣。
	
	本文结构安排如下:第一章介绍课题背景和研究动机;第二章总结随机微分方程数值解的理论基础,包括解的存在唯一性、基本数值方法及其收敛性理论;第三章针对一维立方振子模型,从理论上分析固定空间跨越距离下的运行时间比较准则,并给出相应的数值实验;第四章以随机 Canard 快-慢系统为例,对比分析空间离散与时间离散方法在动力学行为和 逃逸概率 函数上的表现,并通过数值实验进行验证;第五章对全文工作进行总结,并对进一步的研究方向进行展望。
	
	
	\newpage
	%%=====================================================================================================%%
	%%
	%%                第  2 节
	%%
	%%=====================================================================================================%%
	
	
	 \section{预备知识}
	 
	 
	

	\subsection{It\^{o}扩散的无穷小生成元}
	
	 \begin{definition}[马尔可夫过程] 
	 考虑存在一个概率空间 $(\Omega, \mathcal{F}, \mathbb{P})$ ,设 $\left\{X_{t}\right\}_{t \geq 0}$ 是取值于具有 $\sigma$-代数 $\Sigma$ 的状态空间 $\mathcal{X}$ 的随机过程,而且其是$\mathcal{F}_{t^{-}}$适应的,如果对所有 $x \in \mathcal{X}, 0 \leq s \leq t$ 和 $A \in \Sigma$ ,有
	
	 $$
	 \mathbb{P}_{x}\left(X_{t+s} \in A \mid \mathcal{F}_{s}\right)=\mathbb{P}_{x}\left(X_{t+s} \in A \mid X_{s}\right)
	 $$
	
	 几乎必然成立,则称 $\left\{X_{t}\right\}_{t \geq 0}$ 具有马尔可夫性质。
	 具有以上连续时间马尔可夫性的随机过程是一个连续时间的马尔可夫链。
	 \end{definition}
	
	 
	 \begin{definition}[转移函数]
	 	设 $\mathcal{X}$ 为可数状态空间,$\{X_t\}_{t\ge0}$ 为取值于 $\mathcal{X}$ 的随机过程。若对任意 $i,j\in\mathcal{X}$ 以及 $t\ge0$,
	 	\[
	 	定义p_t(i,j) := \mathbb{P}_i\bigl(X_t = j\bigr)
	 	\]
	 	构成一族实值函数,并且满足:
	 	\begin{enumerate}
	 		\item[(i)] 非负性:对任意 $i,j\in\mathcal{X}$ 与 $t\ge0$,有
	 		\( 	p_t(i,j) \ge 0; \)
	 		\item[(ii)] 归一性:对任意 $i\in\mathcal{X}$ 与 $t\ge0$,有
	 		\(\sum_{j\in\mathcal{X}} p_t(i,j) = 1;\)
	 		\item[(iii)] 初始条件:对任意 $i,j\in\mathcal{X}$,有
	 		\(\lim_{t\downarrow 0} p_t(i,j) = p_0(i,j) = \delta_{ij};\)
	 		\item[(iv)] Chapman--Kolmogorov 方程:对任意 $i,j\in\mathcal{X}$ 与 $s,t\ge0$,有
	 		\(p_{s+t}(i,j) = \sum_{k\in\mathcal{X}} p_s(i,k)\,p_t(k,j).	\)
	 	\end{enumerate}
	 	则称 $\{p_t(i,j)\}_{t\ge0}$ 为该过程的转移函数。
	 \end{definition}
	 
	 由转移函数可以得到一族以时间 $t$ 为指标的矩阵 $\{P(t)\}_{t\ge0}$,其中对固定的 $t\ge0$,定义
	 \[
	 P(t) := \bigl(p_t(i,j)\bigr)_{i,j\in\mathcal{X}}, \qquad
	 P(t)_{ij} = p_t(i,j).
	 \]
	 
	 \begin{definition}[无穷小生成元]\label{def:q-matrix}
	 	在可数状态空间 $\mathcal{X}$ 上,矩阵 $Q = (q_{ij})_{i,j\in\mathcal{X}}$ 若满足:
	 	\begin{enumerate}
	 		\item[(i)] 对所有 $i\neq j$,有 $q_{ij}\ge 0$;
	 		\item[(ii)] 对所有 $i\in\mathcal{X}$,有
	 		\[
	 		\sum_{j\in\mathcal{X}} q_{ij} = 0,
	 		\]
	 	\end{enumerate}
	 	则称 $Q$ 为一个无穷小生成元。由 (ii) 可知
	 	\[
	 	q_{ii} = -\sum_{j\neq i} q_{ij}, \qquad
	 	q_i := -q_{ii} = \sum_{j\neq i} q_{ij},
	 	\]
	 	其中 $q_i$ 常被称为从状态 $i$ 的离开率。
	 \end{definition}
	 
	 设 $\{X_t\}_{t\ge0}$ 是一个具有无穷小生成元 $Q$ 的连续时间马尔可夫链。令
	 \[
	 0 = T_0 < T_1 < T_2 < \cdots
	 \]
	 为该过程的跳跃时刻,并定义
	 \[
	 Y_n := X_{T_n}, \qquad n\ge0.
	 \]
	 则 $\{Y_n\}_{n\ge0}$ 构成一个离散时间马尔可夫链,称为连续时间马尔可夫链 $\{X_t\}_{t\ge0}$ 的跳跃链。过程在第 $n$ 次跳跃前停留在状态 $Y_{n-1}$ 的时间定义为
	 \[
	 S_n := T_n - T_{n-1}, \qquad n\ge1,
	 \]
	 称为停留时间(holding time)。因此,与路径 $Y_0,\dots,Y_n$ 关联的停留时间序列为
	 \[
	 S_1,S_2,\dots,S_{n+1}.
	 \]
	 
	 \begin{lemma}[\cite{brereton2014stochastic}]
	 	给定一个具有无穷小生成元 $Q$ 的连续时间马尔可夫链,其跳跃链的转移矩阵 $J = (J_{ij})_{i,j\in\mathcal{X}}$ 定义为
	 	\[
	 	J_{ii} = 0, \qquad
	 	J_{ij} = \frac{q_{ij}}{q_i}, \quad i\neq j,
	 	\]
	 	其中 $q_i = \sum_{j\neq i} q_{ij}$ 为状态 $i$ 的离开率。也就是说,给定当前状态为 $i$,下一个跳跃落在状态 $j\neq i$ 的概率为 $q_{ij}/q_i$。
	 \end{lemma}
	 
	 \begin{lemma}[\cite{brereton2014stochastic}]
	 	在给定跳跃链路径 $Y_0,Y_1,\ldots,Y_n$ 的条件下,对每个 $k=1,\ldots,n+1$,停留时间
	 	\[
	 	S_k = T_k - T_{k-1}
	 	\]
	 	相互独立,且
	 	\[
	 	S_k \sim \mathrm{Exp}\bigl(q_{Y_{k-1}}\bigr),
	 	\]
	 	即 $S_1,S_2,\ldots,S_{n+1}$ 分别是参数为 $q_{Y_0},q_{Y_1},\ldots,q_{Y_n}$ 的指数型随机变量。
	 \end{lemma}
	 
%============================================
%	 
%	 我们可以使用半群理论来研究齐次马尔可夫过程。对于 $t \geq 0$,在可测有界函数 $f$ 上定义算子 $T_t$ 如下:
%	 \[
%	 (T_t f)(x) = \mathbb{E}^x [f(X_t)] = \int_{\mathbb{R}} f(y) \, p(t, x, dy).
%	 \]
%	 根据定义,$T_0$ 是恒等算子。此算子族 $\{T_t\}$ 对任意 $t, s \geq 0$ 满足如下半群性质:
%	 \begin{align*}
%	 	(T_t \circ T_s) f(x) &= \int_{\mathbb{R}} p(t, x, dy) \left( \int_{\mathbb{R}} f(z) \, p(s, y, dz) \right) \\
%	 	&= \int_{\mathbb{R}} f(z) \left( \int_{\mathbb{R}} p(t, x, dy) \, p(s, y, dz) \right) \\
%	 	&= \int_{\mathbb{R}} f(z) \, p(t+s, x, dz) = T_{t+s} f(x).
%	 \end{align*}
%	 最后一个等式由 Chapman-Kolmogorov 方程保证。这些参数化的算子 $\{T_t\}_{t \geq 0}$ 被称为 算子半群。
%	 
%	 对于定义在合适 Banach 空间 $B$ 上的算子半群 $\{T_t\}_{t \geq 0}$,可以如下定义其无穷小生成元 $\mathcal{L}$(关于抽象理论的更多细节,参见 \cite{Yos95})
%	 
%	 \begin{equation} \label{eq:inf-generator-def}
%	 	\mathcal{L}f = \lim_{t \to 0^+} \frac{T_t f - f}{t}
%	 \end{equation}
%	 该定义域是 $B$ 的一个合适子空间,即 $\mathcal{L}$ 的定义域。此处的收敛是指范数收敛:
%	 \[
%	 \lim_{t \to 0^+} \left\| t^{-1}(T_t f - f) - \mathcal{L}f \right\| = 0.
%	 \]
%============================================


	 \begin{theorem}[$Q$-过程的无穷小生成元] \label{ex:q-process-generator}
	 	考虑状态空间 $S = \{1, 2, \dots, I\}$ 上的 $Q$-过程 $X_t$,其生成元 $Q$ 由 \eqref{def:q-matrix}定义。我们有:
	 	\begin{align*}
	 		(\mathcal{A}f)(i) &= \lim_{t \to 0^+} \frac{\mathbb{E}^i [f(X_t)] - f(i)}{t} \\
	 		&= \lim_{t \to 0^+} \frac{1}{t} \left[ \sum_{j \in S} (P_{ij}(t) - \delta_{ij}) f(j) \right] \\
	 		&= \sum_{j \in S} q_{ij} f(j), \quad i \in S.
	 	\end{align*}
	 	推导最后一步用到了 \eqref{def:q-matrix}。因此,生成元 $Q$ 正是 $X_t$ 的无穷小生成元。
	 \end{theorem}
	
	标准的$n$ 维的 It\^{o} 扩散过程是
	 \begin{equation}
		 d X_{t}=\mu\left(X_{t}, t\right) d t+\sigma\left(X_{t}, t\right) d W_{t} 
		 \end{equation}
	 的解,其中 $X_{t} \in \mathbb{R}^{n}, \mu(x, t):[0, T] \times \mathbb{R}^{n} \rightarrow \mathbb{R}^{n}, \sigma(x, t):[0, T] \times \mathbb{R}^{n} \rightarrow \mathbb{R}^{n \times n},\left\{W_{t}\right\}_{t \geq 0}$是 $n$ 维 Brownian 运动。	
	
	\begin{definition}[It\^{o} 扩散的无穷小生成元] \label{thm:infinitesimal-generator-ito}
		
		考虑定义在 $\Omega \in \mathbb{R}^{n}$ 上的 $n$ 维随机微分方程
		\begin{equation} \label{eq:SDE}
			d X_{t}=\mu\left(X_{t}\right) d t+\sigma\left(X_{t}\right) d W_{t}, \quad X(0) \in \Omega 
		\end{equation}	
		其中 $W_{t}$ 是 $n$ 维 Brownian 运动,$\mu: \Omega \rightarrow \mathbb{R}^{n}$ 和 $\sigma: \Omega \rightarrow \mathbb{R}^{n \times n}$ 分别为漂移系数和扩散系数。
		
		若函数 $f \in C^2_B(\mathbb{R}^n)$(即 $f$ 是具有有界二阶偏导数的二次连续可微函数),则 $f$ 属于无穷小生成元 $\mathcal{L}$ 的定义域 $\mathscr{D}_{\mathcal{L}}$,且
		\begin{align} \label{infinitesimal-generator}
			Lf(x) &= Df(x)^T \mu(x) + trace(D^2f(x)\sigma(x)\sigma(x)^T) \notag \\
			&=\sum_{1 \leq i \leq n} \mu_{i}(x) \frac{\partial f(x)}{\partial x_{i}}+\sum_{1 \leq i, j \leq n} M_{i, j}(x) \frac{\partial^{2} f(x)}{\partial x_{i} \partial x_{j}}
		\end{align}
		其中 $M(x)=\frac{1}{2}\sigma(x) \sigma(x)^{T}$ 。
	\end{definition}
	
	\subsection{空间离散格式}
	 利用偏微分方程(PDE)可以从期望值的角度研究随机微分方程\eqref{eq:SDE}的扩散过程。众所周知,$X_{t}$ 的条件期望的时间演化可以用带有初始条件 $u(0, x)=f(x)$ 的 Kolmogorov 方程	
	 \begin{equation} \label{eq:Kolmogorov}
	 \frac{\partial u}{\partial t}(t, x)=L u(t, x) 
	 \end{equation}	
	 描述。Kolmogorov 方程\eqref{eq:Kolmogorov}的解 $u(t, x)=\mathbb{E}_{x} f(X(t))$ ,其中 $\mathbb{E}_{x}$ 表示在 $X(0)=x$条件下 $X(t)$ 的期望。
	 还可以从偏微分方程角度来研究随机微分方程\eqref{eq:SDE}解的动力学。设 $L^{*}$ 是 $L$的自伴算子,由	
	 \begin{equation} \label{adjoint}
	 L^{*} f(x)=-\sum_{1 \leq i \leq n} \frac{\partial\left(\mu_{i} f\right)(x)}{\partial x_{i}}+\sum_{1 \leq i, j \leq n} \frac{\partial^{2}\left(M_{i, j} f\right)(x)}{\partial x_{i} \partial x_{j}} 
	 \end{equation}
	 给出,则 $X(t)$ 的概率密度函数的时间演化可以用 Fokker-Planck 方程
	 \begin{equation}\label{eq:Fokker-Planck}
	 \frac{\partial p}{\partial t}(t, x)=L^{*} p(t, x), \quad p(0, x)=p_{0}(x) 
	 \end{equation}
	 描述,其中 $p_{0}(x)$ 是 $X_{0}$ 的概率密度函数。

Bou-Rabee 与 Vanden-Eijnden\cite{bou2018continuous}指出,除了传统的时间步进格式之外,还可以从 Kolmogorov 方程出发,通过对空间变量进行离散化来构造一类连续时间随机过程,用以近似描述随机微分方程的演化。其基本思路是:先在离散状态空间上建立一个可实现的生成元矩阵 $Q$,再用随机模拟算法精确生成由 $Q$ 所诱导的连续时间马尔可夫跳跃过程,从而得到对原 SDE 轨道的数值近似。

具体而言,设在离散状态空间中,每个状态 $x$ 具有 $K$ 个可能的跳跃通道
\[
x\rightarrow y_i(x),\qquad 1\le i\le K.
\]
据此定义离散空间生成元 $Q$ 对测试函数 $f$ 的作用为
\begin{equation}\label{eq:Qf(x)_rewrite}
	Q f(x):=\sum_{i=1}^{K} q\left(x, y_{i}(x)\right)\Bigl(f\left(y_{i}(x)\right)-f(x)\Bigr),
\end{equation}
其中 $q:\Omega\times\Omega\to[0,\infty)$ 为跳跃速率函数。若 $Q$ 可视为连续系统无穷小生成元 $L$ 的良好近似,则可将 $Q$ 理解为某连续时间马尔可夫链的生成元,并据此采用标准随机模拟生成样本路径。给定当前时刻状态 $X(t)=x$,算法可概括为以下两步:

\emph{(i) 生成等待时间.} 计算总跳跃率
\[
\lambda(x)=\sum_{i=1}^{K} q\left(x, y_{i}(x)\right),
\]
并取等待时间 $\tau$ 为参数 $\lambda(x)$ 的指数分布随机变量。

\emph{(ii) 选择跳跃通道并更新状态.} 在条件 $Y(t)=x$ 下,以
\[
\mathbb{P}\!\left(Y(t+\tau)=y_{i}(x)\mid Y(t)=x\right)
=\frac{q\left(x, y_{i}(x)\right)}{\lambda(x)}
\]
选择通道 $i$,并令系统从 $x$ 跳至 $y_i(x)$。

以 $n$ 维随机微分方程\eqref{eq:SDE}为例,Bou-Rabee 与 Vanden-Eijnden\cite{bou2018continuous}分别给出了基于有限差分离散的 $Q_u$ 方案与基于有限体积离散的 $Q_c$ 方案。设 $\{e_i\}_{i=1}^n$ 为 $\mathbb{R}^n$ 的标准基向量,则每个坐标方向对应一个通道方向,并允许沿该方向前向或后向跳跃,因此通道总数为 $2n$。记 $h_i^{\pm}(x)$ 为状态 $x$ 处沿第 $i$ 个通道方向的前向/后向空间步长,并令
\[
h_i(x)=\frac{h_i^{+}(x)+h_i^{-}(x)}{2}
\]
为平均步长。设 $\sigma_i(x)=\sigma e_i$ 表示噪声矩阵 $\sigma(x)$ 的第 $i$ 列向量(即第 $i$ 个通道方向),并引入变换后的漂移场 $\tilde{\mu}(x)$,其逐点由
\[
M(x)\tilde{\mu}(x)=\mu(x),\qquad M(x)=0.5\,\sigma(x)\sigma(x)^T
\]
定义,其中 $M(x)$ 为 $n\times n$ 的扩散矩阵。于是,有限差分离散化得到的 $Q_u$ 生成元写为
\begin{align}\label{algorithm:Q_u_rewrite}
	Q_{u} f(x)= & \sum_{i=1}^{n}\left(\frac{\left(\tilde{\mu}(x)^{T} \sigma_{i}(x)\right) \vee 0}{h_{i}^{+}(x)}+\frac{1}{h_{i}^{+}(x) h_{i}(x)}\right)\Bigl(f\bigl(x+h_{i}^{+}(x) \sigma_{i}(x)\bigr)-f(x)\Bigr) \notag \\
	& +\left(-\frac{\left(\tilde{\mu}(x)^{T} \sigma_{i}(x)\right) \wedge 0}{h_{i}^{-}(x)}+\frac{1}{h_{i}^{-}(x) h_{i}(x)}\right)\Bigl(f\bigl(x-h_{i}^{-}(x) \sigma_{i}(x)\bigr)-f(x)\Bigr),
\end{align}
其中 $a\vee b=\max(a,b)$,$a\wedge b=\min(a,b)$。

相应地,有限体积离散所得到的 $Q_c$ 方案为
\begin{align}\label{algorithm:Q_c_rewrite}
	Q_{c} f(x)= & \sum_{i=1}^{n} \frac{1}{h_{i}^{+}(x) h_{i}(x)}
	\exp \left(\frac{h_{i}^{+}(x)}{2} \tilde{\mu}(x)^{T} \sigma_{i}(x)\right)
	\Bigl(f\bigl(x+h_{i}^{+}(x) \sigma_{i}(x)\bigr)-f(x)\Bigr) \notag \\
	& +\frac{1}{h_{i}^{-}(x) h_{i}(x)}
	\exp \left(-\frac{h_{i}^{-}(x)}{2} \tilde{\mu}(x)^{T} \sigma_{i}(x)\right)
	\Bigl(f\bigl(x-h_{i}^{-}(x) \sigma_{i}(x)\bigr)-f(x)\Bigr).
\end{align}

进一步地,考虑在 $\mathbb{R}^n$ 上边长为 $h$ 的 Cartesian 网格上,具有加性噪声的随机微分方程
\[
d X_{t}=\mu\left(X_{t}\right) d t+\sigma d W_{t},
\]
其中 $\sigma=\operatorname{diag}(\sigma_{11},\sigma_{22},\ldots,\sigma_{nn})$ 为噪声强度,且对 $i=1,2,\ldots,n$ 有 $\sigma_{ii}\in\mathbb{R}^+$。Zu\cite{zu2023random}在该设置下提出改进的有限差分离散化格式 $\widetilde{Q}_u$,其生成元可写为
\begin{align}\label{algorithm:Q_u_tilde_rewrite}
	\widetilde{Q}_{u} f(x)= & \sum_{i=1}^{n}\left(\frac{\mu_{i}(x) \vee 0}{h}+\frac{M_{i i}}{h^{2}}\right)\Bigl(f\left(x+h e_{i}\right)-f(x)\Bigr) \notag \\
	& +\left(-\frac{\mu_{i}(x) \wedge 0}{h}+\frac{M_{i i}}{h^{2}}\right)\Bigl(f\left(x-h e_{i}\right)-f(x)\Bigr),
\end{align}
其中
\[
M_{ii}=0.5\left(\sigma_{ii}^{2}-\left|\mu_{i}(x)\right| h\right)\vee 0.
\]

	 \subsubsection{连续时间随机游走方法的算法}
	 
	 定义 $Q_{u}$ 步骤,$Q_{c}$ 步骤,$\tilde{Q}_u$ 步骤如下:
	 \begin{align*}
	 	& Q_{u} \text { 步骤 }: M_{i i} \leftarrow 0.5 * \sigma_{i i}^{2} \\
	 	& \quad q_{i 1} \leftarrow\left(\mu_{i}(X) \vee 0\right) / h+M_{i i} / h^{2}, q_{i 2} \leftarrow\left(-\mu_{i}(X) \wedge 0\right) / h+M_{i i} / h^{2}  \\
	 	& Q_{c} \text { 步骤 }: M_{i i} \leftarrow 0.5 * \sigma_{i i}^{2} \\
	 	& \quad q_{i 1} \leftarrow M_{i i} / h^{2} \exp \left(\mu_{i}(X) h /\left(2 * M_{i i}\right)\right), q_{i 2} \leftarrow M_{i i} / h^{2} \exp \left(-\mu_{i}(X) h /\left(2 * M_{i i}\right)\right) \\
	 	& \widetilde{Q}_{u} \text { 步骤 }: M_{i i} \leftarrow 0.5 *\left(\sigma_{i i}^{2}-\left|\mu_{i}(X)\right| h\right) \vee 0   \\
	 	& \quad q_{i 1} \leftarrow\left(\mu_{i}(X) \vee 0\right) / h+M_{i i} / h^{2}, q_{i 2} \leftarrow\left(-\mu_{i}(X) \wedge 0\right) / h+M_{i i} / h^{2} 
	 \end{align*}
	
连续时间随机游走方法的具体步骤如算法~\ref{alg:ctrw} 所示。
\SetAlgoNlRelativeSize{0}
\SetNlSty{}{}{}
\SetAlgoNoLine
\SetNlSkip{1em}          % 控制编号右边的空隙

\begin{algorithm}[htbp]
	\caption{连续时间随机游走方法的算法}\label{alg:ctrw}
	\KwInput{初始值:$X_0$,结束时间:$T$}
	\KwOutput{一条轨线 $X = \{ X_{t_k} \}_{k \in \mathbb{N}}$}
	$t \leftarrow 0$, \quad $X \leftarrow X_0$\;
	\While{$t < T$}{
		\For{$i \leftarrow 1$ \KwTo $n$}{
			根据 $Q_u$ 步骤(或 $\widetilde{Q}_u$ 步骤)计算 $q_{i1}$ 和 $q_{i2}$\;
		}
		$\lambda \leftarrow \sum_{i=1}^n (q_{i1} + q_{i2})$\;
		生成均匀分布随机变量 $r_1, r_2 \sim \mathcal{U}(0,1)$, $\tau \leftarrow -\ln(r_1)/\lambda$, $t \leftarrow t + \tau$\;
		$i, j \leftarrow \min_i \min_j \left( \sum_{l=1}^j q_{il} > r_2 \lambda \right)$,\quad 
		$X \leftarrow X + (-1)^{j-1} h e_i$\;
	}
\end{algorithm}


	
	
	
%	 \subsubsection*{自适应网格细化}
%	 一些定义在 $\Omega=\mathbb{R}_{+}$上的一维标量随机微分方程问题,在某些参数值下可能在原点有奇点。为了有效地解决这种奇异性,Bou-Rabee 和 Vanden-Eijnden ${ }^{[24]}$ 提出可使用自适应网格细化,即可变步长的,在奇点 $x=0$ 附近包含更多点的无限网格 $S=\left\{x_{i}\right\} \in \mathbb{R}^{+}$。他们通过将网格 $S$ 映射到对数空间来构造该网格,以此获得了 $\mathbb{R}$上变换的网格 $\hat{S}$ ,定义如下
%	
%	 $$
%	 \hat{S}=\left\{\xi_{i}\right\}, \quad \xi_{i}=\log \left(x_{i}\right), \quad \forall x_{i} \in S
%	 $$
%	
%	 注意,当 $\xi_{i} \rightarrow-\infty\left(\xi_{i} \rightarrow \infty\right)$ ,有 $x_{i} \rightarrow 0\left(x_{i} \rightarrow \infty\right)$ 。假设对数空间中变换的网格 $\hat{S}$上,相邻网格点之间的距离是固定的,用 $\delta \xi$ 表示,即 $\forall i \in \mathbb{Z}, \delta \xi=\xi_{i+1}-\xi_{i}$ 。由于
%	
%	 $$
%	 x_{i+1}=\exp \left(\xi_{i+1}\right)=\exp (\delta \xi) \exp \left(\xi_{i}\right)=\exp (\delta \xi) x_{i}
%	 $$
%	
%	 因此
%	
%	 $$
%	 h^{+}\left(x_{i}\right)=(\exp (\delta \xi)-1) x_{i}, \quad h^{-}\left(x_{i}\right)=(1-\exp (-\delta \xi)) x_{i}, \quad h\left(x_{i}\right)=\sinh (\delta \xi) x_{i}
%	 $$
%	
%	 对于具有乘性噪声的一维随机微分方程,使用上述的自适应网格细化。
%	
%	
	
	
	\subsection{改进的 Euler--Maruyama 数值格式}\label{subsec:improved-em}
	
%	本节考虑如下 It\^{o} 型随机微分方程
%	\begin{equation}\label{eq:sde-general}
%		dX_t = \mu(X_t)\,dt + \sigma(X_t)\,dW_t,\qquad X_0 = x_0\in\mathbb{R}^n,
%	\end{equation}
%	其中 $\mu:\mathbb{R}^n\to\mathbb{R}^n$ 为漂移项,$\sigma:\mathbb{R}^n\to\mathbb{R}^{n\times m}$ 为扩散项,$W_t$ 为 $m$ 维 Brownian 运动。
	
	考虑方程\eqref{eq:SDE},记时间网格 $t_k = k h$,$k=0,1,\dots,N$,步长 $h=T/N$,Brownian 增量
	\[
	\Delta W_k := W_{t_{k+1}} - W_{t_k}.
	\]
	
	经典 Euler Maruyama格式为
	\[
	X_{k+1} = X_k + \mu(X_k)\,h + \sigma(X_k)\,\Delta W_k.
	\]
	在 $\mu,\sigma$ 全局 Lipschitz 且线性增长的条件下,Euler-Maruyama 具有 $1/2$ 阶强收敛及 1 阶弱收敛。然而,当漂移仅局部 Lipschitz、并在无穷远处呈超线性增长时,即便原方程 \eqref{eq:SDE} 本身是良态并具有唯一强解,Euler-Maruyama格式也可能在有限时间内数值解的矩发散而失去强收敛性。Hutzenthaler 等人的工作表明,在这类非全局 Lipschitz 情形下,Euler-Maruyama 甚至可能在有限时间点 $T$ 处满足
	\[
	\lim_{N\to\infty}
	\Bigl(\mathbb{E}\bigl|X_T - X^{\mathrm{EM}}_N\bigr|^2\Bigr)^{1/2} = +\infty,
	\]
	从而不能作为可靠的强逼近方法。为此,人们提出了对 Euler-Maruyama 进行适当修正的改进,其中具有代表性的是驯服 Euler(Tamed Euler-Maruyama)方法 \cite{hutzenthaler2012strong} 和截断 Euler Maruyama(Truncated Euler-Maruyama)方法 \cite{mao2015truncated}。下面分别对这两种格式作简要介绍。
	
	\subsubsection{驯服 Euler 方法}
	
	Hutzenthaler, Jentzen 和 Kloeden 在 \cite{hutzenthaler2012strong} 中考虑漂移满足单侧 Lipschitz 条件和多项式增长的情形。更具体地,假设存在常数 $K\ge0$ 与整数 $q\ge0$,使得对任意 $x,y\in\mathbb{R}^n$,
	\begin{align}
		\langle x-y,\mu(x)-\mu(y)\rangle &\le K|x-y|^2, \label{cond:onesided}\\
		\|\mu'(x)\| &\le K\bigl(1+|x|^q\bigr), \label{cond:poly-growth}\\
		\|\sigma(x)-\sigma(y)\| &\le K|x-y|.
	\end{align}
	条件 \eqref{cond:onesided} 为全局单侧 Lipschitz 条件,它保证了漂移在方向上不会产生爆炸性发散;条件 \eqref{cond:poly-growth} 则允许 $\mu$ 超线性增长,但其导数仅为多项式阶。此时 SDE \eqref{eq:SDE} 仍存在唯一强解,并具有足够的矩有界性。
	
	在上述假设下,为克服 Euler-Maruyama 方法在大漂移区域由于过大的单步更新而导致的数值发散,\cite{hutzenthaler2012strong} 将漂移增量进行驯服,给出如下数值格式:令
	\[
	t_k = k\delta,\qquad \delta = T/N,
	\]
	驯服 Euler 近似 $\{X_k^{\delta}\}_{k=0}^N$ 定义为
	\begin{equation}\label{scheme:tamed}
		X_{k+1}^{\delta}
		= X_k^{\delta}
		+ \frac{\mu(X_k^{\delta})\,\delta}{1+\delta\,\|\mu(X_k^{\delta})\|}
		+ \sigma(X_k^{\delta})\,\Delta W_k,\qquad k=0,1,\dots,N-1,
	\end{equation}
	其中 $\Delta W_k = W_{t_{k+1}}-W_{t_k}$,$X_0^{\delta}=X_0$。可以看到,无论 $\|\mu(X_k^{\delta})\|$ 多大,总有
	\[
	\biggl\|
	\frac{\mu(X_k^{\delta})\,\delta}{1+\delta\,\|\mu(X_k^{\delta})\|}
	\biggr\|
	\le 1,
	\]
	从而避免了在强漂移区域中出现过大的单步更新;扩散项保持不变,因此该方法仍为显式格式。
	
	\begin{theorem}[Tamed Euler-Maruyama 的强收敛性,见 \cite{hutzenthaler2012strong}]
		在假设 \eqref{cond:onesided}--\eqref{cond:poly-growth} 及 $\sigma$ 全局 Lipschitz 的条件下,对任意 $p\ge1$,存在常数 $C_p>0$ 与步长上界 $\delta_0>0$,使得对所有 $0<\delta\le\delta_0$,
		\begin{equation}\label{eq:tamed-strong}
			\max_{0\le k\le N}
			\Bigl(\mathbb{E}\bigl|X_{t_k}-X_k^{\delta}\bigr|^p\Bigr)^{1/p}
			\le C_p\,\delta^{1/2}.
		\end{equation}
		也就是说,驯服 Euler 格式在上述非全局 Lipschitz 情形下仍保持 $1/2$ 阶强收敛,与标准 EM 在全局 Lipschitz 情形下的收敛阶一致。
	\end{theorem}
	
	与隐式 Euler 相比,\eqref{scheme:tamed} 在每一步不需要求解非线性方程,计算上与显式 Euler-Maruyama 基本同价,因此在大规模 Monte Carlo 模拟中具有明显的效率优势。
	
	\subsubsection{截断 Euler--Maruyama 方法}
	
	Hutzenthaler 等人的结果说明:在漂移仅局部 Lipschitz 且超线性增长时,经典 Euler-Maruyama 的矩可能发散。Mao 在 \cite{mao2015truncated} 中进一步考虑更一般的局部 Lipschitz 情形,并假设漂移与扩散满足 Khasminskii 型条件:存在常数 $K>0$、$p>2$,使得
	\begin{equation}\label{Khasminskii-again}
		x^{\top}\mu(x)+\frac{p-1}{2}\|\sigma(x)\|^2
		\le K\bigl(1+|x|^2\bigr),
		\qquad \forall x\in\mathbb{R}^n.
	\end{equation}
	在 \eqref{Khasminskii-again} 条件下,方程 \eqref{eq:SDE} 仍存在唯一强解,并具有所有阶矩有界。然而,直接应用 Euler-Maruyama 仍可能发散。为此,\cite{mao2015truncated} 先对漂移和扩散在空间上做截断,使其在 $\mathbb{R}^n$ 上有界,再在截断系数上使用标准 Euler-Maruyama。
	
	具体地,令步长
	\[
	t_k = k\Delta,\qquad 0<\Delta\le\Delta^\ast\le1.
	\]
	第一步,构造一个严格递增连续函数
	\[
	\phi:\mathbb{R}^+\to\mathbb{R}^+,
	\]
	满足
	\begin{equation}\label{eq:phi-growth}
		\sup_{|x|\le r}\bigl(\|\mu(x)\|\vee\|\sigma(x)\|\bigr)\le \phi(r),
		\qquad \forall r\ge0,
	\end{equation}
	且 $\phi(r)\to\infty$ 当 $r\to\infty$。第二步,选取严格递减函数
	\[
	\psi:(0,\Delta^\ast]\to(0,\infty)
	\]
	满足
	\begin{equation}\label{eq:psi-cond}
		\psi(\Delta^\ast)\ge\phi(2),\qquad
		\lim_{\Delta\to0}\psi(\Delta)=\infty,\qquad
		\Delta^{1/4}\psi(\Delta)\le1.
	\end{equation}
	据此定义截断半径
	\begin{equation}\label{eq:R-Delta}
		R_\Delta := \phi^{-1}\!\bigl(\psi(\Delta)\bigr),
	\end{equation}
	以及投影算子
	\begin{equation}\label{eq:Pi-Delta}
		\Pi_\Delta(x)
		:=\bigl(|x|\wedge R_\Delta\bigr)\frac{x}{|x|},
		\qquad
		\frac{x}{|x|}:=
		\begin{cases}
			x/|x|, & x\neq0,\\[0.3em]
			0, & x=0.
		\end{cases}
	\end{equation}
	最后,定义截断后的漂移与扩散
	\begin{equation}\label{eq:trunc-coeff}
		\mu_\Delta(x):=\mu\bigl(\Pi_\Delta(x)\bigr),\qquad
		\sigma_\Delta(x):=\sigma\bigl(\Pi_\Delta(x)\bigr).
	\end{equation}
	由 \eqref{eq:phi-growth}--\eqref{eq:R-Delta} 可知,对任意 $x\in\mathbb{R}^n$,
	\[
	\|\mu_\Delta(x)\|\vee\|\sigma_\Delta(x)\|
	\le \psi(\Delta),
	\]
	即在每个固定步长 $\Delta$ 下,截断后的系数在全空间上均有界;同时,当 $\Delta\to0$ 时,截断半径 $R_\Delta\to\infty$,意味着截断对轨道的影响逐渐退化。
	
	在此基础上,截断 Euler--Maruyama 格式定义为
	\begin{equation}\label{scheme:trunc}
		X_{k+1}^{\Delta}
		= X_k^{\Delta}
		+ \mu_\Delta(X_k^{\Delta})\,\Delta
		+ \sigma_\Delta(X_k^{\Delta})\,\Delta W_k,\qquad k=0,1,\dots,N-1,
	\end{equation}
	其中 $X_0^{\Delta}=X_0$,$\Delta W_k=W_{t_{k+1}}-W_{t_k}$。
	
	\begin{theorem}[截断Euler-Maruyama方法的强收敛性,见 \cite{mao2015truncated}]
		假设 $\mu,\sigma$ 局部 Lipschitz,满足 Khasminskii 条件 \eqref{Khasminskii-again},并采用上述截断构造 \eqref{eq:phi-growth}--\eqref{scheme:trunc}。对任意 $q\in[2,p)$,存在常数 $C_q>0$ 与步长上界 $\Delta_0>0$,使得对所有 $0<\Delta\le\Delta_0$,
		\begin{equation}\label{eq:trunc-strong}
			\Bigl(\mathbb{E}\bigl|X_T - X_N^{\Delta}\bigr|^q\Bigr)^{1/q}
			\le C_q\,\Delta^{1/2}.
		\end{equation}
		也就是说,在同时满足局部 Lipschitz和Khasminskii 条件下,截断 Euler-Maruyama 格式同样达到 $1/2$ 阶强收敛。
	\end{theorem}
	
	与驯服 Euler 不同,截断 Euler-Maruyama 并不直接修改漂移增量的形式,而是通过投影 $\Pi_\Delta$ 将状态限制在一个随步长 $\Delta$ 放大的有界区域内,再在该区域上应用标准 Euler-Maruyama。这样一来,离散系数 $\mu_\Delta,\sigma_\Delta$ 全局有界,从而避免了数值解的爆炸;同时,由于 $R_\Delta\to\infty$,截断误差在 $\Delta\to0$ 时可以被严格控制。
	
	\medskip
	
	综上所述,驯服 Euler 与截断 Euler-Maruyama 都是在保持方法显式性的前提下,对漂移/扩散项进行适度修正,以恢复在非全局 Lipschitz 情形下的强收敛性。前者通过对漂移增量施加非线性缩放,后者则通过空间截断保证系数有界。本文后续章节将以 \eqref{scheme:tamed} 与 \eqref{scheme:trunc} 为时间离散的两种代表性的改进 Euler-Maruyama 格式,并在一维立方振子与随机 Canard 系统等模型上,从强收敛、长时间行为以及 逃逸概率/MFPT 等弱指标的角度,系统比较它们与空间离散方法的数值表现。
	
	
	\newpage
	
\section{固定空间距离跨越时间的比较}



考虑一维加性噪声 SDE
\begin{equation}\label{eq:sde-time-mfpt}
	dX_t=\mu(X_t)\,dt+\sigma\,dW_t,\qquad X_0=x,
\end{equation}
其中$x\gg 1$,本文在比较$t^e$与数值方法的时间分析时,仅在顺漂移\footnote{指目标在左(右)侧时,区间内有 $\mu(z)\le0$(或 $\ge0$),漂移将样本点整体推向目标;此时由后向Kolmogorov方程
	\[
	\mu(x)\,u'(x)+\tfrac12\sigma^2(x)\,u''(x)=-1,\quad u(L)=0,
	\]
可得 $u(x)$ 的主导项为 $u(x)\approx\int\frac{dx}{|\mu(x)|}$,与 $t^e$ 同标度。 若存在非零测度的“逆漂移”区间(漂移把样本点推离目标),则到达事件转为噪声主导的罕见激活,其MFPT呈Kramers–Arrhenius型指数律增长,不再与代数量级的 $t^e$ 可比\cite{Hanggi1990ReactionRateKramers}。 另一方面,在可实现的空间离散(CTRW)框架中,状态 $x_i$ 的跳出速率为 $\lambda(x_i)=\sum_{j\ne i}q_{ij}$、驻留时间为 $\mathrm{Exp}(\lambda)$,故 $\mathbb E[S_i]=1/\lambda(x_i)$;采用与流向一致的上风构造时,$\lambda(x_i)\approx|\mu(x_i)|/h$,于是平均驻留时间 $h/|\mu(x_i)|$ 与 $t^e$ 的主导标度一致\cite{bou2018continuous}}并且小位移的情形下采用该基准,%并且小位移对流主导($\mathrm{Pe}\gg1$)的情形下采用该基准。
为此设置目标阈值 $L=x-h$,且
\begin{equation}\label{eq:assum-local}
	0< h \ll x,\quad \mu(x)<0,\quad \text{并在 }[L,x]\text{ 上 }\mu\ \text{光滑}.
\end{equation}


\subsection{空间离散方法的平均驻留时间分析}
假设漂移项 $\mu$ 是可微的,并且满足一个耗散条件:
\begin{align}
	\quad \text{sign}(x)\,\mu(x) \to -\infty \quad \text{当 } |x|\to\infty.
\end{align}
因此,当 $|x|$ 足够大时,SDE 的动力学由漂移主导,并且渐近上,解 $X_t$ 在空间中跨越一个固定距离所需的时间可以通过分析以下常微分方程得到:
\begin{align} \label{eq:ODE dominated}
	\dot{X_t} = \mu(X_t), \quad X(0)=x.
\end{align}
方程 \eqref{eq:ODE dominated} 表明,在两个网格点之间的时间间隔满足:
\begin{align}\label{eq:te-def}
	t^e = \int_{x}^{L} \frac{ds}{|\mu(s)|}, 
\end{align}

为简单起见,假设网格点间距是均匀的,即 $\delta x_i^+ = \delta x_i^- = \delta$,在 $\delta\ll x$ 的局部极限下,对 \eqref{eq:te-def} 在 $s=x$ 处展开,
\begin{equation}\label{eq:te-local-expansion}
	t^e
	=\frac{\delta}{|\mu(x)|}
	-\frac{\mu'(x)}{2\,\mu(x)^2}\,\delta^2
	+\mathcal{O}(\delta^3),
\end{equation}
%其中首项 $t^*:=\delta/|\mu(x)|$ 是常数漂移的近似等待时间,同时$Q_u$ 的平均停留时间可写为
%\begin{equation}\label{t^u}
%t^u = t^{\ast}\;-\;\frac{(t^{\ast})^2}{\,2+|\mu_i|\,\delta x\,}.
%\end{equation}
%
%由\eqref{t^u} 可见,当 $|x_i|$ 很大时,$t^u$ 会趋近于 $t^{\ast}$,\cite{bou2018continuous}中仅给出了\(t^u\)与近似平均等待时间\(t^*\)的比较,
本文给出\(t^u\)与精确平均等待时间\(t^e\)的比较
%\begin{theorem}
%对任意 $\delta x>0, Q_u$ 的平均停留时间满足
%$$
%\frac{\lvert t^u - t^e \rvert}{t^e} \;\longrightarrow\; 0 
%\quad \text{当}\ |x_i|\to\infty.
%$$
%\end{theorem}
%

\begin{theorem}
	假设 $|\mu(x)|$ 足够大且
	\[
	\frac{\mu'(x)}{\mu(x)^2} \sim o\left(\frac{1}{\mu(x)}\right) \quad \text{当} \quad |x| \to 0.
	\]
	对于任意 $h > 0$,$t^u$ 和 $t^e$ 之间的相对误差满足
	\[
	\frac{|t^u - t^e|}{t^e} \sim O\left(\frac{\mu'(x)}{\mu(x)}\right) \to 0, \quad \text{当} \quad |x| \to \infty
	\]

\end{theorem}

\begin{proof}
\begin{align*}
	t^u 
	& = ((Q^u)_{i,i+1} + (Q^u)_{i,i-1})^{-1}  \\
	& = \frac{h^2}{\left( 2 - |\mu_i(x)| h \right) \vee 0 + |\mu_i(x)| h}  \to \frac{h}{|\mu_i(x)|} = t^e \quad \text{当} \quad |x| \to \infty 
\end{align*}

同时可以得到
\[
\frac{|t^u - t^e|}{t^e} = \frac{\left| \frac{1}{2} \frac{\mu'(\xi)}{\mu(\xi)^2} h^2 - \frac{h}{|\mu(x)|} \right|}{\frac{h}{|\mu(x)|}} \sim O\left(\frac{\mu'(x)}{\mu(x)}\right) \to 0, \quad \text{当} \quad |x| \to \infty. \quad \qed
\]

\end{proof}

此外,我们记 \(t^e\)的主部\( t*:= h/|\mu_i(x)| \),当 $\mu(x)$ 的主导项形式为 $-a x^{2p+1}$时,本文给出如下定理

\begin{theorem}
	假设 $\mu(x)=-a x^{2p+1}$,其中 $p \geq 0$ 且 $a > 0$,对于任意 $h > 0$,$t^u$ 和 $t^e$ 之间的相对误差满足
	\[
	\frac{|t^u - t^*|}{t^*}  \to 0, \quad \text{当} \quad |x| \to \infty .
	\]
	
\end{theorem}

\begin{proof}
	
注意此时\eqref{eq:te-local-expansion}中的第二项, 
	
\[
\frac{1}{2} \frac{\mu'(\xi)}{\mu(\xi)^2} h^2 =  \frac{2p + 1}{2a}\frac{h^2}{x^{2p + 2}}  \sim O(x^{-(2p + 2)}) 
\]
	
而此时其中的第一项

\[
\frac{h}{|\mu_i(x)|}  = \frac{h}{a x^{2p+1}} \sim O(x^{-(2p+1)})
\]
于是得到
\[
\lim_{x \to \infty}\frac{|t^u - t^*|}{t^*} =0.
\]
\end{proof}

\newpage

%===============================

\subsection{平均首达时间(MFPT)与 \(t^e\)的一致渐近性分析}\label{subsec:mfpt-pre-asym}

%	\paragraph{设置与记号.}
考虑一维 It\^o 扩散
\begin{equation}\label{eq:sde-mfpt-local}
	dX_t=\mu(X_t)\,dt+\sigma\,dW_t,\qquad X_0=x,
\end{equation}
其无穷小生成元为
\begin{equation}\label{eq:gen-1d-local}
	(L f)(x)=\mu(x)f'(x)+M f''(x),\qquad M=\frac{\sigma^{2}}{2}.
\end{equation}
我们关心从 $x$ 首达左侧阈值 $L=x-\delta$ 的 MFPT
\begin{equation}\label{eq:mfpt-def-local}
	\tau(x\to L):=\mathbb E^{x}\!\big[\inf\{t>0:\ X_t\le L\}\big],
\end{equation}
对 \eqref{eq:mfpt-def-local} 的 $\tau$,由生存概率法\cite{weinan2021applied}可得后向方程
\begin{equation}\label{eq:backward-mfpt-local}
	(L\tau)(y)=-1,\qquad y\in(L,\infty),
\end{equation}
并配边界条件
\begin{equation}\label{eq:bc-local}
	\tau(L)=0,\qquad \tau(y)\ \text{在 }y\to\infty\text{ 有界}.
\end{equation}
考虑一维立方振子模型,此时 $\sigma$ 为常数,\( \mu(x) = -x^3 \),于是由\eqref{eq:gen-1d-local} 得到
%\begin{equation}\label{eq:ode-mfpt-local}
%	\frac{\sigma^2}{2}\,\tau''(y)+\mu(y)\,\tau'(y)=-1,\qquad y\in(L,\infty).
%\end{equation}

%	\paragraph{特例 $\mu(y)=-y^3,\ \sigma=1$ 的\;MFPT 积分表示.}
%此时 $M=\tfrac12$,
%\eqref{eq:ode-mfpt-local} 化为
\begin{equation}\label{eq:ode-cubic}
	\frac{1}{2}\tau''(y)-y^{3}\tau'(y)=-1.
\end{equation}
令 $v(y):=\tau'(y)$,则
\begin{equation}\label{eq:v-ode}
	v'(y)-2y^3\,v(y)=-2.
\end{equation}
取积分因子 $I(y)=\exp\!\big(-\int 2y^3dy\big)=e^{-y^4/2}$,有
\[
\big(v(y)\,e^{-y^4/2}\big)'=-2\,e^{-y^4/2}.
\]
对 \([y,\infty)\) 积分并利用 $v(\infty)e^{-\infty}=0$(由 \eqref{eq:bc-local})得
\begin{equation}\label{eq:v-sol}
	v(y)=\tau'(y)=2\,e^{y^4/2}\int_{y}^{\infty}e^{-z^4/2}\,dz.
\end{equation}
再对 \([L,x]\) 积分并用 $\tau(L)=0$,得到 精确表达式
\begin{equation}\label{eq:tau-int-exact}
	\tau(x\to L)=2\int_{L}^{x} e^{y^{4}/2}
	\Big(\int_{y}^{\infty} e^{-z^{4}/2}\,dz\Big)\,dy\ 
\end{equation}

%	\paragraph{尾积分的 Laplace 渐近与主导项.}
记
\[
\mathcal I(y):=\int_{y}^{\infty} e^{-z^{4}/2}\,dz.
\]
对大 $y$,反复分部积分可得标准尾展开
\begin{equation}\label{eq:tail}
	\mathcal I(y)=\frac{e^{-y^{4}/2}}{2y^{3}}
	\Big(1-\frac{3}{2y^{4}}+\frac{33}{8y^{8}}+\mathcal O(y^{-12})\Big),\qquad y\to\infty.
\end{equation}
将 \eqref{eq:tail} 代入 \eqref{eq:tau-int-exact},得到被积函数
\begin{equation}\label{eq:F-expansion}
	e^{y^{4}/2}\mathcal I(y)
	=\frac{1}{2y^{3}}-\frac{3}{4y^{7}}+\frac{33}{16y^{11}}+\mathcal O(y^{-15}).
\end{equation}
因此
\begin{equation}\label{eq:tau-asym-start}
	\tau(x\to L)
	=2\!\int_{x-\delta}^{x}\!\!\Big(\frac{1}{2y^{3}}-\frac{3}{4y^{7}}+\cdots\Big)dy
	=\underbrace{\int_{x-\delta}^{x}\!y^{-3}dy}_{\mathsf{(A)}}
	\ -\ \underbrace{\frac{3}{2}\int_{x-\delta}^{x}\!y^{-7}dy}_{\mathsf{(B)}}
	\ +\ \cdots.
\end{equation}

把 \eqref{eq:tau-asym-start} 展开到 $\mathcal O(\delta^{2})$.	对 $\mathsf{(A)}$,在小区间 \([x-\delta,x]\) 上作二项式展开:
\[
y^{-3}=(x-(x-y))^{-3}
=x^{-3}\,\Big(1+\frac{3(x-y)}{x}+\frac{6(x-y)^{2}}{x^{2}}+\mathcal O((x-y)^{3}/x^{3})\Big).
\]
对 $y$ 积分得到
\begin{equation}\label{eq:A}
	\mathsf{(A)}
	=\frac{\delta}{x^{3}}+\frac{3}{2}\frac{\delta^{2}}{x^{4}}
	+\mathcal O\!\Big(\frac{\delta^{3}}{x^{5}}\Big).
\end{equation}
对 $\mathsf{(B)}$,
\[
y^{-7}=x^{-7}\,\Big(1+\frac{7(x-y)}{x}+\cdots\Big),
\]
故
\begin{equation}\label{eq:B}
	\mathsf{(B)}=\frac{3}{2}\,\frac{\delta}{x^{7}}
	+\mathcal O\!\Big(\frac{\delta^{2}}{x^{8}}\Big)
	=\mathcal O\!\Big(\frac{\delta}{x^{7}}\Big).
\end{equation}
将 \eqref{eq:A} 与 \eqref{eq:B} 代回 \eqref{eq:tau-asym-start},并注意到在通常的局部距离假设
\(
\delta\gg x^{-2}
\)
下(仍满足 $\delta\ll x$),有
\(
\delta/x^{7}=o(\delta^{3}/x^{5})
\),于是得到
\begin{equation}\label{eq:tau-final-asym}
	\tau(x\to L)=\frac{\delta}{x^{3}}
	+\frac{3}{2}\frac{\delta^{2}}{x^{4}}
	+\mathcal O\!\Big(\frac{\delta^{3}}{x^{5}}\Big)\ 
\end{equation}

与确定性时间 $t^e$ 的一致性到 $\mathcal O(\delta^{2})$. 由 \eqref{eq:te-local-expansion} 且此处 $\mu(s)=-s^{3}$ 可直接积分得
\begin{align}
	t^e&=\int_{x-\delta}^{x}\frac{ds}{s^{3}}
	=\frac{\delta}{x^{3}}+\frac{3}{2}\frac{\delta^{2}}{x^{4}}
	+\mathcal O\!\Big(\frac{\delta^{3}}{x^{5}}\Big).
	\label{eq:te-expansion-local}
\end{align}
对比 \eqref{eq:tau-final-asym} 与 \eqref{eq:te-expansion-local},两者在
\(
\mathcal O(1),\ \mathcal O(\delta)
\)
与
\(
\mathcal O(\delta^{2})
\)
各阶逐项一致,从而验证了
\[
\tau(x\to L)=t^e+\mathcal O\!\Big(\frac{\delta^{3}}{x^{5}}\Big),
\]
即与 \(t^e\) 完全一致到 \(\mathcal O(\delta^{2})\)。

%========12.1加入 ==========


%================== 11.30 新加入内容 =======

\subsection{时间离散格式的平均首达时间的渐近一致性与误差估计分析}

记 驯服Euler方法 的一步更新为
\[
X_{n+1}
=
X_n
+\frac{\mu(X_n)\,\Delta}{1+\Delta\,\lvert\mu(X_n)\rvert}
+\sigma(X_n)\,\Delta W_n,
\]
截断Euler-Maruyama方法的一步更新为
\[
X_{n+1}
=
X_n+\mu_{\Delta}(X_n)\,\Delta+\sigma_{\Delta}(X_n)\,\Delta W_n,
\quad
\mu_{\Delta}(x):=\mu\!\big(\pi_{\Delta}(x)\big),\ \ 
\sigma_{\Delta}(x):=\sigma\!\big(\pi_{\Delta}(x)\big),
\]
其中 \(\pi_{\Delta}\) 为以半径 \(h(\Delta)\) 的截断投影(常用约束 \(\Delta^{1/4}h(\Delta)\le 1\))。
在区间 \([x-\delta,x]\) 内,定义两种时间离散法跨越距离 \(\delta\) 的平均首达时间分别为
\(t^{\delta} \) 与 \(t^{\Delta}\)。

\begin{theorem}[驯服Euler方法的 MFPT 渐近与误差]\label{thm:tamed-mfpt}
	设 \(\mu\in C^{2}\) 且在 \(x\) 的邻域内单调并满足 \(\mu(x)<0\),\(\sigma\) 有界。
	在极限 \(x\to\infty,\ \delta/x\to0,\ \Delta\to0\) 下,有
	\begin{equation}\label{eq:tamed-expansion}
		t^{\delta}
		\;=\;
		\frac{\delta}{\lvert \mu(x)\rvert}
		\;+\;\delta\,\Delta
		\;+\;\mathcal O\!\Big(\Delta^{2}\delta\Big)
		\;+\;\mathcal O\!\Big(\delta^{2}\,\lvert \mu\rvert^{-2}\lvert \mu'\rvert\Big).
	\end{equation}
	进一步地,与漂移主导时间 \(t^{e}\) 的差满足
	\begin{equation}\label{eq:tamed-vs-te}
		t^{\delta} -t^{e}
		\;=\;
		\delta\,\Delta
		\;-\;\frac{\mu'(x)}{2\,\mu(x)^{2}}\,\delta^{2}
		\;+\;o\!\Big(\Delta\delta\Big)
		\;+\;o\!\Big(\delta^{2}\Big).
	\end{equation}
	因此,以 \(t^{e}\) 的主导项 \(\delta/\lvert\mu(x)\rvert\) 计,相对误差为
	\[
	\frac{t^{\delta}-t^{e}}{\delta/\lvert\mu(x)\rvert}
	\;=\;
	\Delta\,\lvert\mu(x)\rvert
	\;+\;\mathcal O\!\Big(\tfrac{\delta}{x}\Big),
	\]
	从而满足
	\(
	\Delta\lesssim \delta\,\lvert\mu(x)\rvert^{-1}
	\)
	可使 \(t^{\delta}\) 与 \(t^{e}\) 在 \(\mathcal O(\delta^{2})\) 阶上一致。
\end{theorem}

\begin{proof}
	忽略噪声对 MFPT 的次主导修正,单步期望位移
	\[
	\mathbb E[X_{n+1}-X_n\mid X_n=y]
	=
	\frac{\mu(y)\Delta}{(1+\Delta\lvert\mu(y)\rvert)}.
	\]
	于是局部有效速度
	\[
	v_{\mathrm{tame}}(y)=\frac{\lvert\mu(y)\rvert}{(1+\Delta\lvert\mu(y)\rvert)},
	\]
	跨越 \(\delta\) 的时间可写为
	\(
	t^{\delta}=\int_{x-\delta}^{x} v_{\mathrm{tame}}(y)^{-1}\,dy
	\).
	对 \(y\in[x-\delta,x]\) 以 \(x\) 为中心展开:
	\begin{align}
	v_{\mathrm{tame}}(y)^{-1} 
%	&=\frac{1+\Delta\lvert\mu(y)\rvert}{\lvert\mu(y)\rvert}\\
	&=
	\frac{1+\Delta\lvert\mu(x)\rvert}{\lvert\mu(x)\rvert}
	\;-\;\frac{\mu'(x)}{\mu(x)^{2}}\,(x-y)
	\;+\;\mathcal O\!\big((x-y)^{2}\big)
	\;+\;\mathcal O(\Delta\,(x-y)).
	\end{align}
	对 \(y\) 积分并用
	\(\int_{x-\delta}^{x}dy=\delta\),\(\int_{x-\delta}^{x}(x-y)\,dy=\delta^{2}/2\),即得
	\eqref{eq:tamed-expansion}。
	另一方面,\eqref{eq:te-def} 中
	\(
	t^{e}=\int_{x-\delta}^{x}\lvert\mu(y)\rvert^{-1}dy
	=
	\delta/\lvert\mu(x)\rvert-(\mu'(x)/2\mu(x)^{2})\,\delta^{2}+\mathcal O(\delta^{3})
	\),
	两式相减即得 \eqref{eq:tamed-vs-te}。
\end{proof}

\begin{theorem}[截断Euler-Maruyama方法的 MFPT 渐近与误差]\label{thm:trunc-mfpt}
	沿用上面的设定,设截断半径 \(h(\Delta)\) 使得在 \([x-\delta,x]\) 内截断生效(即 \(x\gg h(\Delta)\))。
	则存在常数 \(c>0\) 使得
	\begin{equation}\label{eq:trunc-expansion}
		t^{\Delta}
		\;=\;
		\frac{\delta}{\lvert \mu\!\big(h(\Delta)\big)\rvert}
		\;+\;\mathcal O\!\Big(\delta^{2}\Big),
	\end{equation}
	并与 \(t^{e}\) 比较有
	\begin{equation}\label{eq:trunc-vs-te}
		t^{\Delta}-t^{e}
		\;=\;
		\delta\!\left(
		\frac{1}{\lvert \mu\!\big(h(\Delta)\big)\rvert}
		-\frac{1}{\lvert \mu(x)\rvert}
		\right)
		\;+\;\mathcal O\!\Big(\delta^{2}\Big).
	\end{equation}
	若 \(\mu(y)=-y^{3}\) 且采用常用的约束 \(\Delta^{1/4}h(\Delta)\le1\) 并取饱和值
	\(h(\Delta)\asymp \Delta^{-1/4}\),则
	\[
	t^{\Delta}\asymp \delta\,\Delta^{1/4},
	\qquad
	\frac{t^{\Delta}}{t^{e}}
	\asymp
	\Big(\frac{x}{h(\Delta)}\Big)^{3}
	\asymp x^{3}\,\Delta^{1/4}.
	\]
	因此要使相对误差不超过 \(\varepsilon\in(0,1)\),充分条件是
	\begin{equation}\label{eq:trunc-stepsize}
		\Delta\ \lesssim\ \varepsilon^{4}\,x^{-12}.
	\end{equation}
\end{theorem}

\begin{proof}
	当 \(x\gg h(\Delta)\) 时,截断区间内 \(\mu_{\Delta}(y)=\mu(h(\Delta))\) 近似为常数,
	相应有效速度 \(v_{\mathrm{trunc}}(y)=\lvert\mu(h(\Delta))\rvert\)。
	故 \(\int_{x-\delta}^{x} v_{\mathrm{trunc}}(y)^{-1}dy=\delta/\lvert \mu(h(\Delta))\rvert\),得 \eqref{eq:trunc-expansion}。
	与 \(t^{e}\) 相减即得 \eqref{eq:trunc-vs-te}。
	当 \(\mu(y)=-y^{3}\) 时
	\(\lvert\mu(h(\Delta))\rvert=h(\Delta)^{3}\),
	代入给出所述量级估计,并由 \(\Delta^{1/4}h(\Delta)\le1\) 推出 \eqref{eq:trunc-stepsize}。
\end{proof}

在一维立方振子模型中,即\(\mu(x)=-x^{3}\) 时,
\begin{align*}
%	t^{e}
%	&=\frac{\delta}{x^{3}}+\frac{3}{2}\frac{\delta^{2}}{x^{4}}
%	+\mathcal O\!\Big(\frac{\delta^{3}}{x^{5}}\Big),\\
	|t^u - t^e|
	&=\mathcal{O\!} \left(\frac{\delta}{x^2}\right),\\
	|t^{\delta}-t^{e}|
	&=\delta\,\Delta-\frac{3}{2}\frac{\delta^{2}}{x^{4}}
	+o\!\big(\Delta\delta\big)+o\!\big(\delta^{2}\big),\\
	|t^{\Delta}-t^{e}|
	&=\delta\!\left(\frac{1}{h(\Delta)^{3}}-\frac{1}{x^{3}}\right)
	+\mathcal O\!\big(\delta^{2}\big).
\end{align*}

通过以上分析,可以看到在顺漂移和大初值情形下空间离散方法在固定空间跨越距离的时间估计方面具有明显优势。

%\begin{remark}
%	(1) 上述估计是在“顺漂移+\ 小位移”下的\emph{系统性}偏差。噪声带来的随机修正对 MFPT 的影响在该极限下为更高阶(对 \(\mu=-x^{3}\) 可估为 \(\mathcal O(\sigma^{2}\delta/x^{9})\)),不影响主导项比较。\;
%	(2) 若截断Euler-Maruyama方法在区间 \([x-\delta,x]\) 内\emph{不}触发截断(即 \(h(\Delta)\gtrsim x\)),则 \(t^{\Delta}\) 与 \(t^{e}\) 具有与 Euler-Maruyama 同阶的局部一致性,此时 \eqref{eq:trunc-stepsize} 可放宽。
%\end{remark}





\subsection{一维立方振子上的时间比较的数值实验}\label{subsec:cubic-numerics}

本小节在具体的一维立方振子模型上,对前文推导的理论时间进行数值检验,系统比较基于生成元离散的空间方法以及基于时间步进的截断 Euler-Maruyama 与驯服 Euler方法
在从大初值 \(x\) 向左跨越固定空间距离 \(\delta\) 所需时间这一指标上的表现差异。

考虑一维立方振子模型
\begin{equation}\label{eq:cubic-SDE-num}
	dX_t = -X_t^3\,dt + \sqrt{2}\,dW_t,\qquad X_0=x,
\end{equation}
其中漂移项 \(-X_t^3\) 在 \(|X_t| \gg 1 \) 处具有强耗散性,随机扰动由单位强度布朗运动提供。对于给定的大初值 \(x\gg 1\),我们考察从 \(x\) 向左跨越一个固定距离
\[
L = x - \delta,\qquad 0<\delta\ll x
\]
所需的平均时间。根据前文的漂移主导分析在数值实验中,\(t^e\) 被视为精确时间基准,用于比较不同数值方法的误差。为比较空间离散与时间离散两类方法的性能,采用如下统一设定:\\
初始位置取
	\[
	x\in\{5,\,8,\,12,\,16,\,20\},
	\]
	终止阈值为
	\[
	L=x-\delta,\qquad \delta=0.1,
	\]
即在大初值附近考察向左跨越固定距离 \(\delta\) 的时间。对于每个\(x\),分别用空间离散(SSA)与时间离散(截断 Euler-Maruyama、tamed EM)独立模拟 \(N\) 条轨道,估计对应的平均首达时间(MFPT),并与理论值 \(t^e\)做比较。


\paragraph{空间离散格式}

在一维情况下,取均匀网格
\[
x_i = x_0 + i h,\qquad h>0,
\]
实验中默认 \(h=10^{-3}\),并通过选择 \(\delta/h\in\mathbb{N}\) 确保阈值 \(L=x-\delta\) 落在网格节点上。记 \(\mu(x)=-x^3\),\(\sigma^2=2\),对应扩散系数 \(M=\sigma^2/2=1\)。\(Q_{u}\) 格式在节点 \(x_i\) 处的辅助量取为
\begin{equation}\label{eq:M-def-num}
	M(x_i)
	= \max\!\Big( M - \tfrac12|\mu(x_i)|\,h,\; 0\Big)
	= \max\!\Big( 1 - \tfrac12|x_i|^{3}h,\; 0\Big),
\end{equation}
左右跳跃率为
\begin{equation}\label{eq:qpm-num}
	q_i^- = \frac{(-\mu(x_i))_+}{h} + \frac{M(x_i)}{h^2},\qquad
	q_i^+ = \frac{(\mu(x_i))_+}{h} + \frac{M(x_i)}{h^2},
\end{equation}
其中 \((a)_+ := \max(a,0)\)。总跳率 \(\lambda_i = q_i^-+q_i^+\),在 \(x_i\) 处的等待时间服从指数分布
\[
\tau_i\sim\mathrm{Exp}(\lambda_i),
\]
并以概率 \(q_i^-/\lambda_i\) 向左跳至 \(x_{i-1}\),以概率 \(q_i^+/\lambda_i\) 向右跳至 \(x_{i+1}\)。从初始点 \(x\) 出发,重复上述步骤,直到首次到达阈值 \(L=x-\delta\),记录单条轨道的总飞跃时间
\[
T^{(k)}_{\mathrm{SSA}}(x\to L)
= \sum_{m=1}^{N_k}\tau_{i_m},\qquad k=1,\dots,N.
\]
对同一初值 \(x\) 的 \(N\) 条样本取平均,
\[
t^u(x\to L)
:= \frac{1}{N}\sum_{k=1}^{N}
T^{(k)}_{\mathrm{SSA}}(x\to L),
\]
作为 \(Q_{u}\)–SSA 的数值平均驻留时间。实验中默认 \(N_{\mathrm{SSA}}=1500\)。

\paragraph{时间离散格式}

时间离散方法采用固定时间步长 \(\Delta t>0\)。在本节实验中默认 \(\Delta t=10^{-5}\)。为保证对强漂移区域的稳定性与适用性,我们分别使用驯服 Euler和截断 Euler–Maruyama两种格式。

驯服 Euler对应迭代为
\begin{equation}\label{eq:tamed-EM-num}
	X_{n+1}
	= X_n
	+ \frac{\mu(X_n)\,\Delta t}{1+|\mu(X_n)|\Delta t}
	+ \sqrt{2\Delta t}\,\xi_n,\qquad
	\xi_n\sim\mathcal{N}(0,1)\ \text{i.i.d.}
\end{equation}
%当 \(|\mu|\Delta t\ll 1\) 时,该格式在本地与标准 EM 一致;而当 \(|\mu|\Delta t\gtrsim 1\) 时,漂移增量被削顶,从而避免数值解在强漂移区间中发生爆跳。

截断 Euler–Maruyama首先对状态进行截断,再代入漂移项。设截断半径
\[
r_\Delta = C\,\Delta t^{-\alpha},\qquad C>0,\ \alpha>0,
\]
定义截断算子
\[
x_{\mathrm{clip}} = \mathrm{clip}(X_n,-r_\Delta,r_\Delta),
\]
并令 \(\mu_{\mathrm{trunc}} = \mu(x_{\mathrm{clip}})\)。更新公式为
\begin{equation}\label{eq:trunc-EM-num}
	X_{n+1}
	= X_n
	+ \mu_{\mathrm{trunc}}\,\Delta t
	+ \sqrt{2\Delta t}\,\xi_n.
\end{equation}
实验中默认 \(C=5\)、\(\alpha=1/6\),使得在所选 \(x\) 与 \(\Delta t\) 范围内,截断主要在极端大振幅区域生效。%对于本节关注的 \(x\in[5,20]\) 的轨道,绝大部分步长实际等同于标准 EM。

在这两种时间离散方法下,从 \(X_0=x\) 出发迭代 \eqref{eq:tamed-EM-num} 或 \eqref{eq:trunc-EM-num},直到首次满足 \(X_n\le L=x-\delta\);再通过线性插值修正跨越步的首达时刻,得到单条轨道的首达时间
\[
T^{(k)}_{\mathrm{tame}}(x\to L),\qquad
T^{(k)}_{\mathrm{trunc}}(x\to L).
\]
对 \(N\) 条独立样本取平均,定义
\[
t^{\Delta}_{\mathrm{tame}}(x\to L)
:= \frac{1}{N}\sum_{k=1}^{N}T^{(k)}_{\mathrm{tame}}(x\to L),\qquad
t^{\Delta}_{\mathrm{trunc}}(x\to L)
:= \frac{1}{N}\sum_{k=1}^{N}T^{(k)}_{\mathrm{trunc}}(x\to L).
\]
实验中对时间离散同样取 \(N_{\mathrm{time}}=1500\) 条样本。

\subsubsection{误差度量与数值结果}

为比较不同方法在固定空间跨越距离下的时间精度,本文采用以下误差指标。对每个初值 \(x\),定义相对于理论基准 \(t^e(x\to L)\) 的绝对误差
\begin{align}
	E_u(x) 
	&:= \big|t^u(x\to L)-t^e(x\to L)\big|,\\
	E_{\mathrm{tame}}(x)
	&:= \big|t^{\Delta}_{\mathrm{tame}}(x\to L)-t^e(x\to L)\big|,\\
	E_{\mathrm{trunc}}(x)
	&:= \big|t^{\Delta}_{\mathrm{trunc}}(x\to L)-t^e(x\to L)\big|.
\end{align}
此外,为了定量比较时间离散与空间离散的精度差异,考虑误差比值
\begin{equation}
	R_{\mathrm{tame}}(x)
	:= \frac{E_{\mathrm{tame}}(x)}{E_u(x)},\qquad
	R_{\mathrm{trunc}}(x)
	:= \frac{E_{\mathrm{trunc}}(x)}{E_u(x)}.
\end{equation}

图~\ref{fig:time-space-mfpt} 给出了不同初值 \(x\) 下三种方法的误差统计结果。子图~(a) 展示了 \(E_u(x)\)、\(E_{\mathrm{tame}}(x)\)、\(E_{\mathrm{trunc}}(x)\) 随 \(x\) 变化的绝对误差;子图~(b) 则给出了误差比值 \(R_{\mathrm{tame}}(x)\)、\(R_{\mathrm{trunc}}(x)\) 随 \(x\) 的变化情况。

\begin{figure}[!htbp]
	\centering
	\subfloat[不同方法相对于 \(t^e(x\to L)\) 的绝对误差]{%
		\includegraphics[width=0.45\textwidth]{fig/Absolute_error.png}
	}\quad
	\subfloat[时间离散误差与空间离散误差比值]{%
		\includegraphics[width=0.45\textwidth]{fig/Error_ratio.png}
	}
	\caption{一维立方振子模型中,从 \(x\) 向左跨越固定距离 \(\delta\) 的平均时间误差比较:空间离散与时间离散的数值结果。}
	\label{fig:time-space-mfpt}
\end{figure}

从图~\ref{fig:time-space-mfpt}(a) 可以看到,在本节选取的参数范围内,随着初值 \(x\) 增大,三种方法得到的平均首达时间均快速趋近理论基准 \(t^e(x\to L)\),这与前文的分析相符。其中,基于 \(Q_{u}\) 的 SSA 方法在大 \(x\) 区域的误差明显更小,数值上呈现出随 \(x\) 增大而迅速衰减的趋势,与 \(t^e\) 的主阶行为 \(\delta/x^3\) 一致,体现了空间离散在强漂移极限下对局部时间的良好刻度匹配。图~\ref{fig:time-space-mfpt}(b) 的误差比值进一步揭示:在较大的 \(x\) 下,时间离散方法的误差总体上显著大于空间离散误差,即
\[
R_{\mathrm{tame}}(x),\ R_{\mathrm{trunc}}(x) \gg 1,
\]
表明在同样的空间跨越距离和相近的计算代价下,基于生成元离散的 SSA 方法在该模型上能提供更稳定、更精确的时间估计。随着 \(x\) 增大,空间离散误差已降到非常小的量级,而时间离散误差逐渐受限于时间步长与蒙特卡罗方差,难以进一步降低,这也反映了在强漂移、长时间问题中,单纯减小时间步长未必比提高空间离散精度更划算。

综上,一维立方振子数值实验验证了前文关于在顺漂移和大初值情形下空间离散方法在固定空间跨越距离的时间估计方面具有明显优势的理论结论。

	\section{随机 Canard 快-慢系统的动力学行为分析}
	
	\subsection{逃逸概率}
	
	对于随机微分方程~\eqref{eq:SDE},给定两个互不相交的 Borel 集 $A,B\subset D$,通常被理解为两个稳态或代谢态,例如反应物与产物,定义首次到达时间
	\[
	\tau_A := \inf\{t\ge0 : X_t\in A\},\qquad 
	\tau_B := \inf\{t\ge0 : X_t\in B\}.
	\]
	在此基础上,逃逸概率函数定义为
	\[
	q(x) := \mathbb{P}_x\big(\tau_B < \tau_A\big),\qquad x\in D.
	\]
	也就是说,$q(x)$ 描述了从 $x$ 出发的轨道先到达 $B$ 而不是 $A$的概率。经典结果表明,$q$ 是生成元 $L$ 的调和函数,满足如下椭圆型边值问题\cite{EV_TPT,EV_TowardsTP}:
	\begin{equation}\label{eq:committor_PDE}
		\begin{cases}
			L q(x) = 0, & x\in D\setminus(A\cup B),\\[1mm]
			q(x) = 0, & x\in A,\\[1mm]
			q(x) = 1, & x\in B.
		\end{cases}
	\end{equation}
	因此,逃逸概率 可以刻画从 $A$ 到 $B$ 的反应轨道的空间分布、流量以及转移速率等统计性质\cite{EV_TPT,EV_TowardsTP,Metzner_TPT_MJP}。
	
	Canard现象通常出现在二维的慢快平面系统中,当系统参数接近临界值时,解轨道会在吸引子与抑制子(稳定与不稳定慢流形)共存的区域滑行一段异常长的时间,然后突然跃迁。加入随机扰动后,Canard轨道仍可能出现,但噪声会使得轨道对初始条件和参数变化更加敏感,增加了数值模拟的难度。考虑二维快—慢随机系统:
	\begin{equation}\label{eq:canard-SDE}
		\begin{cases}
			\mathrm{d}x_t &= \Big(y_t - \big(\tfrac{x_t^3}{3} - x_t\big)\Big)\,\mathrm{d}t + \sigma_x\,\mathrm{d}W_t^{(x)},\\
			\mathrm{d}y_t &= \varepsilon\,(a - x_t)\,\mathrm{d}t + \sigma_y\,\mathrm{d}W_t^{(y)},
		\end{cases}
	\end{equation}
	其中$0<\varepsilon \ll 1$为表征慢速变量演化的小参数,$x_t$为快变量,$y_t$为慢变量,$a$是系统参数(当$a$取特定值时出现Canard过渡)。
	
	\subsection{CTRW生成元下的离散逃逸概率方程}
	
	%\subsubsection{CTRW生成元离散与马尔可夫跳跃过程}
	%
	%在区域 \(D\subset\mathbb{R}^2\) 上取一族空间网格 \(\{z_i\}_{i\in\mathcal{I}}\subset D\),网格尺寸记为 \(h>0\)。对于每个网格点 \(z_i\),选定有限个跃迁方向 \(z_i\to z_j\) 及对应的跃迁速率 \(q_{ij}\ge 0\),构造算子
	%\begin{equation}\label{eq:Qh-def}
	%	(Q_h\varphi)(z_i)
	%	:= \sum_{j\neq i} q_{ij}\big(\varphi(z_j)-\varphi(z_i)\big),
	%\end{equation}
	%其中 \(Q_h=(q_{ij})_{i,j\in\mathcal{I}}\) 是一个保守的 \(Q\)-矩阵(对每个 \(i\) 有 \(\sum_{j}q_{ij}=0\)、\(q_{ij}\ge0\) 当 \(j\neq i\))。
	%
	%Bou-Rabee–Vanden-Eijnden提出的 \(Q_u,Q_c\) 格式以及Zu提出的改进 \(Q_u\) 格式,都是通过适当构造 \(q_{ij}\) 使 \(Q_h\) 在生成元意义下逼近 \(\mathcal{L}\)。例如在扩散矩阵对角、系数足够光滑的场合,Zu证明了 \(\widetilde{Q}_u\) 在生成元意义具有二阶空间精度,即对足够光滑的 \(\varphi\) 有局部截断误差 \(O(h^2)\)。
	%
	%以 \eqref{eq:Qh-def} 为生成元,可以构造一个连续时间马尔可夫跳跃过程 \(\{Z_t^h\}_{t\ge0}\):当过程处于状态 \(z_i\) 时,以总速率
	%\[
	%\lambda_i := \sum_{j\neq i} q_{ij}
	%\]
	%等待指数分布的停留时间,然后以概率 \(q_{ij}/\lambda_i\) 跳至状态 \(z_j\)。这就是基于生成元离散的连续时间随机游走(CTRW)格式。
	
	%\subsubsection{离散committor与线性方程组}
	
	令 \(A_h,B_h\subset\{z_i\}\) 分别为集合 \(A,B\) 在网格上的近似(例如取其内近邻或外近邻)。定义空间离散下的逃逸概率为
	\[
	q_h(z_i) := \mathbb{P}_{z_i}\big(\tau_B^h < \tau_A^h\big),
	\]
	其中 \(\tau_A^h,\tau_B^h\) 分别为CTRW过程 \(Z_t^h\) 到达 \(A_h,B_h\) 的首次到达时间。
	
	并且根据\cite{bou2018continuous}对任意内部网格点 \(z_i\in D_h^\circ\setminus(A_h\cup B_h)\),离散逃逸概率满足
		\begin{equation}\label{eq:disc-harmonic}
			(Q_h q_h)(z_i) = 0,
		\end{equation}
		并满足边界条件
		\begin{equation}\label{eq:disc-bc}
			q_h(z_i) = 0,\quad z_i\in A_h;\qquad
			q_h(z_i) = 1,\quad z_i\in B_h.
		\end{equation}
	
	为了研究 \(q_h\) 与连续解 \(q\) 的误差,需要对生成元离散的逼近性质和连续问题的正则性作出假设。
	
	\begin{assumption}\label{ass:L-Qh}
		假设如下条件成立:
		\begin{enumerate}
			\item 系统\eqref{eq:canard-SDE}的漂移与扩散系数在 \(D\) 内 \(C^{4}\) 光滑,且扩散矩阵在 \(D\setminus(A\cup B)\) 上一致椭圆,从而\eqref{eq:committor_PDE}存在唯一解 \(q\in C^{2+\alpha}(D\setminus(A\cup B))\cap C(\overline{D})\)。
			
			\item 空间离散生成元 \(Q_h\) 对 \(\mathcal{L}\) 一致逼近。具体地,对任意 \(\varphi\in C_c^{4}(D)\),存在常数 \(C>0\) 和 \(p\ge1\) 使得
			\begin{equation}\label{eq:Qh-consistency}
				\big\| Q_h\varphi - \mathcal{L}\varphi \big\|_{L^\infty(D_h^\circ)}
				\le C h^p.
			\end{equation}
			对于Zu\cite{zu2023random}提出的改进 \(\widetilde{Q}_u\) 格式,可取 \(p=2\)。
			
			\item \(Q_h\) 满足离散最大值原理:若 \(u:\{z_i\}\to\mathbb{R}\) 在内部点上满足 \((Q_h u)(z_i)\ge0\),且在边界上 \(u(z_i)\le 0\),则对所有网格点有 \(u(z_i)\le0\)。
		\end{enumerate}
	\end{assumption}
	
	在以上假设下,可以给出离散逃逸概率的误差界如下:
	
	\begin{theorem}[空间离散下逃逸概率的收敛,见\cite{bou2018continuous}]\label{thm:Qh-committor-convergence}
		在假设\ref{ass:L-Qh}下,存在常数 \(C>0\),当 \(h>0\) 足够小时,离散解 \(q_h\) 与连续解 \(q\) 满足
		\begin{equation}\label{eq:Qh-error}
			\max_{z_i\in D_h} \big|q_h(z_i)-q(z_i)\big|
			\le C h^p,
		\end{equation}
		其中 \(p\) 由\eqref{eq:Qh-consistency}中的逼近阶决定。对改进的 \(\widetilde{Q}_u\) 格式有 \(p=2\),因此空间离散下逃逸概率函数达到二阶精度。
	\end{theorem}
	
	% 对随机Canard系统而言,漂移与扩散系数在有限区域 \(D\) 内通常是光滑的,扩散在感兴趣的慢流形附近区域非退化,且采用 \(\widetilde{Q}_u\) 之类构造时可以验证假设\ref{ass:L-Qh}。因此,在Canard模型上使用CTRW格式求解committor函数,其空间误差可以从理论上控制在 \(O(h^2)\) 量级。
	
	\subsection{驯服/截断Euler-Maruyama格式下的逃逸概率收敛性}
	
	\subsubsection{驯服/截断 Euler-Maruyama格式与强收敛}
	
	将\eqref{eq:canard-SDE}写成向量形式
	\begin{equation}\label{eq:vector-SDE}
		dZ_t = b(Z_t)\,dt + \Sigma(Z_t)\,dW_t,\qquad Z_0=z\in D,
	\end{equation}
	其中 \(b=(f,\varepsilon g)\),\(\Sigma\) 将扩散系数整合成 \(2\times 2\) 矩阵。对于典型Canard模型,\(b\) 具有至多三次多项式增长,不满足全局Lipschitz及线性增长条件,经典Euler–Maruyama方法在此类模型上可能发散。
	
	驯服Euler与截断Euler–Maruyama方法通过对漂移或状态进行适度削弱或截断,在Khasminskii型条件下恢复全局矩有界性与强收敛性。下述表述为二者的统一结论。
	
	\begin{assumption}\label{ass:Khasminskii}
		存在常数 \(p>2\) 与 \(K>0\),对所有 \(z\in\mathbb{R}^2\) 有
		\begin{equation}\label{eq:Khasminskii}
			\langle z,b(z)\rangle + \frac{p-1}{2}\|\Sigma(z)\|^2 \le K\bigl(1+\|z\|^2\bigr),
		\end{equation}
		且 \(b,\Sigma\) 局部Lipschitz。
	\end{assumption}
	
	在假设\ref{ass:Khasminskii}下,可以证明:对任意 \(T>0\) 和 \(q\in[2,p)\),存在常数 \(C>0\) 与 \(\gamma>0\),使得驯服Euler方法或截断Euler-Maruyama方法近似 \(Z_t^\Delta\) 满足
	\begin{equation}\label{eq:strong-convergence}
		\mathbb{E}\Big(\sup_{0\le t\le T}\|Z_t - Z_t^\Delta\|^q\Big)
		\le C\,\Delta^{q\gamma},
	\end{equation}
	其中一般有 \(\gamma=\tfrac12\)。关于证明文献\cite{mao2015truncated}中已有详细论述,此处不赘述。对于随机Canard系统,多项式增长的漂移\(b\)和有界扩散\(\Sigma\)满足上述条件。
	
	\subsubsection{时间离散下的逃逸概率分析}
	
	令时间步长 \(\Delta>0\),离散时刻 \(t_k:=k\Delta\),截断Euler-Maruyama方法给出离散链
	\[
	Z_{k+1}^\Delta = Z_k^\Delta + \Phi_\Delta(Z_k^\Delta,\xi_{k+1}),
	\]
	其中 \(\xi_{k+1}\sim N(0,I)\) 独立同分布,\(\Phi_\Delta\) 由具体格式生成。可通过分段插值构造连续时间过程 \(\widetilde{Z}_t^\Delta\),满足 \(\widetilde{Z}_{t_k}^\Delta = Z_k^\Delta\)。
	
	定义离散跳跃时间
	\begin{equation}\label{eq:disc-tau}
		\tau_A^\Delta := \inf\{k\ge0:\,Z_k^\Delta\in A\}\,\Delta,\qquad
		\tau_B^\Delta := \inf\{k\ge0:\,Z_k^\Delta\in B\}\,\Delta,
	\end{equation}
	并定义时间离散下的逃逸概率函数
	\begin{equation}\label{eq:qDelta-def}
		q_\Delta(z) := \mathbb{P}_z\big(\tau_B^\Delta < \tau_A^\Delta\big).
	\end{equation}
	
	记离散转移算子
	\[
	(P_\Delta\varphi)(z) := \mathbb{E}\big[\varphi(Z_1^\Delta)\mid Z_0^\Delta=z\big],
	\]
	以及对应的伪生成元
	\begin{equation}\label{eq:LDelta-def}
		\mathcal{L}_\Delta\varphi(z) := \frac{(P_\Delta\varphi)(z)-\varphi(z)}{\Delta}.
	\end{equation}
	则 \(q_\Delta\) 满足离散Dirichlet问题
	\begin{equation}\label{eq:qDelta-PDE}
		\begin{cases}
			\mathcal{L}_\Delta q_\Delta(z) = 0, & z\in D\setminus(A\cup B),\\
			q_\Delta(z) = 0, & z\in A,\\
			q_\Delta(z) = 1, & z\in B.
		\end{cases}
	\end{equation}
	可以将\eqref{eq:qDelta-PDE}看作对\eqref{eq:committor_PDE}的生成元层面的时间离散逼近。
	
	下面给出 \(q_\Delta\to q\) 的收敛性分析。首先对任意 \(T>0\),定义截断事件指标
	\[
	H_T := \mathbf{1}_{\{\tau_B<\tau_A,\ \tau_B\le T\}},\qquad
	H_T^\Delta := \mathbf{1}_{\{\tau_B^\Delta<\tau_A^\Delta,\ \tau_B^\Delta\le T\}}.
	\]
	于是
	\[
	q(z) = \lim_{T\to\infty} \mathbb{E}_z H_T,\qquad
	q_\Delta(z) = \lim_{T\to\infty} \mathbb{E}_z H_T^\Delta.
	\]
	
	由强收敛估计\eqref{eq:strong-convergence},对任意 \(\eta>0\) 有
	\begin{equation}\label{eq:strong-path}
		\mathbb{P}_z\Big(\sup_{0\le t\le T}\|Z_t - \widetilde{Z}_t^\Delta\| \ge \eta\Big)
		\le C\,\eta^{-q}\Delta^{q\gamma}.
	\end{equation}
	在Canard系统中,集合 \(A,B\) 之间距离严格正且边界光滑,可以选取 \(0<\eta<\frac12\mathrm{dist}(A,B)\),使得若两条路径在\([0,T]\)上保持距离小于 \(\eta\),则它们在先到达 \(A\) 还是先到达 \(B\)这一事件上是一致的。从而对任意 \(\varepsilon>0\),可先选取\(\eta>0\) 足够小,使得擦边事件的概率不超过 \(\varepsilon\),再利用\eqref{eq:strong-path} 得到
	\[
	\big|\mathbb{E}_z H_T - \mathbb{E}_z H_T^\Delta\big|
	\le \mathbb{P}_z\Big(\sup_{0\le t\le T}\|Z_t - \widetilde{Z}_t^\Delta\| \ge \eta\Big) + \varepsilon
	\le C\,\eta^{-q}\Delta^{q\gamma} + \varepsilon.
	\]
	在固定 \(T,\eta\) 下令 \(\Delta\to0\),便有
	\begin{equation}\label{eq:HT-conv}
		\lim_{\Delta\to0}\big|\mathbb{E}_z H_T - \mathbb{E}_z H_T^\Delta\big| = 0.
	\end{equation}
	%\paragraph{首次到达时间的尾部与极限交换}
	进一步假设SDE\eqref{eq:vector-SDE}在区域 \(D\) 内是适当意义下的正再生过程,\(\tau_A\wedge\tau_B\) 具有指数或多项式衰减的尾部:
	\begin{equation}\label{eq:tail}
		\mathbb{P}_z(\tau_A\wedge\tau_B > T) \le C_1 e^{-cT}
		\quad\text{或}\quad
		\mathbb{P}_z(\tau_A\wedge\tau_B > T) \le C_2 T^{-\beta},
	\end{equation}
	时间离散过程的首次到达时间 \(\tau_A^\Delta\wedge\tau_B^\Delta\) 也满足类似估计。当 \(T\) 足够大时,上述尾部概率可以任意小。
	
	因此,对任意 \(\varepsilon>0\),可以找到 \(T\) 足够大,使得
	\begin{equation}\label{eq:tail-H}
		\big|\mathbb{E}_z H_T - q(z)\big| \le \varepsilon,\qquad
		\big|\mathbb{E}_z H_T^\Delta - q_\Delta(z)\big| \le \varepsilon,
	\end{equation}
	即截断带来的误差不超过 \(\varepsilon\)。结合\eqref{eq:HT-conv},存在 \(\Delta_0>0\),当 \(0<\Delta<\Delta_0\) 时有
	\[
	\big|\mathbb{E}_z H_T - \mathbb{E}_z H_T^\Delta\big|\le\varepsilon.
	\]
	于是
	\[
	|q_\Delta(z)-q(z)|
	\le \big|\mathbb{E}_z H_T^\Delta-\mathbb{E}_z H_T\big|
	+ \big|\mathbb{E}_z H_T - q(z)\big|
	+ \big|\mathbb{E}_z H_T^\Delta-q_\Delta(z)\big|
	\le 3\varepsilon.
	\]
	由于 \(\varepsilon>0\) 任意,得到
	\begin{equation}\label{eq:qDelta-to-q}
		\lim_{\Delta\to0} q_\Delta(z) = q(z).
	\end{equation}
	
	若进一步利用\eqref{eq:strong-path}中的定量估计以及\eqref{eq:tail}的尾部衰减,可以得到一个次优的误差阶
	\[
	|q_\Delta(z)-q(z)| \le C\,\Delta^{\theta},\qquad \theta<\gamma=\tfrac12,
	\]
	表明时间离散下逃逸概率的收敛阶不会优于驯服Euler方法或者截断Euler-Maruyama方法的强收敛阶。%这与逃逸概率作为长时间弱指标的直觉是一致的。
	
	\subsection{随机Canard系统中的比较与讨论}
	
	对随机Canard系统而言,漂移 \(b\) 具有多项式增长,扩散 \(\Sigma\) 有界且光滑,满足Khasminskii型条件\eqref{eq:Khasminskii},因此驯服Euler方法或者截断Euler-Maruyama方法在有限时间区间内对真实轨道具有强收敛性质\eqref{eq:strong-convergence}。另一方面,在一个包含两个慢流形吸引区 \(A,B\) 的有界区域 \(D\) 内,生成元 \(\mathcal{L}\) 的系数光滑且扩散非退化,逃逸概率 PDE\eqref{eq:committor_PDE}存在唯一光滑解 \(q\)。
	
	在这样的背景下,空间离散在假设\ref{ass:L-Qh}下,\(q_h\) 以 \(O(h^p)\) 的速率一致收敛到 \(q\);时间离散通过时间步进构造路径近似 \(Z_t^\Delta\),再定义离散首次到达时间\eqref{eq:disc-tau}和逃逸概率 \(q_\Delta\)。利用强收敛\eqref{eq:strong-convergence}与首次到达时间尾部估计\eqref{eq:tail},命题\eqref{eq:qDelta-to-q}说明 \(q_\Delta\to q\),但误差阶被强收敛阶所限制,一般不优于 \(O(\Delta^{1/2})\)。要获得可见精度,需要较小时间步长和大量样本路径,因此在计算成本上相对昂贵。
	
	对Canard类快慢系统而言,逃逸概率函数度量的是从某一区域出发先落入哪一个慢流形吸引区的概率,这是高度敏感的长时间弱量。从理论收敛阶和结构保持的角度看,基于生成元离散的CTRW空间离散方法,在同等计算代价下更容易获得高精度的逃逸概率近似。
	

\subsection{数值实验}\label{sec:canard_cmp}

本节在随机 Canard 快-慢系统~\eqref{eq:canard-SDE} 上开展数值实验,旨在从逃逸概率比较基于生成元的空间离散方法与时间离散方法的数值表现与代价。通过一系列可重复的数值试验,验证前文关于两类格式收敛性质及适用场景的理论分析。

\subsubsection{实验设置与数值方法}

实验采用的随机 Canard 系统参数取为
\[
\varepsilon=10^{-2},\qquad a=1.0,\qquad (\sigma_x,\sigma_y)=(0,\,0.08),
\]
即仅在慢变量 \(y_t\) 上加入独立噪声,以尽量保持慢流形几何的清晰性。计算域取为
\[
\Omega=[-2.5,\,2.5]\times[-1,\,3],
\]
外边界施加反射条件。初值选在左侧慢流形邻域
\[
x_0=-1.5,\qquad y_0=\frac{x_0^3}{3}-x_0,
\]
用于观察典型的贴靠—折叠—快跳轨道。检测快跳事件的阈值固定为
\[
x_{\mathrm{th}}=1.5,
\]
即首次满足 \(x_t\ge x_{\mathrm{th}}\) 的时刻被视为离开左侧慢流形并完成一次快跳。

在数值方法方面,选取如下两大类:时间离散方法(简记为TD):采用 驯服Euler方法与 截断Euler-Maruyama方法两种稳定格式,在统一的时间步长 \(\Delta t\) 下对 SDE 进行路径模拟。通过镜像反射实现边界处理,从给定起点发射多条样本路径,根据首次满足 \(x_t\ge x_{\mathrm{th}}\) 或首次进入集合 \(A,B\) 的时间,统计快跳时间分布、逃逸概率 与 MFPT。空间离散方法(简记为SD):采用基于生成元的一致离散 CTRW/SSA 框架(记为 \(Q_u\) 格式)。在矩形网格上装配离散生成元 \(Q\),构造最近邻非负跳率,利用 SSA 模拟跳时与通道;对于 逃逸概率或MFPT等弱量,则直接求解离散生成元方程获得全场解。

对于弱量比较,以离散生成元 \(Q\) 上求解的 \(Q\)-解作为参考,在若干评价点 \(\{z_i\}_{i=1}^{N_p}\) 上,以均方根误差
\[
E_q=\Big(\tfrac1{N_p}\sum_{i=1}^{N_p}\lvert \hat q(z_i)-q_{\mathrm{ref}}(z_i)\rvert^2\Big)^{1/2}
\]
度量各方法求得的 逃逸概率。空间离散的代价(work)以自由未知元数 \(\times\) 迭代轮数的总局部更新次数为 proxy;时间离散的代价则以所有评价点、所有样本路径的总时间步数计数,从而在统一的代价指标下绘制 work--error 曲线。

\subsubsection{系统路径与快跳时间分布}

方程~\eqref{eq:canard-SDE} 的 \(x\)-零流形为 \(y=x^3/3-x\),在折叠点附近可分为吸引与反吸引两段慢流形。典型的 Canard 轨道表现为:轨道沿吸引慢流形缓慢贴靠,进入折叠附近后在反吸引分支短暂徘徊,随后发生快跳离开,该过程对噪声与数值扰动极为敏感。

图~\ref{fig:canard_geom_jump} 给出了在时间窗 \(T=40\) 内,Tamed Euler-Maruyama、truncated Euler-Maruyama 与 CTRW/SSA 的代表性相图轨道以及对应的首次越阈时间直方图。可以看到,三种方法均能较好再现贴靠—折叠—快跳的典型几何结构:时间离散方法得到的轨道在相图上呈平滑曲线,而 CTRW/SSA 由于采用事件驱动的跳跃过程,在图像上呈现出明显的阶梯特征。

\begin{figure}[!htbp]
	\centering
	\subfloat[典型样本轨道(背景为 $x$-零流形 $y=x^3/3-x$)]{%
		\label{fig:canard_traj}
		\includegraphics[width=0.45\textwidth]{fig/Sample_trajectories_100.png}
	}\quad
	\subfloat[首次满足 $x\ge x_{\mathrm{th}}$ 的时间直方图]{%
		\label{fig:canard_jump_hist}
		\includegraphics[width=0.45\textwidth]{fig/Fast_jump_time_distribution_100.png}
	}
	\caption{随机 Canard 系统的路径与快跳时间分布:三种数值方法在 \(T=40\) 内的对比。}
	\label{fig:canard_geom_jump}
\end{figure}

从图~\ref{fig:canard_jump_hist} 的直方图可以看出,三种方法得到的快跳时间均集中在区间 \(t\in(10,\,30)\) 以内,分布区间相互重叠,峰值位置略有差异。

图~\ref{fig:canard_occ} 展示了驯服Euler方法 与CTRW 在同一时间窗内的长期驻留直方图。可以看出,两者在整体形状上高度一致,概率质量主要集中在左侧慢流形附近;CTRW 的占据度在慢流形通道内相对更为集中,体现了事件驱动框架下等待时间显式抽样所带来的驻留拉长效应。
\begin{figure}[htbp]
	\centering
	\includegraphics[width=0.60\textwidth]{fig/Occupancy_100.png}
	\caption{占据度热图(长期驻留分布):左图为 驯服Euler方法,右图为 CTRW/SSA。可以看到,两者在慢流形通道内的概率质量分布总体一致。}
	\label{fig:canard_occ}
\end{figure}

综合上述结果,在有限时间、以样本路径和快跳时间为指标比较下,时间离散方法在给定计算代价下更容易获得平滑轨道和较小的路径级误差;CTRW/SSA 在几何结构上也能保持 Canard 的关键特征,但由于跳跃路径的阶梯性,其在路径级强误差指标下并非优势所在。

\subsubsection{逃逸概率 与 MFPT 的空间结构}

为考察 Canard 系统在弱/长期统计量上的差异,本文进一步计算 逃逸概率函数 \(q\) 与平均首达时间(MFPT)\(m\)。它们满足如下生成元型边值问题:
\begin{equation}\label{eq:committor-mfpt}
	\begin{cases}
		Lq = 0, & \text{in }\Omega\setminus(A\cup B),\\
		q|_A=0,\ q|_B=1,\ \partial_{\boldsymbol n}q|_{\partial\Omega}=0,
	\end{cases}
	\qquad
	\begin{cases}
		Lm = -1, & \text{in }\Omega\setminus B,\\
		m|_B=0,\ \partial_{\boldsymbol n}m|_{\partial\Omega}=0,
	\end{cases}
\end{equation}
其中 \(L\) 为 Itô 生成元,\(A,B\) 分别为左、右慢流形附近的吸引区。

空间离散方法中,以离散生成元 \(Q\) 近似 \(L\),在网格上直接求解
\[
Qq = 0,\ q|_A=0,\ q|_B=1;\qquad
Qm = -\mathbf 1,\ m|_B=0,
\]
一次稀疏线性代数求解即可得到全场的 \(q(x,y)\) 与 \(m(x,y)\)。时间离散方法中,则在若干评价点 \(z\) 上,通过 Monte Carlo 估计
\[
\hat q(z)=\mathbb P_z(\tau_B<\tau_A)\approx
\frac{1}{N}\sum_{k=1}^{N}\mathbf 1_{\{\tau_B^{(k)}<\tau_A^{(k)}\}},
\qquad
\hat m(z)\approx \frac{1}{N}\sum_{k=1}^{N}\tau_B^{(k)},
\]
并通过插值构造空间分布。

图~\ref{fig:committor_mfpt_heat} 给出了 逃逸概率 与 MFPT 的热图:左列为 CTRW/\(Q\)-解,右列为 truncated Euler-Maruyama Monte Carlo 结果。可以观察到,逃逸概率 \(q\) 在左侧慢流形区域接近 0,在右侧目标区域 \(B\) 附近接近 1,在折叠邻域形成一条清晰的概率边界层;MFPT \(m\) 在远离 \(B\) 的左支区域取值较大,在靠近 \(B\) 的前沿迅速衰减至 0,体现了几何结构对平均过渡时间的强烈影响。

\begin{figure}[htbp]
	\centering
	\begin{minipage}[b]{0.45\textwidth}
		\includegraphics[width=\textwidth]{fig/committor_heatmap.jpg}
		\caption*{(a) 逃逸概率, CTRW/\(Q\)-解}
	\end{minipage}\hfill
	\begin{minipage}[b]{0.45\textwidth}
		\includegraphics[width=\textwidth]{fig/committor_trunc_heatmap.png}
		\caption*{(b) committor, truncated Euler-Maruyama MC}
	\end{minipage}
	
	\vspace{0.8em}
	\begin{minipage}[b]{0.45\textwidth}
		\includegraphics[width=\textwidth]{fig/mfpt_heatmap.jpg}
		\caption*{(c) MFPT, CTRW/\(Q\)-解}
	\end{minipage}\hfill
	\begin{minipage}[b]{0.45\textwidth}
		\includegraphics[width=\textwidth]{fig/mfpt_trunc_heatmap.png}
		\caption*{(d) MFPT, truncated EM MC}
	\end{minipage}
	
	\caption{随机 Canard 系统中 逃逸概率 与 MFPT 的空间结构:左列为空间离散结果,右列为时间离散结果。}
	\label{fig:committor_mfpt_heat}
\end{figure}

为进一步定量刻画慢流形上的行为,可沿慢流形 \(y=x^3/3-x\) 取若干采样点。例如,在左支深处
\[
(x_0,y_0)=\bigl(-1.5,\ \tfrac{x_0^3}{3}-x_0\bigr),
\]
有数值结果 \(q(x_0,y_0)\approx 0.00,\ m(x_0,y_0)\approx 24.2\),表明从该区域出发几乎不可能直接先达右侧慢流形吸引区 \(B\),同时平均停留时间较长;在靠近折叠后的点
\[
(x_f,y_f)=\bigl(1.0,\ \tfrac{x_f^3}{3}-x_f\bigr),
\]
有 \(q(x_f,y_f)\approx 0.35,\ m(x_f,y_f)\approx 54.0\),说明一方面先达 \(B\) 的概率显著上升,另一方面仍可能在缓慢通道区域徘徊较长时间。这些数值与图~\ref{fig:committor_mfpt_heat} 中的空间分布总体一致,清晰体现了几何与随机性在 Canard 过渡中的耦合作用。

\subsubsection{基于 逃逸概率 的 \texorpdfstring{work--error}{work--error} 比较}
\label{sec:sec63-committor-work-error}

针对 逃逸概率 这一生成元型弱量,本文进一步比较空间离散与时间离散在误差—代价关系上的表现。仍以 \eqref{eq:committor-mfpt} 中的 \(q\) 为对象,选择慢流形 \(y=x^3/3-x\) 上的 \(N_p=10\) 个评价点 \(\{z_i\}\),以较细网格 \(h_{\text{ref}}=0.08\) 下的 \(Q\)-解 \(q_{\text{ref}}\) 作为参考,RMS 误差定义如前所述。实验流程如下:\\
第一步:对于空间离散(SD),在不同网格步长 \(h\in\{0.14,0.12,0.10\}\) 下装配 \(Q\),采用 Gauss–Seidel 迭代求解 \(Qq=0\),记录 sweeps 次数与误差 \(E_q\),并以自由未知元数 \(\times\) sweeps作为代价。\\
第二步:对于时间离散(TD),采用 truncated Euler-Maruyama,在统一步长 \(\Delta t=0.004\)、时间窗 \(T_{\max}\approx 18\) 下,从每个评价点发射 \(N\in\{12,24,48,72\}\) 条轨道,通过 Monte Carlo 估计 逃逸概率,并以所有节点与路径的总时间步数作为 work。
对应的 work--error 曲线见图~\ref{fig:work-error-committor}。可以看到,随着 work 的增加,两类方法的误差均呈下降趋势;但在同一数量级的代价下,空间离散的误差明显更小。

\begin{figure}[htbp]
	\centering
	\includegraphics[width=0.62\textwidth]{fig/work_error_committor_trunc_vs_ctrw.pdf}
	\caption{逃逸概率 的 work--error 曲线:空间离散与时间离散对比。纵轴为相对于参考 \(Q\)-解的 RMS 误差,横轴为代价(work proxy)。}
	\label{fig:work-error-committor}
\end{figure}

表~\ref{tab:sd-committor} 给出了 SD 在不同网格步长下的误差与代价;表~\ref{tab:td-committor} 给出了 TD 在不同样本数下的结果。可以看出,当网格由 \(h=0.14\) 缩小到 \(h=0.10\) 时,SD 的 RMS 误差从 \(0.25\) 左右迅速下降至 \(0.07\) 左右,而代价仍处于 \(10^4\) 量级;相比之下,TD 在样本数从 \(12\) 增加到 \(72\) 时,误差始终停留在 \(0.25\sim 0.28\) 区间,且 work 已经提升到 \(10^5\sim 10^6\) 量级。

\begin{table}[htbp]
	\centering
	\caption{SD(\(Q\)-解,GS 迭代)在不同网格下的 work--error}
	\label{tab:sd-committor}
	\begin{tabular}{cccc}
		\toprule
		网格步长 \(h\) & sweeps & work(proxy) & RMS 误差 \(E_q\) \\
		\midrule
		0.140 & 26 & 9{,}048  & 0.25118 \\
		0.120 & 37 & 17{,}612 & 0.26187 \\
		0.100 & 39 & 27{,}183 & 0.07024 \\
		\bottomrule
	\end{tabular}
\end{table}

\begin{table}[htbp]
	\centering
	\caption{TD(\(\Delta t=0.004\))在不同样本数下的 work--error}
	\label{tab:td-committor}
	\begin{tabular}{ccc}
		\toprule
		每节点样本数 \(N\) & work(总步数) & RMS 误差 \(E_q\) \\
		\midrule
		12 & 172{,}849 & 0.24704 \\
		24 & 309{,}434 & 0.27759 \\
		48 & 667{,}001 & 0.27981 \\
		72 & 924{,}692 & 0.25288 \\
		\bottomrule
	\end{tabular}
\end{table}

为便于直接比较,在表~\ref{tab:equal-work-committor} 中将 TD 的每个代价点与 SD 中 work 最接近的点配对。从表中可以清楚看到,在 work 约为 \(2.7\times 10^4\) 的代价水平上,空间离散已实现 \(E_q\approx 0.07\) 的误差;而时间离散即便将 work 提高到 \(10^5\) 乃至 \(10^6\) 量级,误差仍然停留在 \(0.25\sim 0.28\) 区间。

\begin{table}[htbp]
	\centering
	\caption{同等代价的 RMS 误差对比}
	\label{tab:equal-work-committor}
	\begin{tabular}{cccc}
		\toprule
		TD:work(步) & TD:\(E_q\) & SD(最近 work):work & SD:\(E_q\) \\
		\midrule
		172{,}849 & 0.24704 & 27{,}183 & 0.07024 \\
		309{,}434 & 0.27759 & 27{,}183 & 0.07024 \\
		667{,}001 & 0.27981 & 27{,}183 & 0.07024 \\
		924{,}692 & 0.25288 & 27{,}183 & 0.07024 \\
		\bottomrule
	\end{tabular}
\end{table}

这些结果表明,在 逃逸概率 与 MFPT 等生成元型弱量的比较中,CTRW/\(Q_u\) 空间离散方法一方面能够自然保持反射边界与几何结构,另一方面通过一次稀疏线性求解即可获得全场解,避免了 Monte Carlo 方差,在同等甚至远低于时间离散的代价下即可达到显著更小的误差,这一结论与前文的理论分析相互印证。

\newpage


%=====================================

\section{随机 Canard 快--慢系统中快跳行为的数值比较}
\label{sec:canard-fast-jump}

本节在二维快--慢随机 Canard 系统上考察数值轨道的几何形态,重点比较时间离散格式与基于生成元的空间离散格式在再现 Canard 快跳现象方面的表现。考虑如下缩放后的快--慢系统:
\begin{equation}\label{eq:canard-fast-slow}
	\begin{cases}
		\dfrac{\mathrm{d}x_{1,t}}{\mathrm{d}t}
		= \dfrac{1}{\delta}\Bigl(x_{1,t} + x_{2,t} - \dfrac{x_{1,t}^3}{3}\Bigr),\\[6pt]
		\dfrac{\mathrm{d}x_{2,t}}{\mathrm{d}t}
		= a - x_{1,t},
	\end{cases}
\end{equation}
其中 \(0<\delta\ll 1\) 为快--慢时间尺度比,\(x_1\) 是快变量,\(x_2\) 为慢变量。为了刻画噪声对 Canard 轨道的扰动,进一步在慢变量方向上加入小幅随机扰动,得到随机 Canard 系统
\begin{equation}\label{eq:canard-fast-slow-SDE}
	\begin{cases}
		\mathrm{d}x_{1,t}
		= \dfrac{1}{\delta}\Bigl(x_{1,t} + x_{2,t} - \dfrac{x_{1,t}^3}{3}\Bigr)\,\mathrm{d}t,\\[6pt]
		\mathrm{d}x_{2,t}
		= (a - x_{1,t})\,\mathrm{d}t + \sigma_y\,\mathrm{d}W_t^{(2)},
	\end{cases}
\end{equation}
其中 \(W^{(2)}_t\) 为一维标准布朗运动,噪声强度取 \(\sigma_y=0.08\)。本节数值实验中固定
\[
\delta = 0.1,\qquad
a = 1 - \frac{\delta}{8} - \frac{3\delta^2}{32}
- \frac{173\delta^3}{1024} - 0.01.
\]
上述 \(a\) 的取值位于经典 Canard 参数的附近,使得确定性系统 \eqref{eq:canard-fast-slow} 在折叠点附近存在典型的“贴靠慢流形---短暂沿抑制支漂移---发生快跳”的轨道结构,噪声则进一步增强了对初值与数值扰动的敏感性。

\subsection{快跳事件的定义与数值目标}

方程 \eqref{eq:canard-fast-slow-SDE} 的零流形由
\[
x_2 = \frac{x_1^3}{3} - x_1
\]
给出,在折叠点附近可分为左侧吸引慢流形与右侧抑制(反吸引)慢流形。对给定的初值,如果轨道先在左侧吸引慢流形附近缓慢演化,然后进入折叠邻域,在几乎沿抑制慢流形短暂停留之后突然跨越到右侧区域,则称该轨道发生了一次 Canard 快跳。

为了将快跳现象转化为可计算的统计量,本文采用如下简单的判别方式。给定一条轨道 \(\{x_t\}_{t\ge0}\),定义
\[
\tau_{\mathrm{jump}}
:= \inf\{t>0:\ x_{1,t}\ge x_{\mathrm{th}}\},
\]
其中快跳阈值固定为
\[
x_{\mathrm{th}} = 1.5.
\]
也就是说,\(\tau_{\mathrm{jump}}\) 表示快变量 \(x_1\) 第一次越过折叠点右侧阈值 \(x_{\mathrm{th}}\) 的时间。如果在有限考察时间窗 \([0,T]\) 内满足 \(\tau_{\mathrm{jump}}\le T\),则认为在该时间窗内发生了至少一次快跳。相应地,我们将
\[
p_{\mathrm{jump}} := \mathbb{P}(\tau_{\mathrm{jump}}\le T),\qquad
m_{\mathrm{jump}} := \mathbb{E}[\tau_{\mathrm{jump}}\mid\tau_{\mathrm{jump}}\le T]
\]
分别作为“快跳概率”和“条件平均快跳时间”的数值指标。本节实验固定观察时间窗 \(T=3\),以保证在所选参数下大部分轨道能够完成一次典型的快跳。

下文将比较在相同初值与计算代价下,不同数值格式对 \(p_{\mathrm{jump}}\) 与 \(m_{\mathrm{jump}}\) 的近似效果。特别是,我们关心在\emph{较大时间步长}下,时间离散格式是否仍然能够产生正确的快跳行为,而空间离散格式在这方面是否更稳定、更鲁棒。

\subsection{时间离散格式:Euler--Maruyama 与驯服 EM}

时间离散方法采用固定时间步长 \(\Delta t>0\),在时间网格 \(t_n = n\Delta t\) 上给出数值近似 \(X_n\approx x_{t_n}\)。对方程 \eqref{eq:canard-fast-slow-SDE},标准的 Euler--Maruyama 格式为
\begin{equation}\label{eq:canard-EM}
	\begin{aligned}
		X^{(1)}_{n+1}
		&= X^{(1)}_n
		+ \frac{\Delta t}{\delta}
		\Bigl(X^{(1)}_n + X^{(2)}_n - \frac{(X^{(1)}_n)^3}{3}\Bigr),\\[3pt]
		X^{(2)}_{n+1}
		&= X^{(2)}_n
		+ (a - X^{(1)}_n)\,\Delta t
		+ \sigma_y\,\sqrt{\Delta t}\,\xi_n,
	\end{aligned}
\end{equation}
其中 \(\{\xi_n\}\) 为独立同分布的标准正态随机变量。为了改善强漂移区间的稳定性,也可以采用驯服 Euler--Maruyama 格式,即在快变量漂移上加入驯服因子:
\begin{equation}\label{eq:canard-tamed}
	\begin{aligned}
		X^{(1)}_{n+1}
		&= X^{(1)}_n
		+ \frac{\Delta t}{\delta}
		\frac{X^{(1)}_n + X^{(2)}_n - (X^{(1)}_n)^3/3}
		{1 + \lvert X^{(1)}_n + X^{(2)}_n - (X^{(1)}_n)^3/3\rvert\Delta t},\\[3pt]
		X^{(2)}_{n+1}
		&= X^{(2)}_n
		+ (a - X^{(1)}_n)\,\Delta t
		+ \sigma_y\,\sqrt{\Delta t}\,\xi_n.
	\end{aligned}
\end{equation}
当 \(\Delta t\) 足够小时,\eqref{eq:canard-EM} 与 \eqref{eq:canard-tamed} 在局部一致性与强收敛阶上没有本质差异;但在强漂移区域 \(|x_1|\gg1\) 或 \(|x_1+x_2 - x_1^3/3|\gg1\) 时,驯服格式通过分母中的削顶保证数值解不会发生数值爆跳。

在快--慢系统中,漂移在快变量方向的尺度大致为 \(\mathcal O(1/\delta)\)。因此显式时间步进方法若要解析快时间尺度,必须满足
\begin{equation}\label{eq:canard-dt-fast}
	\frac{\Delta t}{\delta}\lvert x_1 + x_2 - x_1^3/3\rvert \ll 1,
\end{equation}
否则单步增量就可能跨越折叠邻域,或者在驯服因子的作用下将快跳过程严重“平滑化”。在数值实验中,我们将显式格式在“较小步长”(例如 \(\Delta t=10^{-3}\))和“较大步长”(例如 \(\Delta t=5\times10^{-2},\,10^{-1}\))下的结果进行对比,观察 \(\Delta t\) 对快跳概率与快跳时间分布的影响。

\subsection{空间离散格式:基于生成元的 CTRW 跳跃过程}

空间离散方法从生成元的角度出发,先在状态空间上构造离散生成元 \(Q_h\),再通过连续时间随机游走(CTRW)与随机模拟算法(SSA)生成数值轨道。对 SDE
\[
\mathrm{d}X_t = \mu(X_t)\,\mathrm{d}t + \sigma(X_t)\,\mathrm{d}W_t,
\]
其 Kolmogorov 向前算子为
\[
(Lf)(x) = \sum_{i=1}^2 \mu_i(x)\,\partial_{x_i}f(x)
+ \frac12\sum_{i=1}^2\sigma_i^2(x)\,\partial_{x_i}^2 f(x),
\]
本文采用 \cite{zu2023random} 提出的改进 \(Q_u\) 格式在二维正方网格上对 \(L\) 进行离散。设计算域
\[
\Omega = [-2.5,2.5]\times[-1,3],
\]
在其中取均匀网格
\[
x_i = x_{\min} + i h_x,\quad i=0,\dots,N_x;\qquad
y_j = y_{\min} + j h_y,\quad j=0,\dots,N_y,
\]
本文实验中 \(h_x=h_y=0.05\)。对任意内部网格点 \((x_i,y_j)\),记
\[
\mu_1 = \mu_1(x_i,y_j),\qquad
\mu_2 = \mu_2(x_i,y_j),\qquad
M_{11} = \frac{\sigma_x^2}{2},\qquad
M_{22} = \frac{\sigma_y^2}{2},
\]
则 \(Q_u\) 格式在四个相邻网格点上的跃迁率可写为
\begin{align*}
	q_{i,j\to i+1,j}
	&= \frac{\mu_1^+}{h_x} + \frac{M_{11}}{h_x^2},&
	q_{i,j\to i-1,j}
	&= \frac{\mu_1^-}{h_x} + \frac{M_{11}}{h_x^2},\\[2pt]
	q_{i,j\to i,j+1}
	&= \frac{\mu_2^+}{h_y} + \frac{M_{22}}{h_y^2},&
	q_{i,j\to i,j-1}
	&= \frac{\mu_2^-}{h_y} + \frac{M_{22}}{h_y^2},
\end{align*}
其中 \(\mu_k^+ = \max(\mu_k,0)\)、\(\mu_k^- = \max(-\mu_k,0)\)。总跃迁率为
\[
\lambda_{i,j} = \sum_{(i',j')\neq(i,j)} q_{i,j\to i',j'}.
\]
以 \(Q_u\) 为生成元,可以构造连续时间马尔可夫跳跃过程 \(\{Z_t^h\}_{t\ge0}\):若当前状态为 \((x_i,y_j)\),则停留时间服从参数为 \(\lambda_{i,j}\) 的指数分布
\[
\tau\sim\mathrm{Exp}(\lambda_{i,j}),
\]
随后以概率 \(q_{i,j\to i',j'}/\lambda_{i,j}\) 跳至相邻格点 \((x_{i'},y_{j'})\)。重复上述过程即可得到一条近似轨道 \(\{Z_t^h\}\)。

由 \(Q_u\) 的构造可见,当快变量漂移较大时,\(\lvert\mu_1\rvert\) 增大,导致 \(\lambda_{i,j}\) 随之变大,平均停留时间
\[
\mathbb{E}[\tau\mid Z_t^h=(x_i,y_j)]
= \lambda_{i,j}^{-1}
\approx \frac{h_x}{\lvert\mu_1(x_i,y_j)\rvert},
\]
从而在快跳区间自动形成\emph{更细的有效时间步长}。与显式时间步进方法的固定步长 \(\Delta t\) 相比,SSA 在强漂移区域会自适应地缩小停留时间,这为解析 Canard 快跳提供了天然优势。

\subsection{快跳事件对数值扰动的敏感性分析}

在经典 Canard 理论中,快--慢系统可重写为\footnote{这里略去详细的坐标变换与正规形推导,仅给出定性结论。}
\[
\varepsilon\frac{\mathrm{d}x_1}{\mathrm{d}t}
= F(x_1,x_2),\qquad
\frac{\mathrm{d}x_2}{\mathrm{d}t} = G(x_1,x_2),
\]
其中 \(\varepsilon\sim\delta\) 为小参数,\(F\) 在折叠点附近满足 \(F(x_{\mathrm{fold}},x_2)=0\)、\(\partial_{x_1}F(x_{\mathrm{fold}},x_2)\ne0\)。处在 Canard 过渡附近时,存在一类对参数 \(a\) 极其敏感的轨道,它们沿反吸引慢流形停留的时间对 \(a\) 的变化呈指数级敏感,典型尺度为
\[
\mathcal O\bigl(\exp(-C/\varepsilon)\bigr).
\]
这意味着任意 \(\mathcal O(\varepsilon)\) 级别的系统扰动(包括数值离散误差)都可能将真实系统从“有 Canard 快跳”的参数区间推到“无 Canard 或快跳显著提前 / 延迟”的区域。

显式时间步进方法本质上是在时间方向上对生成元 \(L\) 做一次向前 Euler 近似:
\[
u(t+\Delta t,\cdot)\approx u(t,\cdot) + \Delta t\,Lu(t,\cdot),
\]
相应地,\eqref{eq:canard-EM} 与 \eqref{eq:canard-tamed} 可以理解为在每一步对漂移场 \(\mu\) 作了 \(\mathcal O(\Delta t)\) 的扰动。对于 Canard 系统,这种在快变量方向上的有效漂移扰动约为
\[
\delta\mu_1 \approx \mathcal O\Bigl(\frac{\Delta t}{\delta}\Bigr),
\]
当 \(\Delta t\) 与 \(\delta\) 同阶或更大时,\(\delta\mu_1\) 完全可能达到 \(\mathcal O(\varepsilon)\) 的量级,从而改变系统是否处于 Canard 参数窗口。这在数值上表现为:

\begin{itemize}
	\item 当 \(\Delta t\) 很小时,时间离散方法能够正确再现典型的快跳轨道,快跳概率与平均快跳时间与参考解一致;
	\item 当 \(\Delta t\) 增大到 \(\Delta t\gtrsim c\,\delta\) 时,数值漂移的有效扰动足以“推走” Canard 轨道,使得快跳发生显著提前或者延迟,极端情况下甚至在观察时间窗内不再出现典型的快跳。
\end{itemize}

相比之下,基于生成元的 \(Q_u\) 空间离散在\emph{空间}上引入的是 \(\mathcal O(h^2)\) 级别的局部截断误差,并且在快变量方向的有效时间步长由
\[
\Delta t_{\mathrm{eff}}\sim \frac{h_x}{\lvert\mu_1\rvert}
\]
自适应决定。只要网格尺度 \(h_x\) 选取满足
\[
\frac{h_x}{\lvert\mu_1\rvert} \ll 1,
\]
生成元离散在折叠邻域不会引入足以破坏 Canard 结构的一阶漂移偏差,从而在较宽的参数与噪声范围内保持快跳行为的结构稳定性。

综上,在“是否出现快跳”这样的几何敏感指标上,时间离散格式对时间步长 \(\Delta t\) 极为敏感,而基于生成元的空间离散格式在同阶计算代价下具有更好的鲁棒性与结构保持性。

\subsection{数值实验与结果比较}

为了验证上述分析,本节在随机 Canard 系统 \eqref{eq:canard-fast-slow-SDE} 上进行数值实验。计算域取为
\[
\Omega=[-2.5,\,2.5]\times[-1,\,3],
\]
外边界采用反射条件。初值选在左侧慢流形上
\[
x_1(0) = -1.5,\qquad
x_2(0) = \frac{x_1(0)^3}{3}-x_1(0),
\]
快跳阈值为 \(x_{\mathrm{th}}=1.5\),观察时间窗 \(T=3\)。

\paragraph{空间离散(CTRW--SSA)}

空间离散方法采用前述 \(Q_u\) 格式,在 \(h_x=h_y=0.05\) 的均匀网格上构造生成元,并使用 SSA 模拟连续时间随机游走 \(Z_t^h\)。对于给定初值与参数,独立生成 \(N_{\mathrm{SSA}}\) 条样本路径,默认 \(N_{\mathrm{SSA}}=1000\),统计
\[
p_{\mathrm{jump}}^{\mathrm{CTRW}}
= \frac{1}{N_{\mathrm{SSA}}}\sum_{k=1}^{N_{\mathrm{SSA}}}
\mathbf{1}_{\{\tau_{\mathrm{jump}}^{(k)}\le T\}},\qquad
m_{\mathrm{jump}}^{\mathrm{CTRW}}
= \frac{1}{\#\{\tau_{\mathrm{jump}}^{(k)}\le T\}}
\sum_{\tau_{\mathrm{jump}}^{(k)}\le T}\tau_{\mathrm{jump}}^{(k)}.
\]
数值结果表明,在上述网格尺度下,CTRW 得到的快跳概率接近 \(1\),平均快跳时间约为 \(m_{\mathrm{jump}}^{\mathrm{CTRW}}\approx0.5\),与采用极小步长的高精度时间离散参考解基本一致。

\paragraph{时间离散(驯服 EM)}

时间离散方法采用驯服 Euler--Maruyama 格式 \eqref{eq:canard-tamed},取不同时间步长 \(\Delta t\in\{10^{-3},\,5\times10^{-3},\,10^{-2},\,5\times10^{-2},\,10^{-1}\}\),对每个 \(\Delta t\) 独立生成 \(N_{\mathrm{EM}}\) 条样本路径(默认 \(N_{\mathrm{EM}}=1000\)),同样统计
\[
p_{\mathrm{jump}}^{\mathrm{tamed}}(\Delta t),\qquad
m_{\mathrm{jump}}^{\mathrm{tamed}}(\Delta t).
\]
实验结果可以汇总为如下趋势:
\begin{itemize}
	\item 当 \(\Delta t=10^{-3}\) 时,\(p_{\mathrm{jump}}^{\mathrm{tamed}}\) 与 \(m_{\mathrm{jump}}^{\mathrm{tamed}}\) 与 CTRW 参考值几乎一致,说明在足够小时间步长下,时间离散格式能够正确解析快跳行为;
	\item 当 \(\Delta t\) 增大到 \(10^{-2}\) 左右时,快跳仍然出现,但平均快跳时间已经出现可见偏移,表明时间离散误差开始显著影响 Canard 动力学;
	\item 当 \(\Delta t\) 进一步增大到 \(5\times10^{-2}\) 或 \(10^{-1}\) 时,虽然大部分样本路径仍然在时间窗内发生快跳,但 \(m_{\mathrm{jump}}^{\mathrm{tamed}}(\Delta t)\) 明显晚于 CTRW 参考值,偏差可达 \(10\%\sim30\%\) 量级,轨道在折叠附近的几何形状也被明显平滑化,难以辨认典型的 Canard 轨道特征。
\end{itemize}

\paragraph{整体比较与讨论}

为定量比较两类方法在快跳统计上的差异,可以定义相对误差
\[
E_p^{\mathrm{tamed}}(\Delta t)
:= \bigl\lvert p_{\mathrm{jump}}^{\mathrm{tamed}}(\Delta t)
- p_{\mathrm{jump}}^{\mathrm{ref}}\bigr\rvert,\qquad
E_m^{\mathrm{tamed}}(\Delta t)
:= \bigl\lvert m_{\mathrm{jump}}^{\mathrm{tamed}}(\Delta t)
- m_{\mathrm{jump}}^{\mathrm{ref}}\bigr\rvert,
\]
其中 \((p_{\mathrm{jump}}^{\mathrm{ref}},m_{\mathrm{jump}}^{\mathrm{ref}})\) 由 CTRW 在细网格或者极小步长的时间离散参考解给出。相应地,空间离散方法的误差
\[
E_p^{\mathrm{CTRW}}
:= \bigl\lvert p_{\mathrm{jump}}^{\mathrm{CTRW}}
- p_{\mathrm{jump}}^{\mathrm{ref}}\bigr\rvert,\qquad
E_m^{\mathrm{CTRW}}
:= \bigl\lvert m_{\mathrm{jump}}^{\mathrm{CTRW}}
- m_{\mathrm{jump}}^{\mathrm{ref}}\bigr\rvert
\]
在上述实验参数下始终保持在很小的量级内。将误差与计算代价(时间离散的总时间步数,CTRW 的总跳跃次数)共同绘制成 work--error 曲线,可以清晰地看到:

\begin{itemize}
	\item 在相同计算代价下,CTRW 空间离散方法在快跳概率与平均快跳时间上的误差显著小于大步长时间离散方法;
	\item 时间离散方法若要在快跳统计上达到与 CTRW 相当的精度,必须将时间步长缩小至 \(\Delta t\ll\delta\),这会大幅增加总步数与计算成本。
\end{itemize}

\subsection{小结}

本节基于缩放后的随机 Canard 快--慢系统,围绕“快跳是否正确出现以及快跳时间统计”这一几何敏感指标,系统比较了时间离散与空间离散两类数值格式。理论分析与数值实验表明:

\begin{enumerate}
	\item 显式时间离散方法在快变量方向的有效漂移误差为 \(\mathcal O(\Delta t/\delta)\)。当时间步长 \(\Delta t\) 与快时间尺度 \(\delta\) 同阶或更大时,这种误差足以改变系统是否处于 Canard 参数窗口,使得快跳时间出现明显偏移甚至结构性失真;
	\item 基于生成元的 CTRW 空间离散方法在快变量方向具有自适应的有效时间步长 \(\Delta t_{\mathrm{eff}}\sim h_x/\lvert\mu_1\rvert\),在折叠附近自动细化时间尺度,从而在较大的全局时间步长(从时间步进的角度看)下仍能保持 Canard 快跳的几何结构与统计特征;
	\item 在相同计算代价下,空间离散方法在“是否出现快跳行为”这一指标上明显优于时间离散方法:CTRW 能稳定再现快跳及其时间分布,而粗步长时间离散方法则对时间步长极为敏感,数值结果缺乏鲁棒性。
\end{enumerate}

这一数值实验从动力学几何的角度补充了前一节一维立方振子上的 MFPT 比较,进一步说明了在快--慢、多尺度系统中,基于生成元的空间离散方法在长时间、弱量与几何敏感指标上的优势。

%=====================================


\section{结论}

本文围绕时间离散与空间离散两类数值格式在随机微分方程模拟中的比较这一主线,选取一维立方振子和随机 Canard 快-慢系统作为代表模型,从理论分析与数值实验两个层面,对改进的显式时间步进方法(Tamed Euler-Maruyama、truncated Euler-Maruyama)和基于生成元的空间离散方法(CTRW/\(Q_u\) 及其改进格式)进行了系统研究。主要结论可概括如下。

首先,在一维加性噪声 SDE 的漂移主导情形下,本文从固定空间跨越距离所需时间的视角出发,引入了以 ODE 漂移主导时间 \(t^e\) 为参照的时间比较框架。通过对一维立方振子模型的局部渐近分析,证明了在 \(x\gg 1\)、\(\delta\ll x\) 的极限下,改进的 \(Q_u\) 生成元所对应的平均驻留时间 \(t^u\) 与精确漂移时间 \(t^e\) 在主阶与次阶上逐项一致,其相对误差随 \(|x|\to\infty\) 收敛至零,定量说明了空间离散在强漂移区间对局部时间的高精度刻度。相比之下,时间离散方法虽然在强收敛意义下能够逼近真实轨道,但其在固定空间跨越时间上的误差阶受限于时间步进的强收敛阶,通常不优于 \(\mathrm{O}(\Delta^{1/2})\),要获得与空间离散相当的精度需要更小的时间步长和更高的计算代价。

其次,针对随机 Canard 快-慢系统,本文系统引入了 逃逸概率 函数 \(q\) 与平均首达时间\(m\) 这两类生成元型弱量,并将其作为比较时间离散与空间离散优劣的核心指标。在理论方面,一方面,基于随机分析与椭圆型偏微分方程理论,证明了在适当正则性与非退化条件下,CTRW 生成元 \(Q_h\) 所给出的离散 逃逸概率 解 \(q_h\) 以 \(\mathrm{O}(h^p)\) 的阶数一致收敛到连续解 \(q\),对改进的 \(\widetilde Q_u\) 格式可达到二阶空间精度;另一方面,在 Khasminskii 型条件下,分别建立了 Tamed Euler-Maruyama 与 truncated Euler-Maruyama 对原系统的强收敛估计,并基于路径强收敛和首次到达时间尾部估计,证明了时间离散下的离散 逃逸概率 \(q_\Delta\) 收敛到 \(q\),但其误差阶仍受限于时间步进的强收敛阶,只能得到次优的弱收敛控制。本文表明在逃逸概率和首次到达时间这两个弱量上空间离散格式优于时间离散格式。

在数值实验方面,本文首先在一维立方振子模型上,通过 SSA 与 基于截断和驯服 Euler-Maruyama 格式,系统比较了固定距离下平均跨越时间与精确漂移时间 \(t^e\) 的偏差。数值结果表明在初值较大的强漂移区域,时间离散方法的误差在一定步长下则显著受限于统计方差,而空间离散所得的平均驻留时间相对于 \(t^e\) 的绝对误差随 \(X_t\) 增大而快速衰减,与理论渐近一致。在随机 Canard 系统上,本文分别从路径几何、快跳时间分布、占据度热图以及 逃逸概率和首达时间等角度,给出了时间离散方法与 CTRW的对比。数值结果显示:在有限时间条件下驯服Euler方法 与 truncated Euler-Maruyama 更擅长再现 Canard 轨道的几何细节;而在 逃逸概率 与 MFPT 等长期指标上,由于空间离散方法是从总体上对真实解的逼近,通过求解线性方程组就可一次获得全场解,明显优于基于Monte Carlo的时间离散方法。

进一步的 work–error 曲线定量比较表明:在 逃逸概率 的 RMS 误差为例的工作量相同甚至更低的情况下,CTRW/\(Q_u\) 生成元离散往往能够提供远优于 truncated Euler-Maruyama Monte Carlo 的数值精度;而在以路径级强误差为度量的有限时间窗内,时间离散方法的误差随 work 单调衰减,并在相同代价下显著优于 CTRW/SSA。这说明在随机 Canard 系统这类典型快-慢模型中,时间离散与空间离散各自适合不同的指标与应用场景:前者适用于追踪单条轨道的几何演化与局部瞬态行为,后者更适用于求解由生成元刻画的弱量与长期统计性质。

综上,本文的主要贡献可概括为三点:其一,提出以固定空间跨越距离所需时间构建了时间离散与空间离散两类数值方法的统一比较框架;其二,在一维立方振子与随机 Canard 系统上,分别给出了空间离散与时间离散在漂移主导时间和 逃逸概率/MFPT 等弱量上的理论收敛分析;其三,通过理论分析,并且设计相对应的数值实验,从而对两类方法的适用范围与优势场景给出了相关的结论。

%需要指出的是,本文仍有若干局限与可拓展方向。理论方面,目前对空间离散方法的误差分析主要局限于低维、扩散非退化且系数足够光滑的情形,对于高维、退化扩散或具有复杂边界条件的随机系统,生成元离散的稳定性与收敛性仍有待进一步严格证明。时间离散方面,如何在保持强收敛与矩有界性的前提下,设计更高阶的显式或半隐式算法,以及如何在复杂几何域中处理反射与吸收边界,也是值得深入研究的问题。数值方面,可以在更多具有代表性的模型(如多稳态反应网络、高维随机梯度系统等)上验证本文的比较框架,并考虑时间离散与空间离散的自适应与混合策略,例如在不同空间区域或不同方程部分采用不同的离散技术,以充分发挥两类方法的互补优势。

总的来看,本文所建立的时间离散与空间离散比较框架,以及在典型模型上的理论与数值分析,在随机微分方程的数值求解方法在强收敛和弱收敛、短时间的局部行为和长期总体行为之间的比较提供了两种离散格式的不同表现的判断,为在不同场景选取不同的数值格式提供了参考。随着数值分析理论与计算资源的不断发展,结合生成元离散、改进时间步进及机器学习等新技术,有望构造出既稳健又高效的随机模拟算法,为更复杂的随机动力系统的定量研究提供有力工具。



%%=====================================================================================================%%
%%
%%                参考文献
%%
%%=====================================================================================================%%

\newpage
\bibliographystyle{IEEEtran}
\bibliography{references}

% \begin{thebibliography}{99}
% \addcontentsline{toc}{section}{参考文献}

% \bibitem{1} XXXXXXXXXXXXXXXXXXXXXXXXXXXXXXXXXXXXXXXXXXXXX.

% \bibitem{2} XXXXXXXXXXXXXXXXXXXXXXXXXXXXXXXXXXXXXXXXXXXXXXXXX X XXXXXXXXXXXX.

% \bibitem{3} XXXXXXXXXXXXXXXXXXXXXXXXXXXXXXXXXXXXXXXXXXXXX.

% \bibitem{4} XXXXXXXXXXXXXXXXXXXXXXXXXXXXXXXXXXXXXXXXXXXXXXXXX X XXXXXXXXXXXXXXXXXXXXXXXXXXXXXXXXXXXXXXXXXX.

% \bibitem{5} XXXXXXXXXXXXXXXXXXXXXXXXXXXXXXXXXXXXXXXXXXXXX.

% \bibitem{6} XXXXXXXXXXXXXXXXXXXXXXXXXXXXXXXXXXXXXXXXXXXXXXXXX X XXXXXXXXXXXX.

% \bibitem{7} XXXXXXXXXXXXXXXXXXXXXXXXXXXXXXXXXXXXXXXXXXXXX.

% \bibitem{8} XXXXXXXXXXXXXXXXXXXXXXXXXXXXXXXXXXXXXXXXXXXXXXXXX X XXXXXXXXXXXXXXXXXXXXXXXXXXXXXXXXXXXXXXXXXX.

% \bibitem{9} XXXXXXXXXXXXXXXXXXXXXXXXXXXXXXXXXXXXXXXXXXXXX.

% \bibitem{10} XXXXXXXXXXXXXXXXXXXXXXXXXXXXXXXXXXXXXXXXXXXXXXXX X XXXXXXXXXXXXX.

% \bibitem{11} XXXXXXXXXXXXXXXXXXXXXXXXXXXXXXXXXXXXXXXXXXXXX.

% \bibitem{12} XXXXXXXXXXXXXXXXXXXXXXXXXXXXXXXXXXXXXXXXXXXXXXXX X XXXXXXXXXXXXXXXXXXXXXXXXXXXXXXXXXXXXXXXXXXX.

% \end{thebibliography}

\newpage


%%=====================================================================================================%%
%%
%%                附录  该部分若没有,可删去
%%
%%=====================================================================================================%%

\section*{附录~A\quad XXXX统计数据}
\addcontentsline{toc}{section}{附录~A\quad XXXX统计数据}   %  加入目录 section 级  及文本

文字文字文字文字文字文字文字文字文字文字文字文字文字文字文字文字文字文字文字文字文字文字文字文字文字文字文字文字文字文字文字文字。

\newpage

\section*{附录~B\quad XXXX统计数据}
\addcontentsline{toc}{section}{附录~A\quad XXXX统计数据}

文字文字文字文字文字文字文字文字文字文字文字文字文字文字文字文字文字文字文字文字文字文字文字文字文字文字文字文字文字文字文字文字。


%%=====================================================================================================%%
%%
%%                后记
%%
%%=====================================================================================================%%

\newpage
\section*{后\hspace{2\ccwd}记}
\addcontentsline{toc}{section}{后记}

后记(包括致谢):后记主要叙述与学位论文写作工作有关的其他内容,可以包括论文的说明、致谢等内容。
致谢部分对指导教师和给予指导或协助完成学位论文工作的组织和个人表示感谢。内容要求简洁明了、实事求是,语言诚恳、恰当。



%%=====================================================================================================%%
%%
%%                在学期间取得创新性成果情况
%%
%%=====================================================================================================%%

\newpage

\section*{在学期间取得创新性成果情况}
\addcontentsline{toc}{section}{在学期间取得创新性成果情况}

\begin{center}
\begin{tblr}{width=\textwidth,colspec={|Q[c,44mm]|Q[c,20mm]|Q[c,40mm]|Q[c,24mm]|X[c]|},  %  表格宽度、列格式
             rowspec={|[1pt]Q[m]|Q[m]|Q[m]|Q[m]|Q[m]|Q[m]|[1pt]},}                       %  行格式
{\heiti 成果名称} & {\heiti 成果类别} & {\heiti 刊物名称/出版社名称} & {\heiti 刊发时间} & {\heiti 作者\\ 次序}\\
          & 学术论文  &                      &           &         \\
          & 学术著作  &                      &           &         \\
          & $\cdots\cdots$ &                 &           &         \\
          &           &                      &           &         \\
          &           &                      &           &         \\
\end{tblr}
\end{center}

%%%%%%%%%%%%%%%%%%%%%%%%%%%%%%%%%%%%%%%%%%%%%%%%%%%%%%%%%%%%%%%%%%%%%%%%%%%%%%%
%%%%%%%%%%%%%%%%%%%%%%%%%%%%%%%%%%%%%%%%%%%%%%%%%%%%%%%%%%%%%%%%%%%%%%%%%%%%%%%

\end{document}
