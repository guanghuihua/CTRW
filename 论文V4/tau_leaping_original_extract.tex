


%%%%%%%%%%%%%%%%%%%%%%%%%%%%%%%%%%%%%%%%%%%%%%%%%%%%%%%%%%%%%%%%%%%%%%%%%%%%%%%%%%%%%%%%%%%%%%%%%%%%%%%%
%%%%%%%%%%%%%%%%   东北师范大学硕士学位论文模板(2023版) v1.0alpha
%%%%%%%%%%%%%%%%   DESIGNED BY ZHAO HONGLIANG
%%%%%%%%%%%%%%%%   在 CTeX_3.0.212.1 中使用  pdflatex 编译通过
%%%%%%%%%%%%%%%%   在 texlive 2022 中使用  pdflatex 编译通过
%%%%%%%%%%%%%%%%%%%%%%%%%%%%%%%%%%%%%%%%%%%%%%%%%%%%%%%%%%%%%%%%%%%%%%%%%%%%%%%%%%%%%%%%%%%%%%%%%%%%%%%%%
%%%%%%%%%%%%%%%%   需将校徽和校名图片文件 xiaohui.png 和 xiaoming.png 放在源程序文件夹中
%%%%%%%%%%%%%%%%   在\begin{document} 命令后输入论文相关信息,用以自动生成封面等
	%%%%%%%%%%%%%%%%   使用 pdflatex 编译
	%%%%%%%%%%%%%%%%   脚注序号为带圈数字,每页最大编号为 10,超过 10 需手动处理
	%%%%%%%%%%%%%%%%%%%%%%%%%%%%%%%%%%%%%%%%%%%%%%%%%%%%%%%%%%%%%%%%%%%%%%%%%%%%%%%%%%%%%%%%%%%%%%%%%%%%%%%%%
	%%%%%%%%%%%%%%%%
	%%%%%%%%%%%%%%%%   导言区
	%%%%%%%%%%%%%%%%
	%%%%%%%%%%%%%%%%%%%%%%%%%%%%%%%%%%%%%%%%%%%%%%%%%%%%%%%%%%%%%%%%%%%%%%%%%%%%%%%%%%%%%%%%%%%%%%%%%%%%%%%%%
	%%=====================================================================================================%%
	%%
	%%                设置文档类别   和   页面格式
	%%
	%%=====================================================================================================%%
	\documentclass[a4paper,twoside,openany,UTF8]{article}  %A4纸张,双面排版,每章后不留空白页,UTF-8编码
	\usepackage[heading=true]{ctex}  %添加中文及版式的支持
	\usepackage[total={160mm,257mm},inner=25mm,outer=25mm,top=20mm,includeheadfoot,%
	headheight=15mm,headsep=8mm,footskip=17.5mm,centering]{geometry}  % 使用 geomerty 宏包设置页面格式
	\renewcommand{\baselinestretch}{1.25}  %设置行距=默认行距(1.2倍)*1.25=1.5倍
	%%=====================================================================================================%%
	%%
	%%                加载所需宏包,可根据需要增删 (宏包功能可参考 LaTeX 编辑部 网站的简要说明)
	%%
	%%=====================================================================================================%%
	\usepackage{mathrsfs}  % 加载 mathrsfs 字体宏包,在数学中使用 Raph Smith’s Formal Script 字体
	\usepackage{amsmath,amssymb,amsthm}  % 加载数学公式、数学符号、定理和证明排版宏包
	\usepackage{graphicx,curves,epic}  % 加载图形宏包、绘图宏包、绘图宏包
	\usepackage{subfig}  % 加载子图宏包。 subfig 宏包是 subfigure 宏包的升级版,且二者冲突
	\usepackage{tikz,pgfplots,circuitikz}  % 加载绘图、2D3D和散点图绘制、电路图绘制宏包
	\usepackage{xcolor}  % 加载颜色处理宏包,是 color 宏包的加强版
	\usepackage{calc}  % 加载 LaTeX 的算术运算增强宏包
	\usepackage{array,tabularx}  % array 和 tabular 环境功能增强宏包、自动计算表格列宽宏包
	\usepackage{booktabs}  % 表格顶部、中部和底部使用不同粗细的水平线宏包
	\usepackage{tabularray}  % 超好用的新一代表格排版宏包
	\usepackage[labelsep=quad]{caption}  % 加载图表标题宏包,本文设置分隔符为一个\quad
	\usepackage[T1]{fontenc}  % 加载字体宏包,调用 T1 科克编码字体
	\usepackage{extarrows}  % 加载长度自适应箭标宏包
	\usepackage{bm}  % 以粗体方式显示数学公式宏包。它提供一个在数学模式中使用的 \bm{数学式} 命令
	%\usepackage{appendix}  % 加载附录宏包
	\usepackage{float,floatflt}  % 加载新浮动体宏包、图文混排宏包
	%\usepackage{floatrow}  % 加载灵活排版插图和表格浮动体宏包,建议同时加载 graphicx 宏包和 subcaption 宏包
	%\usepackage{graphicx}
	%\usepackage{subcaption}  % 加载设置子标题宏包
	%\usepackage{wrapstuff}  % 加载另一个图文混排宏包
	%%=====================================================================================================%%
	%%
	%%                设置页眉页脚,页眉居中显示东北师范大学硕士学位论文,页脚居中显示页码,页眉有横线
	%%
	%%=====================================================================================================%%
	\usepackage{fancyhdr}  %change page margings and sizes, headers and footers,
	\pagestyle{fancy}   %紧跟 \usepackage{fancyhdr},
	\fancyhead{}  %清除页眉页脚
	%\fancyhead[L,R]{}  %设置页眉左右位置为空
	\fancyhead[C]{东北师范大学硕士学位论文}  %设置页眉居中位置
	\fancyfoot{}  %清除页脚
	%\fancyfoot[L,R]{}  %设置页角左右位置为空
	\fancyfoot[C]{\thepage}  %设置角眉居中位置显示页码
	\renewcommand{\headrulewidth}{1pt}  %设置页眉线宽度为 1 磅
	%%=====================================================================================================%%
	%%
	%%                 使用 CTeX 宏包设置节、小节、小小节标题格式
	%%
	%%=====================================================================================================%%
	\ctexset{
		section={  %  设置节标题格式
			name         = {,\hspace{-0.5\ccwd}},
			beforeskip   = 48pt,
			fixskip      = true,
			format       = \centering\heiti\bf\zihao{3},
			numberformat = \heiti\bf\zihao{3},
			afterskip    = 24pt,
		},
		subsection={  %  设置小节标题格式
			name         = {,\hspace{-0.5\ccwd}},
			beforeskip   = 6pt,
			format       = \heiti\bf\zihao{4},
			numberformat+ = \heiti\bf\zihao{4},
			afterskip    = 0pt,
		},
		subsubsection={  %  设置小小节标题格式
			name         = {,\hspace{-0.5\ccwd}},
			beforeskip   = 6pt,
			format       = \songti\bf\zihao{-4},
			numberformat = \bf\zihao{-4},
			afterskip    = 0pt,
		}
	}
	%%=====================================================================================================%%
	%%
	%%                 设置目录深度、格式
	%%
	%%=====================================================================================================%%
	\usepackage{titletoc} %加载目录格式设置宏包
	%==============    设置章节目录格式
	\setcounter{tocdepth}{3}  %设置章节目录深度。article版式没有章层次标题,一级标题为节
	\renewcommand\contentsname{目\hspace{2\ccwd}录}     %修改目录标题
	\titlecontents{section}[0\ccwd]                     %标题名:节,左间距为 0(首行无缩进与突出)
	{\addvspace{.3\baselineskip}\zihao{-4}\heiti}   %标题格式:与上一个标题增加0.3倍行距,小四号黑体
	{\contentslabel{1\ccwd}}                        %标题标志:标题标志宽度为 1个汉字宽度
	{\hspace*{-1\ccwd}}                             %无序号标题格式:前移 1个汉字宽度
	{\hspace{0.5\ccwd}\titlerule*{.}\contentspage}  %指引线与页码:与标题内容间距半个汉字,点填充,页码
	
	\titlecontents{subsection}[1\ccwd]                  %标题名:小节,左间距为 1个汉字宽度(首行无缩进与突出)
	{\addvspace{.3\baselineskip}\zihao{-4}\songti}
	{\contentslabel{1\ccwd}\hspace{1\ccwd}}         %标题标志:标题标志宽度为 1个汉字宽度,后面增加 1个汉字宽度
	{\hspace*{-1\ccwd}}
	{\hspace{0.5\ccwd}\titlerule*{.}\contentspage}
	
	\titlecontents{subsubsection}[2\ccwd]
	{\addvspace{.3\baselineskip}\zihao{-4}\songti}
	{\contentslabel{1\ccwd}\hspace{1.5\ccwd}}
	{\hspace*{-1\ccwd}}
	{\hspace{0.5\ccwd}\titlerule*{.}\contentspage}
	
	%==============    设置插图目录格式
	\renewcommand{\listfigurename}{插图目录}             %修改插图目录标题
	\titlecontents{figure}[0\ccwd]                      %标题名:图,左间距为 0(首行无缩进与突出)
	{\addvspace{.3\baselineskip}\zihao{-4}\songti}  %标题格式:与上一个标题增加0.3倍行距,小四号宋体
	{图~\thecontentslabel{\makebox[3mm]{}}}         %标题标志:标题标志与标题内容间距 3mm
	{}                                              %无序号标题格式:空置
	{\hspace{0.5\ccwd}\titlerule*{.}\contentspage}  %指引线与页码:与标题内容间距半个汉字,点填充,页码
	
	%==============    设置附表目录格式
	\renewcommand{\listtablename}{附表目录}             %修改表格目录标题
	\titlecontents{table}[0\ccwd]                       %标题名:表,左间距为 0(首行无缩进与突出)
	{\addvspace{.3\baselineskip}\zihao{-4}\songti}  %标题格式:与上一个标题增加0.3倍行距,小四号黑体
	{表~\thecontentslabel{\makebox[3mm]{}}}         %标题标志:标题标志与标题内容间距 3mm
	{}                                              %无序号标题格式:前移 1个汉字宽度
	{\hspace{0.5\ccwd}\titlerule*{.}\contentspage}  %指引线与页码:与标题内容间距半个汉字,点填充,页码
	
	%%=====================================================================================================%%
	%%
	%%                 设置脚注显示符号为带圈的数字,大于 10 的不能正常显示
	%%
	%%=====================================================================================================%%
	%\renewcommand{\thefootnote}{\fnsymbol{footnote}}
	\usepackage{pifont}  %加载提供文稿中常见的符号的宏包,选择命令 \ding {代号}
	\usepackage[perpage,stable,symbol*]{footmisc}  %加载自定义脚注符号宏包,每页独立编号,
	\newcommand*\dingctr[1]{\protect\ding{\number\numexpr\value{#1}+171\relax}}  % 调用 \ding 中带圈的数字
	\renewcommand*\thefootnote{\dingctr{footnote}}  % 生成带圈数字脚注,大于 10 的不能正常显示
	
	\makeatletter
	%%%% 悬挂的脚注格式
	\renewcommand\@makefntext[1]{%
		\setlength\leftskip{1.2\ccwd}%
		\setlength\parindent{2\ccwd}\selectfont
		\noindent\llap{\@thefnmark\ }#1}
	%%%% 无悬挂的脚注格式
	%\renewcommand\@makefntext[1]{%
		%    \setlength\parindent{2\ccwd}\selectfont
		%    \@thefnmark\ #1}
	\makeatother
	
	\renewcommand{\footnotesize}{\zihao{-5}}
	
	%%=====================================================================================================%%
	%%
	%%                 其它设置
	%%
	%%=====================================================================================================%%
	
	%==============    默认罗马字体
	\renewcommand{\rmdefault}{ptm}  % pdf架构下设置默认罗马字体为 Times New Roman
	
	%==============    设置公式、图表编号格式
	\numberwithin{equation}{section}  %needs amsmath packge %公式在节内编号
	\renewcommand{\theequation}{\thesection-\arabic{equation}}
	\renewcommand{\thefigure}{\thesection.\arabic{figure}}
	\renewcommand{\thetable}{\thesection.\arabic{table}}
	
	\numberwithin{equation}{section}  % 这样所有公式(不管是 equation 还是 align)都会按照 (章节号.公式号) 的规则编号
	%==============    设置参考文献名
	\CTEXoptions[bibname={参考文献}]
	
	%==============    设置附录名
	\renewcommand\appendix{\par
		\setcounter{section}{0}
		\setcounter{subsection}{0}
		\gdef\thesection{附录 \Alph{section}}}
	
	%==============    定义新的列表环境,使说明文字左对齐,用以排版等
	\newenvironment{newdescription}[1]%
	{\begin{list}{}{\renewcommand{\makelabel}[1]{\songti{##1}\hfil}%
				\settowidth{\labelwidth}{\songti{#1}}%
				\setlength{\labelsep}{0.5\ccwd}%
				\setlength{\parsep}{0pt}%
				\setlength{\itemsep}{2.5pt}%
				\setlength{\leftmargin}{\labelwidth+\labelsep}}}%
		{\end{list}}
	
	%==============    使用 \tikz 定义带圈数字
	\newcommand*\circled[1]{\tikz[baseline=(char.base)]{\node[shape=circle,draw,inner sep=0.2pt] (char) {#1};}}
	
	%==============俄文字母
	\font\ewenb=wncyb10 \font\eweni=wncyi10 \font\ewenr=wncyr10
	\font\ewensc=wncysc10 \font\ewenss=wncyss10
	
	%==============定理设置==============%
	\newtheorem{corollary}{推论}[section]
	\newtheorem{criterion}{Criterion}[section]
	\newtheorem{definition}{定义}[section]
	\newtheorem{example}{例}[section]
	\newtheorem{lemma}{引理}[section]
	\newtheorem{notation}{Notation}[section]
	\newtheorem{proposition}{命题}[section]
	\newtheorem{remark}{Remark}[section]
	\newtheorem{theorem}{定理}[section]
	\newtheorem{assumption}{假设}[section]
	
	%==============定义带左右标号的公式环境==============%
	\makeatletter
	\def\xlabel#1#2{%
		{\@bsphack\protected@write\@auxout{}%
			{\string\newlabel{#2}{{#1}{\thepage}}}%
			\@esphack}{\mathrm(#1)}}
	\makeatother
	%定义结束%
	%下面是一个例子,注意&&的用法%
	%\begin{flalign}
	%\xlabel{H1}{eq:refL}&&x=y+z&&
	%\label{eq:ee1}\\
	%\xlabel{H2}{eq:xxy}&&a=b^2+c^2-a&&
	%\label{eq:ee2}
	%\end{flalign}
	%例子结束%
	
	\renewcommand{\thetheorem}{\thesection.\arabic{theorem}}
	\renewcommand{\thelemma}{\thesection.\arabic{lemma}}
	\renewcommand{\thecorollary}{\thesection.\arabic{corollary}}
	\renewcommand{\theremark}{\thesection.\arabic{remark}}
	\renewcommand{\thedefinition}{\thesection.\arabic{definition}}
	\renewcommand{\theproposition}{\thesection.\arabic{proposition}}
	\renewcommand{\theexample}{\thesection.\arabic{example}}
	
	%==============定义上角标引用参考文献==============%
	\newcommand{\upcite}[1]{\textsuperscript{\cite{#1}}}
	
	%==============定义新函数==============%
	\DeclareMathOperator*{\esssup}{\mathrm{ess}\sup}
	
	%==============设置算法环境==============%
	\usepackage[linesnumbered,ruled,vlined]{algorithm2e}
	\usepackage{bm}
	% \usepackage[ruled]{algorithm2e}
	% 中文替换与格式设定
	\renewcommand{\algorithmcfname}{算法}
	\SetKwInOut{KwInput}{输入}     % 取消缩进的输入
	\SetKwInOut{KwOutput}{输出}    % 取消缩进的输出
	\SetNlSty{}{}{}                % 行号样式(无加粗括号)
	\LinesNumbered                 % 显示行号
	\SetAlCapHSkip{0pt}           % 标题与正文左对齐
	
	\SetAlgoNlRelativeSize{0}  % 控制编号字号
	\SetNlSkip{0.2em}          % 控制编号右边的空隙
	\SetNlSty{textbf}{\ }{}    % 控制编号格式
	
	%==============生成书签==============%
	\usepackage{hyperref}
	
	%%%%%%%%%%%%%%%%%%%%%%%%%%%%%%%%%%%%%%%%%%%%%%%%%%%%%%%%%%%%%%%%%%%%%%%%%%%%%%%%%%%%%%%%%%%%%%%%%%%%%%%%%
	%%%%%%%%%%%%%%%%
	%%%%%%%%%%%%%%%%   开始正文区
	%%%%%%%%%%%%%%%%
	%%%%%%%%%%%%%%%%%%%%%%%%%%%%%%%%%%%%%%%%%%%%%%%%%%%%%%%%%%%%%%%%%%%%%%%%%%%%%%%%%%%%%%%%%%%%%%%%%%%%%%%%%
	
	\begin{document}
		
	%=========================================================
	%  §2.3 tau-leaping 原理(恢复公式版)
	%=========================================================
	\section{tau-leaping 原理}
	
	考虑搅拌均匀的化学反应系统,包含 $n$ 种反应物种 $\{S_i\}_{i=1}^n$,具有 $M$ 条反应通道。
	系统状态记为
	\[
	X(t) = (X_1(t),\dots,X_n(t))\in \mathbb{Z}_{\ge 0}^n,
	\]
	其中 $X_i(t)$ 表示时刻 $t$ 物种 $S_i$ 的分子数。
	第 $j$ 条反应通道由倾向函数 $a_j(x)$ 与状态改变量(化学计量向量)
	\[
	\nu_j=(\nu_{1j},\dots,\nu_{nj})^\top
	\]
	描述:对给定状态 $X(t)=x$,在小时间段 $[t,t+dt)$ 内第 $j$ 条反应发生的概率近似为
	$a_j(x)\,dt$;当该反应发生时,系统状态变化为 $x\mapsto x+\nu_j$。
	
	系统动力学满足化学主方程(CME):
	\begin{equation}\label{eq:CME}
		\frac{d}{dt}P(x,t\,|\,x_0,t_0)
		=
		\sum_{j=1}^M\Bigl[
		a_j(x-\nu_j)\,P(x-\nu_j,t\,|\,x_0,t_0)
		-
		a_j(x)\,P(x,t\,|\,x_0,t_0)
		\Bigr].
	\end{equation}
	
	除少数简单系统外,直接求解 \eqref{eq:CME} 代价很高,因此常用随机模拟算法(SSA)近似轨道。
	令
	\[
	a_0(x)=\sum_{j=1}^M a_j(x).
	\]
	在状态 $x$ 处,下一个反应发生等待时间 $T$ 服从参数为 $a_0(x)$ 的指数分布,
	反应通道 $j$ 被触发的概率为 $a_j(x)/a_0(x)$。
	SSA 通过两个独立的均匀随机数 $r_1,r_2\sim U(0,1)$ 生成:
	\begin{equation}\tag{2.13}\label{eq:SSA_time}
		\tau
		=
		\frac{1}{a_0(x)}
		\ln\!\Bigl(\frac{1}{r_1}\Bigr),
	\end{equation}
	并选择满足
	\begin{equation}\tag{2.14}\label{eq:SSA_channel}
		\sum_{j=1}^{\mu-1}a_j(x)\;<\;r_2\,a_0(x)\;\le\;\sum_{j=1}^{\mu}a_j(x)
	\end{equation}
	的通道 $\mu$,随后更新
	\[
	X(t+\tau)=X(t)+\nu_\mu.
	\]
	
	SSA 对每一次反应都逐事件推进,计算量很大。
	Gillespie 提出 \textit{tau-leaping} 思想:将时间轴划分为连续小区间,
	在每个区间内倾向函数变化不大,则可放弃反应精确发生时刻,直接从一个区间跳到下一个区间。
	设 $K_j(\tau,x)$ 表示在固定区间 $[t,t+\tau)$ 内第 $j$ 条反应的真实发生次数。
	若在该区间内满足 \textit{leap condition}(倾向函数近似常数),则有近似
	\begin{equation}\tag{2.15}\label{eq:Poisson_leap}
		K_j(\tau,X(t)) \;\approx\; \mathrm{Poisson}\bigl(a_j(X(t))\,\tau\bigr),
		\qquad j=1,\dots,M,
	\end{equation}
	并可一次性更新状态
	\begin{equation}\tag{2.16}\label{eq:leap_update}
		X(t+\tau)
		=
		X(t)+\sum_{j=1}^M \nu_j\,K_j(\tau,X(t)).
	\end{equation}
	常用的一种跳跃条件写为:存在给定阈值 $\varepsilon\in(0,1)$,使得对所有 $j$,
	\begin{equation}\tag{2.17}\label{eq:leap_condition}
		\bigl|a_j(X(t+\tau)) - a_j(X(t))\bigr|
		\le
		\varepsilon\,a_0(X(t)).
	\end{equation}
	
	%=========================================================
	%  §3.1 基于空间离散生成元Q的 tau-leaping(恢复公式版)
	%=========================================================
	\subsection{基于空间离散格式的 tau-leaping 算法}
	
	考虑一维随机微分方程
	\begin{equation}\label{eq:SDE_1d}
		dX_t=b(X_t)\,dt+\sigma\,dW_t,
	\end{equation}
	其生成元为
	\[
	\mathcal{L}f(x)=b(x)f'(x)+\frac{\sigma^2}{2}f''(x).
	\]
	在区间 $\Omega=[x_{\min},x_{\max}]$ 上取均匀网格 $x_i=x_{\min}+ih$。
	离散生成元 $Q$ 的基本思想是:构造一个在离散状态空间上跳跃的连续时间马尔可夫链,
	使其无穷小生成元矩阵 $Q$ 近似 $\mathcal{L}$。
	
	\paragraph{(1) 中心差分生成元 $Q_c$(两点跳跃)}
	令 $D=\sigma^2/2$,在节点 $x_i$ 处定义向右、向左跳跃速率
	\begin{equation}\label{eq:Qc_rates}
		q^{c}_{i,i+1}=\frac{D}{h^2}+\frac{b(x_i)}{2h},\qquad
		q^{c}_{i,i-1}=\frac{D}{h^2}-\frac{b(x_i)}{2h},
	\end{equation}
	并令
	\[
	q^{c}_{i,i}=-(q^{c}_{i,i+1}+q^{c}_{i,i-1}).
	\]
	当漂移较大时,\eqref{eq:Qc_rates} 可能导致非负性问题(某些速率为负)。
	
	\paragraph{(2) 迎风生成元 $Q_u$(保证非负性)}
	典型的迎风构造可写为(按 $b(x_i)$ 符号分段):
	\begin{equation}\label{eq:Qu_rates}
		\begin{aligned}
			&\text{若 } b(x_i)\ge 0:\qquad
			q^{u}_{i,i+1}=\frac{D}{h^2}+\frac{b(x_i)}{h},\qquad
			q^{u}_{i,i-1}=\frac{D}{h^2},\\
			&\text{若 } b(x_i)<0:\qquad
			q^{u}_{i,i+1}=\frac{D}{h^2},\qquad
			q^{u}_{i,i-1}=\frac{D}{h^2}-\frac{b(x_i)}{h}.
		\end{aligned}
	\end{equation}
	并同样设 $q^{u}_{i,i}=-(q^{u}_{i,i+1}+q^{u}_{i,i-1})$。
	
	\paragraph{(3) 在 $Q$-跳过程上的 tau-leaping 更新}
	将离散跳过程理解为“两个反应通道”:向右跳、向左跳。
	在当前位置 $x_i$,取时间步长 $\tau$,生成独立泊松随机数
	\[
	K_+ \sim \mathrm{Poisson}(q_{i,i+1}\tau),\qquad
	K_- \sim \mathrm{Poisson}(q_{i,i-1}\tau),
	\]
	则一次跳跃步更新为
	\begin{equation}\tag{3.2}\label{eq:Q_tau_update}
		i \leftarrow i + K_+ - K_-,
		\qquad\text{即}\qquad
		X \leftarrow x_{\,i+K_+-K_-}.
	\end{equation}
	多维情形对应为 $2d$ 个方向的泊松计数与向量更新(例如二维为四个方向)。
	
	%=========================================================
	%  §3.3 中点格式的 tau-leaping(恢复公式版,按原意)
	%=========================================================
	\subsection{中点格式的 tau-leaping 算法}
	
	为提高精度,可借鉴中点思想:先构造“中点状态”,再在中点处计算速率并采样泊松次数。
	设当前离散状态为 $x$(或索引为 $i$),令速率向量记为 $q(x)$(包含各方向/各通道速率)。
	记跳跃方向向量为 $\{\nu_j\}$(一维情形 $\nu_+=+h,\nu_-=-h$)。
	定义中点预测(用期望漂移近似):
	\begin{equation}\label{eq:midpoint_predict}
		x_{\mathrm{mid}}
		=
		x+\frac{\tau}{2}\sum_j \nu_j\,q_j(x),
	\end{equation}
	然后在 $x_{\mathrm{mid}}$ 处计算速率 $q_j(x_{\mathrm{mid}})$ 并生成
	\[
	K_j \sim \mathrm{Poisson}\bigl(q_j(x_{\mathrm{mid}})\tau\bigr),
	\]
	最后更新
	\begin{equation}\label{eq:midpoint_update}
		x \leftarrow x+\sum_j \nu_j\,K_j.
	\end{equation}
	
	%=========================================================
	%  §3.4 带拒绝机制的 tau-leaping(恢复公式版,按原意)
	%=========================================================
	\subsection{带拒绝机制的 tau-leaping 算法}
	
	在实际模拟中可能出现泊松跳跃次数过大,使得 leap condition 不满足,
	或者使粒子跳出计算区域 $\Omega$,从而导致该步无效。
	因此可引入拒绝机制(rejection-based):
	
	给定当前状态 $x$ 与步长 $\tau$,先生成
	\[
	K_j \sim \mathrm{Poisson}\bigl(q_j(x)\tau\bigr),
	\]
	得到候选更新
	\[
	x^\star = x+\sum_j \nu_j K_j.
	\]
	若发生以下任一情形,则拒绝该步:
	\begin{itemize}
		\item $x^\star\notin \Omega$(越界);
		\item leap condition 不满足,例如
		\begin{equation}\label{eq:reject_leap}
			\bigl|q_j(x^\star)-q_j(x)\bigr|\le \varepsilon\,q_0(x)
			\quad (j=1,\dots,\text{通道数}),
			\qquad q_0(x)=\sum_j q_j(x),
		\end{equation}
		不成立(这里 $\varepsilon\in(0,1)$ 为阈值)。
	\end{itemize}
	拒绝后令步长缩小(常用 $\tau\leftarrow\tau/2$)并重新采样;若通过检查则接受:
	\[
	x\leftarrow x^\star,\qquad t\leftarrow t+\tau.
	\]
	
	
	%=========================================================
	% 新增算例:随机 Canard 慢快系统的平稳密度(插入到“数值算例”章)
	%=========================================================
	
	\section{随机Canard慢快系统的平稳密度计算}\label{subsec:canard}
	
	\subsection{模型、参数与Canard爆炸背景}
	
	本节将前述基于空间离散生成元 \(Q\) 的 \textit{tau-leaping} 方法应用于随机慢快 Canard 系统,
	用于计算其二维平稳密度并讨论其与 Canard 爆炸现象的联系。考虑如下带加性噪声的慢快系统:
	\begin{equation}\label{eq:canard_sde}
		\begin{cases}
			\delta\, dX_t = \bigl(Y_t - \dfrac{X_t^3}{3} + X_t\bigr)\,dt + \sigma_x\, dW_t^{(1)},\\[6pt]
			dY_t = (a - X_t)\,dt + \sigma_y\, dW_t^{(2)},
		\end{cases}
	\end{equation}
	其中 \(0<\delta\ll 1\) 为慢快参数,\(a\) 为控制参数,\(W_t^{(1)},W_t^{(2)}\) 为相互独立的标准布朗运动。
	将其写成标准形式:
	\begin{equation}\label{eq:canard_standard}
		\begin{aligned}
			dX_t &= b_x(X_t,Y_t)\,dt + \tilde\sigma_x\, dW_t^{(1)},\qquad
			b_x(x,y)=\frac{1}{\delta}\Bigl(y-\frac{x^3}{3}+x\Bigr),\quad \tilde\sigma_x=\frac{\sigma_x}{\delta},\\
			dY_t &= b_y(X_t,Y_t)\,dt + \sigma_y\, dW_t^{(2)},\qquad
			b_y(x,y)=a-x .
		\end{aligned}
	\end{equation}
	
	在确定性情形(\(\sigma_x=\sigma_y=0\))下,该系统存在典型的慢流形/临界流形结构。
	当 \(a\) 穿越一个窄参数区间时,轨道振幅会从小振幅迅速过渡到大振幅回转(即 Canard 爆炸)。
	为使系统处于接近爆炸阈值的区域,本文采用与慢快渐近展开一致的参数选取(示例):
	\begin{equation}\label{eq:canard_param}
		\delta=0.1,\qquad
		a = 1 - \frac{\delta}{8} - \frac{3\delta^2}{32} - \frac{173\delta^3}{1024} - \eta,
	\end{equation}
	其中 \(\eta>0\) 为人为偏移量(用于将系统放置在“接近但略偏离”爆炸临界的区域)。
	随机扰动加入后,轨道会在吸引/排斥慢流形附近产生随机偏移,并在长时间尺度上呈现稳定的统计占据结构,
	从而诱导出二维平稳密度 \(p_\infty(x,y)\)。
	
	\subsection{生成元与空间离散:四方向跳跃速率 \(q_1,\dots,q_4\)}
	
	对足够光滑的测试函数 \(f\),\eqref{eq:canard_standard} 的生成元为
	\begin{equation}\label{eq:canard_generator}
		\mathcal{L}f
		= b_x\,\partial_x f + b_y\,\partial_y f
		+ D_x\,\partial_{xx}f + D_y\,\partial_{yy}f,
		\qquad
		D_x=\frac{\tilde\sigma_x^2}{2}=\frac{\sigma_x^2}{2\delta^2},\quad
		D_y=\frac{\sigma_y^2}{2}.
	\end{equation}
	在矩形区域 \(\Omega=[x_{\min},x_{\max}]\times[y_{\min},y_{\max}]\) 上取均匀网格步长 \(h_x,h_y\),
	对 \(\mathcal{L}\) 做空间离散得到离散生成元 \(Q\)。
	本文采用“每次向四个相邻方向跳一步”的 CTRW/SSA 结构:
	\[
	(x_i,y_j)\to (x_{i\pm 1},y_j),\qquad (x_i,y_j)\to(x_i,y_{j\pm 1}).
	\]
	记四个方向的跳跃速率分别为
	\[
	q_1=q_x^+,\quad q_2=q_x^-,\quad q_3=q_y^+,\quad q_4=q_y^-.
	\]
	为在漂移占优与扩散占优区域均保持数值鲁棒性,可采用指数拟合(exponential fitting)形式的速率构造(示例写法):
	\begin{equation}\label{eq:canard_rates}
		\begin{aligned}
			q_x^\pm(x,y)
			&= \frac{D_x}{h_x^2}\exp\!\Bigl(\pm \frac{h_x\,b_x(x,y)}{2D_x}\Bigr)
			= \frac{D_x}{h_x^2}\exp\!\Bigl(\pm \frac{h_x\,b_x(x,y)}{\tilde\sigma_x^2}\Bigr),\\
			q_y^\pm(x,y)
			&= \frac{D_y}{h_y^2}\exp\!\Bigl(\pm \frac{h_y\,b_y(x,y)}{2D_y}\Bigr)
			= \frac{D_y}{h_y^2}\exp\!\Bigl(\pm \frac{h_y\,b_y(x,y)}{\sigma_y^2}\Bigr).
		\end{aligned}
	\end{equation}
	于是 \(Q\) 过程可解释为连续时间马尔可夫跳过程,其总速率为
	\[
	\lambda(x,y)=q_1+q_2+q_3+q_4.
	\]
	\textbf{边界处理:} 若跳跃会越出 \(\Omega\),可采用反射边界(将越界跳跃映射回边界点)或直接置零相应越界通道速率。
	为避免边界截断影响平稳密度,\(\Omega\) 需足够覆盖轨道的主要占据区域。
	
	\subsection{\textit{tau-leaping}在随机Canard中的实现要点}
	
	在 SSA 中逐事件推进会非常耗时。本节直接使用前文已给出的 \textit{tau-leaping} 主算法(保持不变),
	在本算例中需要强调以下实现要点:
	
	\paragraph{(1) 步长 \(\tau\) 的自适应控制}
	由于 \(b_x\) 含有 \(1/\delta\) 因子,慢快结构导致在某些区域局部速率变化显著。
	因此采用简单的上界控制原则(示例):
	\begin{equation}\label{eq:tau_choice}
		\tau(x,y)=\min\Bigl\{\tau_{\max},\, \frac{\Lambda}{\max\{q_1,q_2,q_3,q_4\}}\Bigr\},
	\end{equation}
	其中 \(\Lambda\in(0,1)\) 控制单步内期望跳跃次数上限,\(\tau_{\max}\) 给出最大步长。
	当进入快区(速率变大)时 \(\tau\) 自动缩小,以满足 leap condition 的要求。
	
	\paragraph{(2) 拒绝机制与越界控制}
	一次 \textit{tau-leaping} 可能产生较大泊松计数导致越界或 leap condition 失效。
	可使用前文的拒绝机制:若出现越界或 \(\max_j K_j\) 超过阈值,则拒绝该步并令 \(\tau\leftarrow \tau/2\) 重试。
	
	\paragraph{(3) 平稳密度统计:burn-in 与占据时间加权}
	设总模拟时间为 \(T\),丢弃前 \(T_{\mathrm{burn}}\) 段作为 burn-in,
	在 \([T_{\mathrm{burn}},T]\) 上统计直方图。
	对连续时间跳过程,更推荐以“停留时间”加权:
	每一步推进 \(t\leftarrow t+\tau\) 时,将 \(\tau\) 累加到当前箱体的占据时间上,
	最后归一化得到 \(\hat p_\infty(x,y)\)。
	
	\subsection{数值设置与结果展示}
	
	本节给出一组可复现实验设置(你可按实际计算替换具体数值):
	\begin{itemize}
		\item 参数:\(\delta=0.1\),\(a\) 由 \eqref{eq:canard_param} 给出(例如取 \(\eta=0.01\)),噪声强度 \(\sigma_x,\sigma_y\) 取 \(10^{-2}\sim 10^{-1}\) 量级;
		\item 区域与网格:\(\Omega=[x_{\min},x_{\max}]\times[y_{\min},y_{\max}]\),网格步长 \(h_x=h_y=h\);
		\item \textit{tau-leaping}:\(\tau\) 由 \eqref{eq:tau_choice} 自适应选取,启用拒绝机制;
		\item 统计:总时长 \(T\),burn-in 为 \(T_{\mathrm{burn}}\),箱体数 \(N_x\times N_y\)。
	\end{itemize}
	
	\paragraph{(1) 平稳密度的几何形状与慢快结构}
	数值平稳密度 \(\hat p_\infty(x,y)\) 往往沿慢流形附近呈现“带状集中”,并在靠近转折区域出现明显的概率聚集。
	当 \(a\) 位于 Canard 爆炸临界附近时,随机扰动会使轨道在“小振幅振荡区域”与“大振幅回转区域”之间产生随机切换,
	从密度角度通常表现为:
	\begin{itemize}
		\item 在相图中出现两类高概率区域(或一条高概率通道);
		\item 当 \(\sigma\) 增大时,密度从尖锐集中逐渐变得更弥散;
		\item 当 \(\eta\) 逐渐减小(更接近临界)时,密度沿大回转方向的尾部显著增强,体现“爆炸”概率增加。
	\end{itemize}
	
	\paragraph{(2) 与时间离散方法的对比(可选,用于投稿时强调优势)}
	为突出本文方法在长时间统计上的优势,可增加一组对照:使用 Euler--Maruyama(或 tamed/truncated EM)
	在相同总物理时间 \(T\) 下统计密度,并报告:
	\[
	\text{CPU 时间},\qquad
	\|\hat p_{\mathrm{TL}}-\hat p_{\mathrm{ref}}\|_{L^1},\qquad
	\|\hat p_{\mathrm{EM}}-\hat p_{\mathrm{ref}}\|_{L^1},
	\]
	其中 \(\hat p_{\mathrm{ref}}\) 可由更细网格/更小步长或更长时间 SSA 得到。
	在慢快 Canard 系统中,通常可观察到:\textit{tau-leaping} 在保持可接受密度误差的同时显著减少事件推进次数,
	尤其在需要极长时间采样以稳定直方图时更为明显。
	
	\begin{figure}[htbp]
		\centering
		%\includegraphics[width=0.48\linewidth]{canard_phase_paths.pdf}
		%\includegraphics[width=0.48\linewidth]{canard_density_heatmap.pdf}
		\caption{随机Canard系统的样本轨道与平稳密度示意图(左:相平面轨道;右:二维平稳密度热力图)。图由程序生成。}
		\label{fig:canard_density}
	\end{figure}
	
	\subsection{小结:平稳密度与Canard爆炸的联系}
	
	从统计角度看,Canard 爆炸可理解为在极窄参数区间内,系统对轨道振幅的“选择”发生突变。
	加入噪声后,这种突变被随机化为“概率分配”的快速改变:
	当参数靠近临界时,平稳密度在大振幅回转区域的占据概率迅速上升。
	因此,\(\hat p_\infty(x,y)\) 提供了一种定量刻画 Canard 爆炸随机化效应的手段:
	不仅能观察轨道几何结构,还能比较不同参数/噪声强度下“大回转事件”的发生概率与占据时间。
	
	%=========================================================
	% (可选)在参考文献中补充 Canard 经典文献占位:
	% \bibitem{Benoit}
	% E. Benoît et al., ... (Canard solutions in slow-fast systems)
	% \bibitem{KrupaSzmolyan}
	% M. Krupa and P. Szmolyan, ... (Extending geometric singular perturbation theory)
	%=========================================================
	\section{模型与快慢分解}
	
	考虑二维快慢系统($\varepsilon=\delta\in(0,1)$ 很小):
	\begin{equation}\label{eq:model}
		\varepsilon \dot x = y - f(x),\qquad \dot y = a-x,
		\qquad f(x)=\frac{x^3}{3}-x .
	\end{equation}
	
	\subsection{快时间与层系统(layer problem)}
	引入快时间 $\tau=t/\varepsilon$,记 $\frac{d}{d\tau}(\cdot)=(\cdot)'$,则
	\begin{equation}\label{eq:layer}
		x' = y-f(x),\qquad y'=0 .
	\end{equation}
	因此在快时间尺度上 $y$ 可视为参数;$x$ 很快被吸引/排斥到快子系统的平衡点集合。
	
	\subsection{临界流形(critical manifold)与稳定性}
	层系统平衡点满足 $y=f(x)$,得到临界流形
	\begin{equation}\label{eq:C0}
		\mathcal C_0=\{(x,y): y=f(x)\}.
	\end{equation}
	线性化:$\partial_x(y-f(x)) = -f'(x)=-(x^2-1)$。
	因此
	\[
	|x|>1\Rightarrow f'(x)>0 \Rightarrow \partial_x(y-f(x))<0 \ \text{(吸引支)},\qquad
	|x|<1\Rightarrow \partial_x(y-f(x))>0 \ \text{(排斥支)}.
	\]
	折点(saddle--node/ fold)由 $f'(x)=0$ 给出:
	\begin{equation}\label{eq:folds}
		x=\pm 1,\qquad y=f(\pm 1)=\mp\frac{2}{3}.
	\end{equation}
	
	\section{慢系统(reduced problem)与 Fenichel 慢流形}
	
	\subsection{慢极限($\varepsilon\to0$)上的约化方程}
	在慢时间 $t$ 下,令 $\varepsilon=0$,得到代数约束 $y=f(x)$,并且
	\[
	\dot y = f'(x)\dot x.
	\]
	由 $\dot y=a-x$ 得到约化慢方程(在 $\mathcal C_0$ 上):
	\begin{equation}\label{eq:reduced}
		\dot x = \frac{a-x}{f'(x)}=\frac{a-x}{x^2-1},
		\qquad y=f(x).
	\end{equation}
	注意在折点 $x=\pm1$ 处分母为 $0$,意味着慢流在折点处“失效”,轨道会发生快跳。
	
	\subsection{Fenichel 理论给出的 $\mathcal C_\varepsilon^{a/r}$}
	当 $\varepsilon>0$ 足够小,远离折点的吸引支与排斥支会分别扰动为
	\[
	\mathcal C_\varepsilon^{a},\ \mathcal C_\varepsilon^{r},
	\]
	并保持与 $\mathcal C_0$ 同胚(光滑)且具有指数吸引/排斥性质(这就是 Canard 理论的几何骨架)。
	(Sowers 用一般三次型 $f$ 将 $\mathcal C_0$ 分成稳定支 $S_L,S_R$ 与不稳定支 $U$:稳定与不稳定的划分思想与此一致。)%
	% 资料来源(一般三次型的慢流形分支划分)::contentReference[oaicite:0]{index=0}
	
	\section{确定性动力学:平衡点、松弛振荡、鸭解(regular duck / headless duck)}
	
	\subsection{平衡点与线性稳定性}
	由 \eqref{eq:model} 得平衡点
	\[
	x^\ast=a,\qquad y^\ast=f(a)=\frac{a^3}{3}-a.
	\]
	Jacobi 矩阵为
	\[
	J=
	\begin{pmatrix}
		\frac{1}{\varepsilon}(1-a^2) & \frac{1}{\varepsilon}\\[3pt]
		-1 & 0
	\end{pmatrix},
	\quad
	\mathrm{tr}(J)=\frac{1-a^2}{\varepsilon},\quad
	\det(J)=\frac{1}{\varepsilon}>0.
	\]
	故当 $|a|>1$ 时 $\mathrm{tr}(J)<0$,平衡点稳定;当 $|a|<1$ 时平衡点不稳定。
	特别地,$a=\pm1$ 是临界(“Hopf 型临界”)位置,并且在本快慢问题中它恰好与折点重合($a=1$ 对应 $(1,-2/3)$,$a=-1$ 对应 $(-1,2/3)$),这正是 Canard 爆炸发生的典型几何情形。
	
	\subsection{为什么会出现松弛振荡(relaxation oscillation)?}
	当 $|a|<1$ 时,平衡点落在排斥区(因为 $|x^\ast|=|a|<1$),轨道不能收敛到平衡点。
	在 $\varepsilon\ll1$ 时,典型轨道呈现“两段慢爬 + 两次快跳”的结构:
	
	\begin{itemize}
		\item 在吸引慢流形 $\mathcal C_\varepsilon^{a}$ 上按 \eqref{eq:reduced} 慢慢移动;
		\item 接近折点 $x=1$(或 $x=-1$)时,吸引慢流形终止,轨道在快时间尺度上沿层系统 \eqref{eq:layer} 快速跳到另一条吸引支附近;
		\item 重复上述过程形成闭合周期轨道(松弛振荡)。
	\end{itemize}
	
	\paragraph{奇异极限下的“跳跃落点”可以显式写出。}
	例如到达右折点 $(1,-2/3)$ 时,$y$ 近似保持 $y=-2/3$ 不变,快方程把 $x$ 拉到 $y=f(x)$ 的稳定根。
	解 $f(x)=-2/3$:
	\[
	\frac{x^3}{3}-x=-\frac{2}{3}
	\iff x^3-3x+2=0
	\iff (x-1)^2(x+2)=0.
	\]
	因此从 $x=1$ 会快跳到另一稳定根 $x=-2$(而不是停留在重根 $x=1$)。
	同理,在左折点 $(-1,2/3)$ 处解 $f(x)=2/3$:
	\[
	x^3-3x-2=0 \iff (x+1)^2(x-2)=0,
	\]
	故从 $x=-1$ 会快跳到 $x=2$。
	
	\paragraph{周期的慢段可由积分近似。}
	在右吸引支上 $x$ 从 $2$ 慢走到 $1$;在左吸引支上 $x$ 从 $-2$ 慢走到 $-1$。
	由 \eqref{eq:reduced} 得慢段时间近似
	\begin{equation}\label{eq:period}
		T \approx \int_{1}^{2}\frac{x^2-1}{x-a}\,dx \;+\; \int_{-2}^{-1}\frac{x^2-1}{a-x}\,dx,
		\qquad (|a|<1,\ \varepsilon\ll1).
	\end{equation}
	这体现了松弛振荡的“长时间慢漂移 + 短时间快跳”。
	
	\subsection{为什么会出现 regular duck 与 headless duck?(Canard 几何机制)}
	
	\paragraph{核心几何:排斥慢流形附近的“指数放大”。}
	当参数 $a$ 接近临界值(尤其在 $a\approx 1$ 且 $\varepsilon\ll1$)时,轨道可能在排斥慢流形 $\mathcal C_\varepsilon^{r}$ 附近停留一段时间。
	在排斥支上,快方向的线性化增长率约为
	\[
	\dot \xi \approx \frac{1}{\varepsilon}(1-x^2)\,\xi,
	\]
	而在 $|x|<1$ 时 $1-x^2>0$,因此任何微小偏差 $\xi$ 会以 $e^{c/\varepsilon}$ 的量级被放大。
	这导致一个著名现象:$a$ 只需发生极微小(常常是指数小窗口)的变化,周期轨道的幅值就会从“小振幅”突然跃迁到“松弛大振幅”(Canard explosion)。
	
	\paragraph{regular duck vs headless duck(用“跳回哪条稳定支”区分)。}
	沿排斥支走一段后,轨道终究会“跑不动”并从排斥区脱离,回到某条吸引支。
	如果它回到左吸引支,就得到 \textbf{regular duck};若回到右吸引支,就得到 \textbf{headless duck}。
	Sowers 在一般三次型快慢系统中用图像语言明确指出:轨道在排斥慢流形 $U$ 上“绕行”后,最终回到稳定慢流形 $S$;
	回到 $S_L$ 是 regular duck,回到 $S_R$ 是 headless duck。%
	% 资料来源(regular/headless 的定义性描述)::contentReference[oaicite:1]{index=1}
	
	\paragraph{与参数 $a,\varepsilon$ 的关系(定性结论)。}
	\begin{itemize}
		\item 当 $|a|>1$:平衡点稳定,轨道通常收敛到平衡点(不振荡)。
		\item 当 $|a|<1$ 且远离临界:存在稳定松弛振荡(大振幅)。
		\item 当 $a$ 位于靠近 $1$(或 $-1$)的极窄窗口:出现 Canard 轨道;
		窗口内随 $a$ 变化会出现 headless duck $\leftrightarrow$ regular duck $\leftrightarrow$ relaxation oscillation 的快速切换(Canard 爆炸)。
	\end{itemize}
	
	\paragraph{(可用于写作的具体参数展开)}
	经典 van der Pol/Canard 理论给出“canard 点”参数 $a=a_c(\varepsilon)$ 的渐近展开(你的代码里使用过):
	\[
	a_c(\varepsilon)=1-\frac{\varepsilon}{8}-\frac{3\varepsilon^2}{32}-\frac{173\varepsilon^3}{1024}+\cdots,
	\]
	在 $a$ 穿过 $a_c(\varepsilon)$ 的极窄区间时,轨道从 headless 转向 regular,并迅速爆炸到松弛振荡。
	(该展开的严格推导通常用 blow-up/匹配渐近或正则形化方法完成,见参考文献中的 Krupa--Szmolyan 等。)
	
	\section{加入噪声后的 Canard:慢变量噪声 $\Rightarrow$ 随机选择;快变量噪声 $\Rightarrow$ 随机共振}
	
	\subsection{两种加噪方式}
	对 \eqref{eq:model} 常见的两种加性白噪声版本为:
	
	\paragraph{(A) 噪声加在慢变量($y$)上:}
	\begin{equation}\label{eq:slow-noise}
		dx_t=\frac{1}{\varepsilon}\bigl(y_t-f(x_t)\bigr)\,dt,\qquad
		dy_t=(a-x_t)\,dt+\sigma\,dW_t .
	\end{equation}
	
	\paragraph{(B) 噪声加在快变量($x$)上:}
	\begin{equation}\label{eq:fast-noise}
		dx_t=\frac{1}{\varepsilon}\bigl(y_t-f(x_t)\bigr)\,dt+\frac{\sigma}{\varepsilon}\,dW_t,\qquad
		dy_t=(a-x_t)\,dt .
	\end{equation}
	
	Sowers 明确说明:本文主要研究噪声进入慢变量时导致“regular vs headless”的\textbf{随机选择};
	而把噪声放进快变量会出现\textbf{随机共振}等不同现象,并引用相关工作。%
	% 资料来源(慢变量噪声导致随机选择的研究目标;快变量噪声可致随机共振的说明)::contentReference[oaicite:2]{index=2}:contentReference[oaicite:3]{index=3}
	
	\subsection{慢变量噪声为什么会导致“随机选择”?}
	
	\subsubsection{机制 1:排斥慢流形上的指数放大把噪声“放大成选择”}
	在 canard 瓶颈区域,轨道必须贴着排斥慢流形 $\mathcal C_\varepsilon^{r}$ 走一段。
	设与排斥慢流形的快向偏差为 $\xi$,其主导项满足(见上一节)
	\[
	\dot \xi \approx \frac{\lambda(t)}{\varepsilon}\xi,\qquad \lambda(t)>0.
	\]
	因此任何由噪声引入的、哪怕极微小的随机偏差,在时间推进中都会被放大为 $O(1)$ 的差异,
	从而决定轨道最终回到哪条吸引支($S_L$ 或 $S_R$),即决定 \textbf{regular duck 或 headless duck}。
	这就是“随机选择”的本质:\textbf{确定性系统中由参数决定的极其敏感的分界,被噪声模糊成概率分界。}
	
	\subsubsection{机制 2:Sowers 的极限定理——分界量的符号趋于高斯分布}
	Sowers 构造了刻画“靠近慢流形时偏离量”的量 $D_\varepsilon(\cdot)$,
	并在合适缩放下证明它的极限分布含有高斯项,因而轨道以某个概率趋向 $+\infty$ 或 $-\infty$,
	对应“落到左支/右支”的随机选择概率(高斯分布函数 $\Phi$ 给出)。
	例如在其结论之一中,极限概率写成 $\int_{-\infty}^{m} e^{-z^2/2}dz$ 的形式。%
	% 资料来源(极限分布含高斯项,从而得到趋向 ±∞ 的概率公式)::contentReference[oaicite:4]{index=4}
	这正是“随机选择(random decision/selection)”的严格概率论版本:\textbf{在 canard 分界附近,轨道去 regular 或 headless 不再由参数单独决定,而是由噪声驱动的随机变量决定。}
	
	\subsubsection{机制 3:折点附近的局部正则形(Berglund--Gentz 的样本路径推导)}
	Berglund--Gentz 在 van der Pol 型系统中研究了\emph{慢变量加噪}:
	\begin{equation}\label{eq:BG-619}
		dx_t=\frac{1}{\varepsilon}\Bigl(y_t+x_t-\frac{x_t^3}{3}\Bigr)dt,\qquad
		dy_t=-x_t\,dt+\sigma\,dW_t .
	\end{equation}
	% 资料来源(式 (6.1.19))::contentReference[oaicite:5]{index=5}
	
	它的关键动力学发生在折点(鞍结分岔点)附近。以右折点 $(x_c,y_c)=(1,-2/3)$ 为例,
	作局部变换(其符号选择使正则形更标准):
	\[
	(\tilde x_t,\tilde y_t)=(x_t-x_c,\; -(y_t-y_c)).
	\]
	在折点邻域可化为正则形的小扰动:
	\begin{equation}\label{eq:BG-620}
		d\tilde x_t=\frac{1}{\varepsilon}\bigl(-\tilde y_t-\tilde x_t^2\bigr)\,dt,\qquad
		d\tilde y_t=dt+\sigma\,dW_t .
	\end{equation}
	% 资料来源(式 (6.1.20))::contentReference[oaicite:6]{index=6}
	
	\paragraph{“贴慢流形的绝热解”与偏差方程。}
	确定性情形 $\sigma=0$ 存在绝热解 $\bar x(\tilde y,\varepsilon)$ 追踪慢流形 $x^\ast(y)=|y|^{1/2}$,
	并且当 $\tilde y\lesssim -\varepsilon^{2/3}$ 时偏离量随 $|\tilde y|$ 变化具有可控估计。
	定义噪声诱导偏差
	\[
	\xi_t=\tilde x_t-\bar x(\tilde y_t,\varepsilon),
	\]
	则其满足(含漂移、非线性项与有效噪声强度):
	\begin{equation}\label{eq:BG-621}
		d\xi_t=\frac{1}{\varepsilon}\Bigl(a(\tilde y_t)\xi_t-\xi_t^2-\frac{1}{2}\sigma^2\varepsilon\,\partial_{yy}\bar x(\tilde y_t,\varepsilon)\Bigr)\,dt
		+\sigma\,g(\tilde y_t)\,dW_t,
	\end{equation}
	其中
	\begin{equation}\label{eq:BG-622-623}
		a(y)=-2\bar x(y,\varepsilon),\qquad g(y)=-\partial_y\bar x(y,\varepsilon).
	\end{equation}
	并且在 $y\lesssim -\varepsilon^{2/3}$ 时
	\[
	a(y)\sim -|y|^{1/2},\qquad g(y)\sim |y|^{-1/2}.
	\]
	% 资料来源(式 (6.1.21)--(6.1.23) 及渐近行为)::contentReference[oaicite:7]{index=7}
	
	\paragraph{线性近似下的方差增长:解释“为何折点附近噪声影响被放大”。}
	取线性近似
	\begin{equation}\label{eq:BG-624}
		d\xi^{0}_t=\frac{1}{\varepsilon}a(\tilde y^{\mathrm{det}}_t)\xi^{0}_t\,dt+\sigma\,g(\tilde y^{\mathrm{det}}_t)\,dW_t,
	\end{equation}
	其方差增长满足
	\[
	\mathrm{Var}(\xi^{0}_t)\sim \frac{\sigma^2\varepsilon}{|\tilde y^{\mathrm{det}}_t|^{3/2}},
	\qquad
	\text{典型扩散尺度 }\ |\xi|\sim \frac{\sigma\,\varepsilon^{1/2}}{|\tilde y|^{3/4}}.
	\]
	% 资料来源(方差与典型扩散尺度)::contentReference[oaicite:8]{index=8}
	
	这条公式直接告诉你:当轨道慢慢接近折点($|\tilde y|\to0$)时,
	即使 $\sigma$ 固定不大,快方向的随机扩散尺度也会被 $|\tilde y|^{-3/4}$ 放大,
	从而更容易触发“提前/延后跳跃”以及在 canard 分界附近的“落左/落右”的随机选择。
	
	\paragraph{与“regular/headless”的连接(写作建议)。}
	在 canard 窗口内,确定性系统中“落左/落右”对参数 $a$ 极端敏感;
	而 \eqref{eq:BG-624} 的扩散放大意味着:在折点/排斥段附近,轨道会获得一个
	近似高斯的横向随机偏移,随后又被排斥动力学指数放大,
	最终把“横向偏移的符号/大小”转化成“落到 $S_L$ 或 $S_R$ 的概率”——这正是随机选择的数学机制。
	
	\subsection{快变量噪声为什么会导致“随机共振(stochastic resonance)”?}
	
	\subsubsection{随机共振的标准定义}
	随机共振通常指:系统在\textbf{弱周期信号}(或慢调制)作用下,本来难以跨越阈值/势垒发生切换;
	加入\textbf{适当强度}的噪声后,切换事件与周期信号产生最强同步,导致输出的信号-噪声比或谱峰在某个噪声强度处达到最大。
	
	在快慢系统中,慢变量(或慢流形几何)往往提供一个缓慢变化的“势垒高度/阈值位置”;
	而当噪声直接作用于快变量时,快变量在短时间内被噪声“踢过”分界(例如跨过不稳定支或在折点前提前跳跃),
	于是产生与慢调制相关的“近周期切换”。当噪声太小,几乎不切换;噪声太大,切换太随机;
	中间某个噪声强度使平均切换时间与慢时间尺度(或外加周期)匹配,从而出现“共振”。
	
	\subsubsection{用 Kramers 型速率解释“为何存在最优噪声强度”}
	在许多等效一维阈值/势垒模型中,跨越事件的速率可近似为
	\[
	r(t)\approx C(t)\exp\Bigl(-\frac{\Delta V(t)}{\sigma^2}\Bigr),
	\]
	其中 $\Delta V(t)$ 随慢变量缓慢变化(慢变量相当于对势垒做慢调制)。
	当慢调制有特征周期 $T_{\mathrm{slow}}$(例如松弛振荡或外部微弱周期输入),
	随机共振常对应
	\[
	\mathbb E[\tau_{\mathrm{switch}}]\approx \frac{T_{\mathrm{slow}}}{2}
	\quad\Longleftrightarrow\quad
	r(\sigma)\approx \frac{2}{T_{\mathrm{slow}}},
	\]
	从而在某个 $\sigma$ 处同步最强、谱峰最明显。
	
	\subsubsection{与 Canard/van der Pol 几何的直接对应}
	对 \eqref{eq:fast-noise},噪声项在 $x$ 方程中被 $1/\varepsilon$ 放大(若仍用加性白噪写法),
	这意味着快方向的随机扰动会更直接、更强烈地影响“何时离开吸引支、何时跨过不稳定区”,
	从而把“慢变量的缓慢变化”转化为“随机切换事件的相位锁定”。
	因此快变量噪声下更典型地观察到随机共振/相干共振等现象,而不是像慢变量噪声那样主要表现为
	“regular vs headless 的随机选择”。
	(Sowers 也指出:把噪声放入快变量会导致随机共振等不同现象,并引相关文献。)%
	% 资料来源(快变量噪声可致随机共振的说明)::contentReference[oaicite:9]{index=9}
	
	\section{在快变量和慢变量上分别加噪声}
	
	
	\begin{enumerate}
		\item \textbf{模型层}:写出慢变量加噪的 van der Pol SDE \eqref{eq:BG-619},并强调“关键发生在折点”。%
		% 资料来源:式 (6.1.19) 与折点重要性叙述:contentReference[oaicite:10]{index=10}
		\item \textbf{局部正则形层}:给出折点邻域正则形 \eqref{eq:BG-620},指出它是鞍结正则形的随机扰动。%
		% 资料来源:式 (6.1.20):contentReference[oaicite:11]{index=11}
		\item \textbf{样本路径估计层}:给出偏差 SDE \eqref{eq:BG-621}--\eqref{eq:BG-624} 与方差/扩散尺度结论,
		用 $|\tilde y|^{-3/4}$ 的放大量解释“为什么接近折点时噪声效果突然变强”,再把它与 canard 分界的指数敏感性结合,
		得到“随机选择”的机制闭环。%
		% 资料来源:式 (6.1.21)--(6.1.24) 与方差结论:contentReference[oaicite:12]{index=12}
	\end{enumerate}
	
	
	\begin{thebibliography}{99}
		
		\bibitem{Sowers2008}
		R.\ B.\ Sowers,
		\newblock \textit{Random Perturbations of Canards},
		\newblock Journal of Theoretical Probability, 21 (2008), 824--889.
		% 你上传的 PDF:Random Perturbations of Canards.pdf
		% 相关出处:随机选择(摘要):contentReference[oaicite:13]{index=13};regular/headless 定义:contentReference[oaicite:14]{index=14};
		% 快变量噪声与随机共振提示:contentReference[oaicite:15]{index=15}。
		
		\bibitem{BerglundGentzBook}
		N.\ Berglund and B.\ Gentz,
		\newblock \textit{Noise-Induced Phenomena in Slow-Fast Dynamical Systems: A Sample-Paths Approach},
		\newblock Springer, 2006.
		% 你上传的 4 页摘录:式 (6.1.19)--(6.1.24) 与方差结论:contentReference[oaicite:16]{index=16}。
		
		\bibitem{Benoit1981}
		E.\ Beno\^{\i}t, J.-L.\ Callot, F.\ Diener, and M.\ Diener,
		\newblock \textit{Chasse au canard},
		\newblock Collectanea Mathematica, 32 (1981), 37--119.
		% Canard 经典起源文献。
		
		\bibitem{KrupaSzmolyan}
		M.\ Krupa and P.\ Szmolyan,
		\newblock \textit{Extending Geometric Singular Perturbation Theory to Nonhyperbolic Points---Fold and Canard Points},
		\newblock (系列论文,2001--2004 年间多篇;用于 canard 点展开与严格几何证明).
		
		\bibitem{Gammaitoni1998}
		L.\ Gammaitoni, P.\ H\"anggi, P.\ Jung, and F.\ Marchesoni,
		\newblock \textit{Stochastic Resonance},
		\newblock Reviews of Modern Physics, 70 (1998), 223--287.
		% 随机共振综述(给出标准定义、Kramers 速率解释与 SNR 最优噪声强度思想)。
		
	\end{thebibliography}
	
	
	\end{document}

